\IfFileExists{t2doc.cls}{%
    \documentclass[documentation]{subfiles}
}{
    \errmessage{Error: could not find t2doc.cls}
}

\begin{document}

\trantitle
    {voipDetector}
    {VoIP}
    {Tranalyzer Development Team}

\section{voipDetector}\label{s:voipDetector}

\subsection{Description}
The idea of this plugin is to identify SIP, RTP and RTCP flows independently of each other, so
that also non standard traffic can be detected. Moreover certain QoS values are extracted.

\subsection{Configuration Flags}
The following flags can be used to control the output of the plugin:

\begin{longtable}{lcll}
    \toprule
    {\bf Variable} & {\bf Default} & {\bf Description} & {\bf Flags}\\
    \midrule\endhead%
    {\tt VOIP\_ANALEN}  &  1 & 1: additional check report len against payload length\\
                        &    & 0: only ssrc check\\
    {\tt VOIP\_V\_SAVE} &  0 & save rtp content to VOIP\_RM\_DIR \\
    {\tt VOIP\_RM\_DIR} &  0 & rm RTP content directory & {\tt\small VOIP\_V\_SAVE=1}\\
    {\tt VOIP\_PLDOFF}  &  0 & offset to payload pointer to save content & {\tt\small VOIP\_V\_SAVE=1}\\
    {\tt SIPNMMAX}      & 40 & maximal sip caller name length in flow file \\
    {\tt VOIP\_PATH}  & {\tt\small ``/tmp/''} & default path of content directory \\
    {\tt VOIP\_FNAME} & {\tt\small ``eier''} & default content file name prefix \\
    \bottomrule
\end{longtable}

%\subsection{Required Files}
%none

\subsection{Flow File Output}
The voipDetector plugin outputs the following columns:

\begin{longtable}{lll}
    \toprule
    {\bf Column} & {\bf Type} & {\bf Description} \\
    \midrule\endhead%
    {\tt \nameref{voipStat}} & H16    & Status \\
    {\tt voipID}             & H32    & RTP/RTCP ID \\
    {\tt voipSRCnt}          & U8     & RTP SID/RTCP record count \\
    {\tt voipTyp}            & U8     & RTP/RTCP type \\
    {\tt voipPMCnt}          & U32    & RTP packet miss count \\
    {\tt voipPMr}            & F      & RTP packet miss ratio \\
    \\
    {\tt voipSIPStatCnt}     & U8     & SIP stat count \\
    {\tt voipSIPReqCnt}      & U8     & SIP request count \\
    {\tt voipSIPCID}         & S      & SIP Call ID \\
    {\tt voipSIPStat}        & R(U16) & SIP stat \\
    {\tt voipSIPReq}         & R(S)   & SIP request \\
    \\
    {\tt voipTPCnt}          & U32    & RTCP cumulated transmitter packet count \\
    {\tt voipTBCnt}          & U32    & RTCP cumulated transmitter byte count \\
    {\tt voipCPMCnt}         & U32    & RTCP cumulated packet miss count\\
    {\tt voipMaxIAT}         & U32    & RTCP maximal Inter Arrival Time \\
    \bottomrule
\end{longtable}

\subsubsection{voipStat}\label{voipStat}
The {\tt voipStat} column is to be interpreted as follows:

\begin{longtable}{rll}
    \toprule
    {\bf voipStat} & {\bf Name} & {\bf Description}\\
    \midrule\endhead%
    $2^{0}$  (={\tt 0x0001}) & RTP     & RTP detected \\
    $2^{1}$  (={\tt 0x0002}) & RTCP    & RTCP detected \\
    $2^{2}$  (={\tt 0x0004}) & SIP     & SIP detected \\
    $2^{3}$  (={\tt 0x0008}) & STUN    & STUN present \\
    $2^{4}$  (={\tt 0x0010}) & X       & RTP: extension header  \\
    $2^{5}$  (={\tt 0x0020}) & P       & RTP: padding present \\
    $2^{6}$  (={\tt 0x0040}) & -       & -\\
    $2^{7}$  (={\tt 0x0080}) & M       & RTP: data marker set\\
    $2^{8}$  (={\tt 0x0100}) & WROP    & RTP: content write operation \\
    $2^{9}$  (={\tt 0x0200}) & -       & -\\
    $2^{10}$ (={\tt 0x0400}) & -       & -\\
    $2^{11}$ (={\tt 0x0800}) & -       & -\\
    $2^{12}$ (={\tt 0x1000}) & PKTLSS  & RTP: packet loss detected \\
    $2^{13}$ (={\tt 0x2000}) & RTPNFRM & RTP: new frame header flag \\
    $2^{14}$ (={\tt 0x4000}) & -       & -\\
    $2^{15}$ (={\tt 0x8000}) & -       & -\\
    \bottomrule
\end{longtable}

\subsection{TODO}

\begin{itemize}
    \item Skype
    \item Google Talk
\end{itemize}

\end{document}
