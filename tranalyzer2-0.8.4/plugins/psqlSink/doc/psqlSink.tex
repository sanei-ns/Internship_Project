\IfFileExists{t2doc.cls}{
    \documentclass[documentation]{subfiles}
}{
    \errmessage{Error: could not find 't2doc.cls'}
}

\begin{document}

\trantitle
    {psqlSink} % Plugin name
    {PostgreSQL} % Short description
    {Tranalyzer Development Team} % author(s)

\section{psqlSink}\label{s:psqlSink}

\subsection{Description}
The psqlSink plugin outputs flow files to PostgreSQL database.

\subsection{Dependencies}

\subsubsection{External Libraries}
This plugin depends on the {\bf libpq} library.
\paragraph{Ubuntu:} {\tt sudo apt-get install libpq-dev}
\paragraph{Arch:} {\tt sudo pacman -S postgresql-libs}
\paragraph{Mac OS X:} {\tt brew install postgresql}

\subsection{Configuration Flags}
The following flags can be used to control the output of the plugin:
\begin{longtable}{lcl}
    \toprule
    {\bf Name} & {\bf Default} & {\bf Description}\\
    \midrule\endhead%
    {\tt PSQL\_OVERWRITE\_DB}       & 2                  & 0: abort if DB already exists\\
                                    &                    & 1: overwrite DB if it already exists\\
                                    &                    & 2: reuse DB if it already exists\\
    {\tt PSQL\_OVERWRITE\_TABLE}    & 2                  & 0: abort if table already exists\\
                                    &                    & 1: overwrite table if it already exists\\
                                    &                    & 2: append to table if it already exists\\
    {\tt PSQL\_TRANSACTION\_NFLOWS} & 40000              & 0: one transaction\\
                                    &                    & > 0: one transaction every $n$ flows\\
    {\tt PSQL\_QRY\_LEN}            & 32768              & Max length for query\\
    {\tt PSQL\_HOST}                & {\tt "127.0.0.1"}  & Address of the database\\
    {\tt PSQL\_PORT}                & 5432               & Port of the database\\
    {\tt PSQL\_USER}                & {\tt "postgres"}   & Username to connect to DB\\
    {\tt PSQL\_PASS}                & {\tt "postgres"}   & Password to connect to DB\\
    {\tt PSQL\_DBNAME}              & {\tt "tranalyzer"} & Name of the database\\
    {\tt PSQL\_TABLE\_NAME}         & {\tt "flow"}       & Name of the table\\
    \bottomrule
\end{longtable}

\subsection{Post-Processing}
The following queries can be used to analyze bitfields in PostgreSQL:
\begin{itemize}
    \item Select all A flows:\\
        \begin{ttfamily}
            \textcolor{darkblue}{\bf SELECT} to\_hex(\textcolor{red}{"flowStat"}::\textcolor{darkblue}{bigint}), *\\
            \textcolor{darkblue}{\bf FROM} flow\\
            \textcolor{darkblue}{\bf WHERE} (\textcolor{red}{"flowStat"}::\textcolor{darkblue}{bigint} \& \textcolor{cyan}{1}) = \textcolor{cyan}{0}::\textcolor{darkblue}{bigint}
        \end{ttfamily}
    \item Select all IPv4 flows:\\
        \begin{ttfamily}
            \textcolor{darkblue}{\bf SELECT} *\\
            \textcolor{darkblue}{\bf FROM} flow\\
            \textcolor{darkblue}{\bf WHERE} (\textcolor{red}{"flowStat"}::\textcolor{darkblue}{bigint} \& x`\textcolor{red}{4000}'::\textcolor{darkblue}{bigint}) != \textcolor{cyan}{0}::\textcolor{darkblue}{bigint}
        \end{ttfamily}
    \item Select all IPv6 flows:\\
        \begin{ttfamily}
            \textcolor{darkblue}{\bf SELECT} to\_hex(\textcolor{red}{"flowStat"}::\textcolor{darkblue}{bigint}), *\\
            \textcolor{darkblue}{\bf FROM} flow\\
            \textcolor{darkblue}{\bf WHERE} (\textcolor{red}{"flowStat"}::\textcolor{darkblue}{bigint} \& x`\textcolor{red}{8000}'::\textcolor{darkblue}{bigint}) != \textcolor{cyan}{0}::\textcolor{darkblue}{bigint}
        \end{ttfamily}
\end{itemize}

\end{document}
