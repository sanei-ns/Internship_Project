\IfFileExists{t2doc.cls}{
    \documentclass[documentation]{subfiles}
}{
    \errmessage{Error: could not find 't2doc.cls'}
}

\begin{document}

\trantitle
    {mongoSink} % Plugin name
    {MongoDB} % Short description
    {Tranalyzer Development Team} % author(s)

\section{mongoSink}\label{s:mongoSink}

\subsection{Description}
The mongoSink plugin outputs flow files to MongoDB.

\subsection{Dependencies}

\subsubsection{External Libraries}
This plugin depends on the {\bf libmongoc} library.
\paragraph{Ubuntu:} {\tt sudo apt-get install libmongoc-dev}
\paragraph{Arch:} {\tt sudo pacman -S mongo-c-driver}
\paragraph{Mac OS X:} {\tt brew install mongo-c-driver}

\subsection{Configuration Flags}
The following flags can be used to control the output of the plugin:
\begin{longtable}{lcl}
    \toprule
    {\bf Name} & {\bf Default} & {\bf Description}\\
    \midrule\endhead%
    {\tt MONGO\_QRY\_LEN}    & 2048               & Max length for query\\
    {\tt MONGO\_HOST}        & {\tt "127.0.0.1"}  & Address of the database\\
    {\tt MONGO\_PORT}        & {\tt "27017"}      & Port the database is listening to\\
    {\tt MONGO\_DBNAME}      & {\tt "tranalyzer"} & Name of the database\\
    {\tt MONGO\_TABLE\_NAME} & {\tt "flow"}       & Name of the database flow table\\
    {\tt MONGO\_NUM\_DOCS}   & 1                  & Number of documents (flows) to write in bulk\\\\

    {\tt BSON\_SUPPRESS\_EMPTY\_ARRAY} & 1 & Whether or not to output empty fields\\
    {\tt BSON\_DEBUG}                  & 0 & Print debug messages\\
    \bottomrule
\end{longtable}

\end{document}
