\IfFileExists{t2doc.cls}{
    \documentclass[documentation]{subfiles}
}{
    \errmessage{Error: could not find 't2doc.cls'}
}

\begin{document}

\trantitle
    {binSink}
    {Binary Output}
    {Tranalyzer Development Team}

\section{binSink}\label{s:binSink}

\subsection{Description}
The binSink plugin is one of the basic output plugin for Tranalyzer2. It uses the output prefix ({\tt --w} option) to generate a binary flow file with suffix {\tt \_flows.bin}. All standard output from every plugin is stored in binary format in this file.

\subsection{Dependencies}

\subsubsection{External Libraries}
If gzip compression is activated ({\tt GZ\_COMPRESS=1}), then {\bf zlib} must be installed.

\paragraph{Kali/Ubuntu:} {\tt sudo apt-get install zlib1g-dev}
\paragraph{Arch:} {\tt sudo pacman -S zlib}
\paragraph{Fedora/Red Hat:} {\tt sudo yum install zlib-devel}
\paragraph{Gentoo:} {\tt sudo emerge zlib}
\paragraph{OpenSUSE:} {\tt sudo zypper install zlib-devel}
\paragraph{Mac OS X:} {\tt brew install zlib}\footnote{Brew is a packet manager for Mac OS X that can be found here: \url{https://brew.sh}}

\subsection{Configuration Flags}

The following flags can be used to control the output of the plugin:

\begin{longtable}{lcl}
    \toprule
    {\bf Name} & {\bf Default} & {\bf Description} \\
    \midrule\endhead%
    {\tt GZ\_COMPRESS} & 0 & Compress the output (gzip)\\
    {\tt SFS\_SPLIT} & 1 & Split the output file (Tranalyzer {\tt --W} option)\\\\
    {\tt FLOWS\_SUFFIX} & {\tt ``\_flows.bin''} & Suffix to use for the output file\\
    {\tt STD\_BUFSHFT} & {\tt BUF\_DATA\_SHFT * 4} & \\
    \bottomrule
\end{longtable}

\subsection{Post-Processing}

\subsubsection{tranalyzer-b2t}
The program {\tt tranalyzer-b2t} can be used to transform binary Tranalyzer files into text or json files.
The converted file uses the same format as the one generated by the \tranrefpl{txtSink} or \tranrefpl{jsonSink} plugin.\\

The program can be found in {\tt\$T2HOME/utils/tranalyzer-b2t/} and can be compiled by typing {\tt make}.\\

The use of the program is straightforward:
\begin{itemize}
    \item bin$\rightarrow$txt: {\tt ./tranalyzer-b2t -r FILE\_flows.bin -w FILE\_flows.txt}\\
    \item bin$\rightarrow$json: {\tt ./tranalyzer-b2t -r FILE\_flows.bin -j -w FILE\_flows.json}\\
\end{itemize}

If the {\tt --w} option is omitted, the destination default to stdout.\\
Additionally, the {\tt --n} option can be used {\bf not} to print the name of the columns as the first row.

\subsection{Custom File Output}
\begin{itemize}
    \item {\tt PREFIX\_flows.bin}: Binary representation of Tranalyzer output
\end{itemize}

\end{document}
