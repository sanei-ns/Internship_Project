\IfFileExists{t2doc.cls}{
    \documentclass[documentation]{subfiles}
}{
    \errmessage{Error: could not find 't2doc.cls'}
}

\begin{document}

\trantitle
    {stpDecode} % Plugin name
    {Spanning Tree Protocol (STP)} % Short description
    {Tranalyzer Development Team} % author(s)

\section{stpDecode}\label{s:stpDecode}

\subsection{Description}
The stpDecode plugin analyzes STP traffic.

%\subsection{Dependencies}

%\traninput{file} % use this command to input files
%\traninclude{file} % use this command to include files

%\tranimg{image} % use this command to include an image (must be located in a subfolder ./img/)

%\subsubsection{External Libraries}
%This plugin depends on the {\bf XXX} library.
%\paragraph{Ubuntu:} {\tt sudo apt-get install XXX}
%\paragraph{Arch:} {\tt sudo pacman -S XXX}
%
%\subsubsection{Other Plugins}
%This plugin requires the {\bf XXX} plugin.
%
%\subsubsection{Required Files}
%The file {\tt file.txt} is required.

%\subsection{Configuration Flags}
%The following flags can be used to control the output of the plugin:
%\begin{longtable}{lcll}
%    \toprule
%    {\bf Name} & {\bf Default} & {\bf Description} & {\bf Flags}\\
%    \midrule\endhead%
%    {\tt FLAG1} & 0 & Whether (1) or not (0) to activate FLAG1\\
%    {\tt OPT2} & 1 & 0: no OPT2, 1: one OPT2, 2: two OPT2 & {\tt FLAG1=1}\\
%    \bottomrule
%\end{longtable}

\subsection{Flow File Output}
The stpDecode plugin outputs the following columns:
\begin{longtable}{llll}
    \toprule
    {\bf Column} & {\bf Type} & {\bf Description}\\
    \midrule\endhead%
    {\tt \nameref{stpStat}}    & H8 & Status \\
    %{\tt \nameref{stpProto}}   & H16 & Protocol Identifier \\ % Always 0
    {\tt \nameref{stpVersion}} & U8 & Protocol Version Identifier \\
    {\tt \nameref{stpType}}    & H8 & Aggregated BPDU Types \\
    {\tt \nameref{stpFlags}}   & H8 & Aggregated BPDU flags \\
    \bottomrule
\end{longtable}

\subsubsection{stpStat}\label{stpStat}
The {\tt stpStat} column is to be interpreted as follows:
\begin{longtable}{rl}
    \toprule
    {\bf stpStat} & {\bf Description}\\
    \midrule\endhead%
    {\tt 0x01} & Flow is STP\\
    \bottomrule
\end{longtable}

\subsubsection{stpProto}\label{stpProto}
The {\tt stpProto} column is to be interpreted as follows:
\begin{longtable}{rl}
    \toprule
    {\bf stpProto} & {\bf Description}\\
    \midrule\endhead%
    {\tt 0x0000} & Spanning Tree Protocol\\
    \bottomrule
\end{longtable}

\subsubsection{stpVersion}\label{stpVersion}
The {\tt stpVersion} column is to be interpreted as follows:
\begin{longtable}{rl}
    \toprule
    {\bf stpVersion} & {\bf Description}\\
    \midrule\endhead%
    0 & Spanning Tree\\
    2 & Rapid Spanning Tree\\
    3 & Multiple Spanning Tree\\
    4 & Shortest Path Tree\\
    \bottomrule
\end{longtable}

\clearpage
\subsubsection{stpType}\label{stpType}
The {\tt stpType} column is to be interpreted as follows:
\begin{longtable}{rl}
    \toprule
    {\bf stpType} & {\bf Description}\\
    \midrule\endhead%
    {\tt 0x00} & Configuration\\
    %{\tt 0x01} & ---\\
    {\tt 0x02} & Rapid/Multiple Spanning Tree\\
    %{\tt 0x04} & ---\\
    %{\tt 0x08} & ---\\
    %{\tt 0x10} & ---\\
    %{\tt 0x20} & ---\\
    %{\tt 0x40} & ---\\
    {\tt 0x80} & Topology Change Notification\\
    \bottomrule
\end{longtable}

\subsubsection{stpFlags}\label{stpFlags}
The {\tt stpFlags} column is to be interpreted as follows:
\begin{longtable}{rl}
    \toprule
    {\bf stpFlags} & {\bf Description}\\
    \midrule\endhead%
    $2^0$ (={\tt 0x0\textcolor{magenta}{1}}) & Topology Change Acknowledgment\\
    $2^1$ (={\tt 0x0\textcolor{magenta}{2}}) & Agreement\\
    $2^2$ (={\tt 0x0\textcolor{magenta}{4}}) & Forwarding\\
    $2^3$ (={\tt 0x0\textcolor{magenta}{8}}) & Learning\\
    $2^4$ (={\tt 0x\textcolor{magenta}{1}0}) & \multirow{2}{*}{Port Role: {\tt 0x00}: Unknown, {\tt 0x10}: Alternate or Backup, {\tt 0x20}: Root, {\tt 0x30}: Designated}\\
    $2^5$ (={\tt 0x\textcolor{magenta}{2}0}) & \\
    $2^6$ (={\tt 0x\textcolor{magenta}{4}0}) & Proposal\\
    $2^7$ (={\tt 0x\textcolor{magenta}{8}0}) & Topology Change\\
    \bottomrule
\end{longtable}

\subsection{Packet File Output}
In packet mode ({\tt --s} option), the stpDecode plugin outputs the following columns:
\begin{longtable}{lll}
    \toprule
    {\bf Column} & {\bf Type} & {\bf Description}\\% & {\bf Flags}\\
    \midrule\endhead%
    {\tt \nameref{stpProto}}   & H16 & Protocol Identifier\\
    {\tt \nameref{stpVersion}} & U8  & Protocol Version Identifier\\
    {\tt \nameref{stpType}}    & H8  & BPDU Type\\
    {\tt \nameref{stpFlags}}   & H8  & BPDU flags\\
    {\tt stpRootPrio}          & U16 & Root Priority\\
    {\tt stpRootHw}            & MAC & Root System ID\\
    {\tt stpRootCost}          & U32 & Root Path Cost\\
    {\tt stpBridgePrio}        & U16 & Bridge Priority\\
    {\tt stpBridgeHw}          & MAC & Bridge System ID\\
    {\tt stpPort}              & H16 & Port Identifier\\
    {\tt stpMsgAge}            & U16 & Message Age\\
    {\tt stpMaxAge}            & U16 & Max Age\\
    {\tt stpHello}             & U16 & Hello Time\\
    {\tt stpForward}           & U16 & Forward Delay\\
	{\tt stpPvstOrigVlan}      & U16 & Originating VLAN (PVSTP+)\\
    \bottomrule
\end{longtable}

\subsection{Plugin Report Output}
The number of STP packets is reported.

%\subsection{Post-Processing}
%
%\subsection{Example Output}
%
%\subsection{Known Bugs and Limitations}
%
%\subsection{TODO}
%\begin{itemize}
%    \item TODO1
%    \item TODO2
%\end{itemize}
%
%\subsection{References}
%\begin{itemize}
%    \item \href{https://tools.ietf.org/html/rfcXXXX}{RFCXXXX}: Title
%    \item \url{https://www.iana.org/assignments/}
%\end{itemize}

\end{document}
