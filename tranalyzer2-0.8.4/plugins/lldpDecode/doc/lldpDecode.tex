\IfFileExists{t2doc.cls}{
    \documentclass[documentation]{subfiles}
}{
    \errmessage{Error: could not find 't2doc.cls'}
}

\begin{document}

\trantitle
    {lldpDecode} % Plugin name
    {Link Layer Discovery Protocol (LLDP)} % Short description
    {Tranalyzer Development Team} % author(s)

\section{lldpDecode}\label{s:lldpDecode}

\subsection{Description}
The lldpDecode plugin analyzes LLDP traffic.

%\subsection{Dependencies}

%\traninput{file} % use this command to input files
%\traninclude{file} % use this command to include files

%\tranimg{image} % use this command to include an image (must be located in a subfolder ./img/)

%\subsubsection{External Libraries}
%This plugin depends on the {\bf XXX} library.
%\paragraph{Ubuntu:} {\tt sudo apt-get install XXX}
%\paragraph{Arch:} {\tt sudo pacman -S XXX}
%
%\subsubsection{Other Plugins}
%This plugin requires the {\bf XXX} plugin.
%
%\subsubsection{Required Files}
%The file {\tt file.txt} is required.

\subsection{Configuration Flags}
The following flags can be used to control the output of the plugin:
\begin{longtable}{lcl}
    \toprule
    {\bf Name} & {\bf Default} & {\bf Description}\\
    \midrule\endhead%
    {\tt LLDP\_TTL\_AGGR} &   1 & Whether (1) or not (0) to aggregate TTL values\\
    {\tt LLDP\_NUM\_TTL}  &   8 & Number of different TTL values to store\\
    {\tt LLDP\_OPT\_TLV}  &   1 & Whether or not to output optional TLVs info\\
    {\tt LLDP\_STRLEN}    & 512 & Maximum length of strings to store\\
    \bottomrule
\end{longtable}

\subsection{Flow File Output}
The lldpDecode plugin outputs the following columns:
\begin{longtable}{llll}
    \toprule
    {\bf Column} & {\bf Type} & {\bf Description} & {\bf Flags}\\
    \midrule\endhead%
    {\tt \nameref{lldpStat}} & H16 & Status & \\
    {\tt lldpChassis} & SC & Chassis ID\\
    {\tt lldpPort} & S & Port ID\\
    {\tt lldpTTL} & RU16 & Time To Live (sec)\\
    {\tt lldpPortDesc} & S & Port description & {\tt LLDP\_OPT\_TLV=1}\\
    {\tt lldpSysName} & S & System name & {\tt LLDP\_OPT\_TLV=1}\\
    {\tt lldpSysDesc} & S & System description & {\tt LLDP\_OPT\_TLV=1}\\
    {\tt \hyperref[lldpCaps]{lldpCaps\_Enabled}} & H16\_H16 & Supported and enabled capabilities & {\tt LLDP\_OPT\_TLV=1}\\
    {\tt lldpMngmtAddr} & SC & Management address & {\tt LLDP\_OPT\_TLV=1}\\
    \bottomrule
\end{longtable}

\subsubsection{lldpStat}\label{lldpStat}
The {\tt lldpStat} column is to be interpreted as follows:
\begin{longtable}{rl}
    \toprule
    {\bf lldpStat} & {\bf Description}\\
    \midrule\endhead%
    {\tt 0x0001} & Flow is LLDP\\
    {\tt 0x0002} & Mandatory TLV missing\\
    {\tt 0x0004} & Optional TLVs present\\
    {\tt 0x0008} & Reserved TLV type used\\
    {\tt 0x0010} & Organization specific TLV used\\
    {\tt 0x0020} & Unhandled TLV used\\
    {\tt 0x2000} & String truncated\ldots increase {\tt LLDP\_STRLEN}\\
    {\tt 0x4000} & Too many TTL\ldots increase {\tt LLDP\_NUM\_TTL}\\
    {\tt 0x8000} & Snapped payload\\
    \bottomrule
\end{longtable}

\subsubsection{lldpCaps}\label{lldpCaps}
The {\tt lldpCaps\_Enabled} column is to be interpreted as follows:
\begin{longtable}{rl}
    \toprule
    {\bf lldpCaps} & {\bf Description}\\
    \midrule\endhead%
    {\tt 0x0001}         & Other\\
    {\tt 0x0002}         & Repeater\\
    {\tt 0x0004}         & Bridge\\
    {\tt 0x0008}         & WLAN access point\\
    {\tt 0x0010}         & Router\\
    {\tt 0x0020}         & Telephone\\
    {\tt 0x0040}         & DOCSIS cable device\\
    {\tt 0x0080}         & Station only\\
    {\tt 0x0100--0x8000} & Reserved\\
    \bottomrule
\end{longtable}

%\subsection{Packet File Output}
%In packet mode ({\tt --s} option), the lldpDecode plugin outputs the following columns:
%\begin{longtable}{llll}
%    \toprule
%    {\bf Column} & {\bf Type} & {\bf Description} & {\bf Flags}\\
%    \midrule\endhead%
%    {\tt lldpDecodeCol1} & I8 & describe col1 & \\
%    \bottomrule
%\end{longtable}

\subsection{Plugin Report Output}
The following information is reported:
\begin{itemize}
    \item Number of LLDP packets
    %\item Aggregated status flags ({\tt\nameref{lldpStat}})
\end{itemize}

%\subsection{Additional Output}
%Non-standard output:
%\begin{itemize}
%    \item {\tt PREFIX\_suffix.txt}: description
%\end{itemize}

%\subsection{Post-Processing}
%
%\subsection{Example Output}
%
%\subsection{Known Bugs and Limitations}

%\subsection{TODO}
%\begin{itemize}
%    \item TODO1
%    \item TODO2
%\end{itemize}

%\subsection{References}
%\begin{itemize}
%    \item \href{https://tools.ietf.org/html/rfcXXXX}{RFCXXXX}: Title
%    \item \url{https://www.iana.org/assignments/}
%\end{itemize}

\end{document}
