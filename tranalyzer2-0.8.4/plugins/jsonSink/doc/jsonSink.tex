\IfFileExists{t2doc.cls}{
    \documentclass[documentation]{subfiles}
}{
    \errmessage{Error: could not find 't2doc.cls'}
}

\begin{document}

\trantitle
    {jsonSink}
    {JSON Output}
    {Tranalyzer Development Team} % author(s)

\section{jsonSink}\label{s:jsonSink}

\subsection{Description}
The jsonSink plugin generates JSON output in a file {\tt PREFIX\_flows.json}, where {\tt PREFIX} is provided via Tranalyzer {\tt --w} or {\tt --W} option.

\subsection{Dependencies}

\subsubsection{External Libraries}
If gzip compression is activated ({\tt GZ\_COMPRESS=1}), then {\bf zlib} must be installed.

\paragraph{Kali/Ubuntu:} {\tt sudo apt-get install zlib1g-dev}
\paragraph{Arch:} {\tt sudo pacman -S zlib}
\paragraph{Fedora/Red Hat:} {\tt sudo yum install zlib-devel}
\paragraph{Gentoo:} {\tt sudo emerge zlib}
\paragraph{OpenSUSE:} {\tt sudo zypper install zlib-devel}
\paragraph{Mac OS X:} {\tt brew install zlib}\footnote{Brew is a packet manager for Mac OS X that can be found here: \url{https://brew.sh}}

\subsection{Configuration Flags}

The following flags can be used to control the output of the plugin:

\begin{longtable}{lcll}
    \toprule
    {\bf Name} & {\bf Default} & {\bf Description} & {\bf Flags}\\
    \midrule\endhead%
    {\tt SOCKET\_ON} & 0 & Whether to output to a socket (1) or to a file (0)\\\\
    {\tt SOCKET\_ADDR} & {\tt\small ``127.0.0.1''} & Address of the socket & {\tt\small SOCKET\_ON=1}\\
    {\tt SOCKET\_PORT} & 5000 & Port of the socket & {\tt\small SOCKET\_ON=1}\\\\
    {\tt GZ\_COMPRESS} & 0 & Compress (gzip) the output\\
    {\tt JSON\_SPLIT} & 1 & Split the output file & {\tt\small SOCKET\_ON=0}\\
                        && (Tranalyzer {\tt --W} option)\\\\
    {\tt JSON\_ROOT\_NODE} & 0 & Add a root node (array)\\
    {\tt SUPPRESS\_EMPTY\_ARRAY} & 1 & Do not output empty fields\\
    {\tt JSON\_NO\_SPACES} & 1 & Suppress unnecessary spaces\\
    {\tt JS\_BUFFER\_SIZE} & 1024*1024 & Size of output buffer\\\\
    {\tt JSON\_SUFFIX} & {\tt\small ``\_flows.json''} & Suffix for output file & {\tt\small SOCKET\_ON=0}\\
    \bottomrule
\end{longtable}

\subsection{Custom File Output}
\begin{itemize}
    \item {\tt PREFIX\_flows.json}: JSON representation of Tranalyzer output
\end{itemize}

\subsection{Example}
To send compressed data over a socket ({\tt SOCKET\_ON=1} and {\tt GZ\_COMPRESS=1}):
\begin{enumerate}
    \item {\tt nc -l 127.0.0.1 5000 | gunzip}
    \item {\tt tranalyzer -r file.pcap}
\end{enumerate}

\end{document}
