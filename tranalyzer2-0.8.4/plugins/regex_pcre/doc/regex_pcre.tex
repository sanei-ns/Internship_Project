\IfFileExists{t2doc.cls}{
    \documentclass[documentation]{subfiles}
}{
    \errmessage{Error: could not find t2doc.cls}
}

\begin{document}

\trantitle
    {regex\_pcre}
    {PCRE}
    {Tranalyzer Development Team}

\section{regex\_pcre}\label{s:regex_pcre}

\subsection{Description}
The regex\_pcre plugin provides a full PCRE compatible regex engine.

\subsection{Dependencies}

\subsubsection{External Libraries}
This plugin depends on the {\bf pcre} library.
\paragraph{Ubuntu:} {\tt sudo apt-get install libpcre3-dev}
\paragraph{OpenSUSE:} {\tt sudo zypper install pcre-devel}
%\paragraph{Arch:} {\tt sudo pacman -S XXX}
\paragraph{Mac OS X:} {\tt brew install pcre}

\subsubsection{Other Plugins}
If {\tt LABELSCANS=1}, then this plugin requires the \tranrefpl{tcpFlags} plugin.

\subsubsection{Required Files}
The file {\tt regexfile.txt} is required. See \refs{regexfile.txt} for more details.

\subsection{Configuration Flags}

\subsubsection{regfile\_pcre.h}
The compiler constants in {\em regfile\_pcre.h} control the pre-processing and
compilation of the rule sets supplied in the regex file during the initialisation phase of Tranalyzer.

\begin{longtable}{lcl}
    \toprule
    {\bf Name} & {\bf Default} & {\bf Description} \\
    \midrule\endhead%
    {\tt RULE\_OPTIMIZE} & 0 & 0: No opt rules allocated 1: Allocate opt rule structure \& compile regex \\
    {\tt REGEX\_MODE} & {\tt PCRE\_DOTALL} & Regex compile time options \\
    %{\tt HDRSELMX} & 5 & \\
    \bottomrule
\end{longtable}

\subsubsection{regex\_pcre.h}
The compiler constants in {\em regex\_pcre.h} control the execution and
the output the rule matches.

\begin{longtable}{lcll}
    \toprule
    {\bf Variable} & {\bf Default} & {\bf Description} & {\bf Flags}\\
    \midrule\endhead%
    {\tt EXPERTMODE} &  0 & 0: Alarm with highest severity: class type \& severity,\\
                     &    & 1: full info \\
    {\tt PKTTIME}    &  0 & 0: no time, 1: timestamp when rule matched \\
    {\tt LABELSCANS} &  0 & 0: No scans, 1: label scans (depends on \tranrefpl{tcpFlags}) \\
    {\tt MAXREGPOS}  & 30 & Maximal \# of matches stored / flow \\
    {\tt OVECCOUNT}  &  1 & regex internal: maximal \# of regex output vectors\\
    {\tt REXPOSIX\_FILE} & {\tt\small "regexfile.txt"} & Name of regex file under {\em ./tranalyzer/plugins} \\
    \bottomrule
\end{longtable}

\subsubsection{regexfile.txt}\label{regexfile.txt}
The {\em regexfile.txt} file has the following format:
\begin{center}
\begin{lstlisting}
# ID Predecessor Flags ANDMask	ANDPin ClassID Severity Sel	Dir	Proto srcPort	dstPort	offset	Regex
# single rule
1	0	0x80	0x0000	0x0000	15	3	0x8b	0x0001	6	0	80	0	\x6A.{1,}\x6B\x3C\x24\x0B\x60\x6A.*
# single rule
3	1	0x80	0x0000	0x0000	15	3	0x82    0x0001	6	0	80	8	\x31\xDB\x8D\x43\x0D\xCD\x80\x66.*\x31
# root rules to following tree
202	0	0x11	0x0000	0x0000	20	4	0x41	0x0001	6	0	80	20	^http
203	0	0x10	0x0000	0x0000	20	4	0x41	0x0001	6	0	80	20	GET
# sucessors and predesessors
204	202	0x01	0x0000	0x0001	43	2	0x85	0x0001	6	0	445	0	Volume Serial Number
204	203	0x40	0x0000	0x0002	40	2	0x8f	0x0001	6	666	666	0	(?i)Command completed(?-i)
# successors 20t5 & 205 to 204 AND ruleset
205	204	0x81	0x0003	0x0000	40	3	0x00	0x0001	0	0	20	0	^get .*porno.*
206	204	0x80	0x0002	0x0000	35	3	0x00	0x0000	0	0	21	0	^FTP
\end{lstlisting}
\end{center}

Lines starting with a {\tt '\#'} denote a comment line and will be ignored.
All kind of rule trees can be formed using rules also acting on multiple packets using different {\tt ID}'s and {\tt Predecessor} as outlined in the example above.
Regex rules with the same {\tt ID} denote combined predecessors to other rules. Default is an OR operation unless {\tt ANDPin} bits are set. These bits denote the different inputs to a bitwise AND. The output is then provided
to the successor rule which compares with the {\tt ANDMask} bit field whether all necessary rules are matched.
Then an evaluation of the successor rule can take place. Thus, arbitrary rule trees can be constructed and results of
predecessors can be used for multiple successor rules. The variable {\tt Flags} controls the basic PCRE rule interpretation and the flow alarm production (see the table below), e.g. only if bit eight is set and alarm flow output is produced. {\tt ClassID} and {\tt Severity} denote information being printed in the flow file if the rule fires.

\begin{longtable}{rl}
    \toprule
    {\bf Flags} & {\bf Description} \\
    \midrule\endhead%
    $2^0$ (={\tt 0x01}) & {\tt PCRE\_CASELESS} \\
    $2^1$ (={\tt 0x02}) & {\tt PCRE\_MULTILINE} \\
    $2^2$ (={\tt 0x04}) & {\tt PCRE\_DOTALL} \\
    $2^3$ (={\tt 0x08}) & {\tt PCRE\_EXTENDED} \\
    $2^4$ (={\tt 0x10}) & Internal state: successor found \\
    $2^5$ (={\tt 0x20}) & Internal state: predecessor matched \\
    $2^6$ (={\tt 0x40}) & Preserve alarm in queue for later use \\
    $2^7$ (={\tt 0x80}) & Print alarm in flow file \\
    \bottomrule
\end{longtable}

The {\tt Sel} column controls the header selection of a rule in the lower nibble and the start of regex evaluation
in the higher nibble. The position of the bits in the control byte are outlined below:

\begin{longtable}{rl}
    \toprule
    {\bf Sel} & {\bf Description} \\
    \midrule\endhead%
    $2^0$ (={\tt 0x01}) & Activate dir field \\
    $2^1$ (={\tt 0x02}) & Activate L4Proto field \\
    $2^2$ (={\tt 0x04}) & Activate srcPort field \\
    $2^3$ (={\tt 0x08}) & Activate dstPort field \\
    $2^4$ (={\tt 0x10}) & Header start: Layer 2 \\
    $2^5$ (={\tt 0x20}) & Header start: Layer 3 \\
    $2^6$ (={\tt 0x40}) & Header start: Layer 4 \\
    $2^7$ (={\tt 0x80}) & Header start: Layer 7 \\
    \bottomrule
\end{longtable}

The higher nibble selects which flow direction (A={\tt 0} or B={\tt 1}), protocol, source and destination port will be
evaluated per rule, all others will be ignored. The {\tt dir} field might contain other bits meaning more selection
options in future.
The {\tt offset} column depicts the start of the regex evaluation from the selected header start, default value 0.
The {\tt Regex} column accepts a full PCRE regex term. If the regex is not correct, the rule will be discarded
displaying an error message in the Tranalyzer report.

\subsection{Flow File Output}
The {\tt regex\_pcre} plugin outputs the following columns:
\begin{longtable}{llll}
    \toprule
    {\bf Column} & {\bf Type} & {\bf Description} & {\bf Flags}\\
    \midrule\endhead%
    {\tt RgxCnt} & U16 & Regexp match count \\
    {\tt RgxClTyp} & U8 & Classtype & {\tt EXPERTMODE=0}\\
    {\tt RgxSev} & U8 & Severity & {\tt EXPERTMODE=0}\\
    {\tt RgxN\_B\_RID\_} & R(4xU16\_) & Packet, byte position, regfile ID, & {\tt EXPERTMODE=1\&\&}\\
    {\tt \quad Amsk\_F\_CT\_Sv} & \quad H8\_2xU8) & \quad AND mask, flags, classtype, severity & {\tt PKTTIME=0}\\
    {\tt RgxT\_N\_B\_RID\_} & R(TS\_4xU16\_ & Time, packet, byte position, regfile ID, & {\tt EXPERTMODE=1\&\&}\\
    {\tt \quad Amsk\_F\_CT\_Sv} & \quad H8\_2xU8) & \quad AND mask, flags, classtype, severity & {\tt PKTTIME=1}\\
    \bottomrule
\end{longtable}

\subsection{Plugin Report Output}
The following information is reported:
\begin{itemize}
    \item Number of alarms
\end{itemize}

\end{document}
