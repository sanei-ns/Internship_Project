\IfFileExists{t2doc.cls}{
    \documentclass[documentation]{subfiles}
}{
    \errmessage{Error: could not find t2doc.cls}
}

\begin{document}

\trantitle
    {ntpDecode}
    {Network Time Protocol (NTP)}
    {Tranalyzer Development Team}

\section{ntpDecode}\label{s:ntpDecode}

\subsection{Description}
The ntpDecode plugin produces a flow based view of NTP operations between computers for anomaly detection
and troubleshooting.

\subsection{Configuration Flags}
The following flags can be used to control the output of the plugin:
\begin{longtable}{lcl}
    \toprule
    {\bf Name} & {\bf Default} & {\bf Description} \\
    \midrule\endhead%
    {\tt NTP\_TS}        & 1 & 1: print NTP time stamps, 0: no time stamps\\
    {\tt NTP\_LIVM\_HEX} & 0 & Leap indicator, version and mode: 0: split into three values, 1: aggregated hex number\\
    \bottomrule
\end{longtable}

\subsection{Flow File Output}
The ntpDecode plugin outputs the following columns:
\begin{longtable}{llll}
    \toprule
    {\bf Name} & {\bf Type} & {\bf Description} & {\bf Flags}\\
    \midrule\endhead%
    {\tt \nameref{ntpStat}} & H8 & NTP status, warnings and errors \\
    {\tt \nameref{ntpLiVM}} & H8 & NTP leap indicator, version number and mode & {\tt NTP\_LIVM\_HEX=1}\\
    {\tt \hyperref[ntpLiVM]{ntpLi\_V\_M}} & U8\_U8\_U8 & NTP leap indicator, version number and mode & {\tt NTP\_LIVM\_HEX=0}\\
    {\tt \nameref{ntpStrat}} & H8 & NTP stratum \\
    {\tt ntpRefClkId} & IP4 & NTP root reference clock ID (stratum $\geq$ 2)\\
    {\tt \nameref{ntpRefStrId}} & SC & NTP root reference string (stratum $\leq$ 1)\\
    {\tt ntpPollInt} & U32 & NTP poll interval \\

    {\tt ntpPrec}      & F & NTP precision \\
    {\tt ntpRtDelMin}  & F & NTP root delay minimum \\
    {\tt ntpRtDelMax}  & F & NTP root delay maximum \\
    {\tt ntpRtDispMin} & F & NTP root dispersion minimum \\
    {\tt ntpRtDispMax} & F & NTP root dispersion maximum \\

    {\tt ntpRefTS}  & TS & NTP reference timestamp & {\tt NTP\_TS=1}\\
    {\tt ntpOrigTS} & TS & NTP originate timestamp & {\tt NTP\_TS=1}\\
    {\tt ntpRecTS}  & TS & NTP receive timestamp   & {\tt NTP\_TS=1}\\
    {\tt ntpTranTS} & TS & NTP transmit timestamp  & {\tt NTP\_TS=1}\\
    \bottomrule
\end{longtable}

\subsubsection{ntpStat}\label{ntpStat}
The {\tt ntpStat} column is to be interpreted as follows:
\begin{longtable}{rl}
    \toprule
    {\bf ntpStat} & {\bf Description}\\
    \midrule\endhead%
    $2^0$ (=0x01) & NTP port detected \\
    %$2^1$ (=0x02) & --- \\
    %$2^2$ (=0x04) & --- \\
    %$2^3$ (=0x08) & --- \\
    %$2^4$ (=0x10) & --- \\
    %$2^5$ (=0x20) & --- \\
    %$2^6$ (=0x40) & --- \\
    %$2^7$ (=0x80) & --- \\
    \bottomrule
\end{longtable}

\newpage
\subsubsection{ntpLiVM}\label{ntpLiVM}
The {\tt ntpLiVM} column is to be interpreted as follows (refer to \refs{ntp:examples} for some examples):
\begin{longtable}{rl}
    \toprule
    {\bf ntpLiVM} & {\bf Description}\\
    \midrule\endhead%
    {\tt xx.. ....}  & Leap indicator\\
    {\tt ..xx~ x...} & Version number\\ % XXX tilda is a hack to fix the output
    {\tt .... .xxx}  & Mode\\
    \bottomrule
\end{longtable}

The {\tt Leap Indicator} bits are to be interpreted as follows:
\begin{longtable}{rl}
    \toprule
    {\bf Leap Indicator} & {\bf Description}\\
    \midrule\endhead%
    {\tt 0x0} & No warning\\
    {\tt 0x1} & Last minute has 61 seconds\\
    {\tt 0x2} & Last minute has 59 seconds\\
    {\tt 0x3} & Alarm condition, clock not synchronized\\
    \bottomrule
\end{longtable}

The {\tt Mode} bits are to be interpreted as follows:
\begin{longtable}{rl}
    \toprule
    {\bf Mode} & {\bf Description}\\
    \midrule\endhead%
    {\tt 0x0} & Reserved\\
    {\tt 0x1} & Symmetric active\\
    {\tt 0x2} & Symmetric passive\\
    {\tt 0x3} & Client\\
    {\tt 0x4} & Server\\
    {\tt 0x5} & Broadcast\\
    {\tt 0x6} & NTP control message\\
    {\tt 0x7} & Private use\\
    \bottomrule
\end{longtable}

\subsubsection{ntpStrat}\label{ntpStrat}
The {\tt ntpStrat} column is to be interpreted as follows:
\begin{longtable}{rl}
    \toprule
    {\bf ntpStrat} & {\bf Description}\\
    \midrule\endhead%
    {\tt 0x00} & Unspecified \\
    {\tt 0x01} & Primary reference\\
    {\tt 0x02--0xff} & Secondary reference\\
    \bottomrule
\end{longtable}

\subsubsection{ntpRefStrId}\label{ntpRefStrId}
The interpretation of the {\tt ntpRefStrId} column depends on the value of \nameref{ntpStrat}.
The following table lists some suggested identifiers:
\begin{longtable}{rrl}
    \toprule
    {\bf ntpStrat} & {\bf ntpRefStrId} & {\bf Description}\\
    \midrule\endhead%
    {\tt 0x00} & {\tt DCN}      & DCN routing protocol\\
    {\tt 0x00} & {\tt NIST}     & NIST public modem\\
    {\tt 0x00} & {\tt TSP}      & TSP time protocol\\
    {\tt 0x00} & {\tt DTS}      & Digital Time Service\\
    {\tt 0x01} & {\tt ATOM}     & Atomic clock (calibrated)\\
    {\tt 0x01} & {\tt VLF}      & VLF radio\\
    {\tt 0x01} & {\tt callsign} & Generic radio\\
    {\tt 0x01} & {\tt LORC}     & LORAN-C\\
    {\tt 0x01} & {\tt GOES}     & GOES UHF environment satellite\\
    {\tt 0x01} & {\tt GPS}      & GPS UHF positioning satellite\\
    \bottomrule
\end{longtable}

\subsection{Examples}\label{ntp:examples}
\begin{itemize}
    \item Extract the NTP leap indicator:\\
          {\tt tawk 'NR > 1 \{ print rshift(and(strtonum(\$ntpLiVM), 0xc0), 6) \}' out\_flows.txt}
    \item Extract the NTP version:\\
          {\tt tawk 'NR > 1 \{ print rshift(and(strtonum(\$ntpLiVM), 0x38), 3) \}' out\_flows.txt}
    \item Extract the NTP mode:\\
            {\tt tawk 'NR > 1 \{ printf "\%\#x\textbackslash{}n", and(strtonum(\$ntpLiVM), 0x7) \}' out\_flows.txt}
\end{itemize}

\end{document}
