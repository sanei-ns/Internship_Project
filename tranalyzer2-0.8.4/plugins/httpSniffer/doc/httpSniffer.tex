\IfFileExists{t2doc.cls}{
    \documentclass[documentation]{subfiles}
}{
    \errmessage{Error: could not find t2doc.cls}
}

\begin{document}

\trantitle
    {httpSniffer}
    {Hypertext Transfer Protocol (HTTP)}
    {Tranalyzer Development Team}

\section{httpSniffer}\label{s:httpSniffer}
The httpSniffer plugin processes HTTP header and content information of a flow. The idea is to identify
certain HTTP features using flow parameters and to extract certain content such as text or
images for further investigation. The httpSniffer plugin requires no dependencies and produces
only output to the flow file. User defined compiler switches in {\em httpSniffer.h} produce
optimized code for the specific application.

\subsection{Configuration Flags}
The flow based output and the extracted information can be controlled by switches and constants listed in the table below.
They control the output of host, URL and method counts, names and cookies and the function of content storage.
{\bf WARNING:} The amount of being stored on disk can be substantial, make sure that the number of concurrent file handles
is large enough, use {\tt ulimit -n}.

\begin{longtable}{lcll}
    \toprule
    {\bf Name} & {\bf Default} & {\bf Description} & {\bf Flags}\\
    \midrule\endhead%
    {\tt HTTP\_MIME}   & 1 & mime types\\
    {\tt HTTP\_STAT}   & 1 & status codes\\
    {\tt HTTP\_MCNT}   & 1 & mime count: get, post \\
    {\tt HTTP\_HOST}   & 1 & hosts\\
    {\tt HTTP\_URL}    & 1 & URLs\\
    {\tt HTTP\_COOKIE} & 1 & cookies\\
    {\tt HTTP\_IMAGE}  & 1 & image names\\
    {\tt HTTP\_VIDEO}  & 1 & video names\\
    {\tt HTTP\_AUDIO}  & 1 & audio names\\
    {\tt HTTP\_MSG}    & 1 & message names\\
    {\tt HTTP\_APPL}   & 1 & application names\\
    {\tt HTTP\_TEXT}   & 1 & text names\\
    {\tt HTTP\_PUNK}   & 1 & post/else/unknown names\\
    {\tt HTTP\_BODY}   & 1 & analyse body and print anomalies\\
    {\tt HTTP\_BDURL}  & 1 & refresh and set-cookie URLs & {\tt HTTP\_BODY=1}\\
    {\tt HTTP\_USRAG}  & 1 & user agents \\
    {\tt HTTP\_XFRWD}  & 1 & X-Forward \\
    {\tt HTTP\_REFRR}  & 1 & Referer \\
    {\tt HTTP\_VIA}    & 1 & Via \\
    {\tt HTTP\_LOC}    & 0 & Location \\
    {\tt HTTP\_SERV}   & 1 & Server \\
    {\tt HTTP\_PWR}    & 1 & Powered by \\\\

    {\tt HTTP\_STATA}   & 1 & aggregate status response codes\\
    {\tt HTTP\_HOSTAGA} & 1 & aggregate hosts\\
    {\tt HTTP\_URLAGA}  & 1 & aggregate URLs\\
    {\tt HTTP\_USRAGA}  & 1 & aggregate user agents\\
    {\tt HTTP\_XFRWDA}  & 1 & aggregate X-Forward-For \\
    {\tt HTTP\_REFRRA}  & 1 & aggregate Referer \\
    {\tt HTTP\_VIAA}    & 1 & aggregate Via \\
    {\tt HTTP\_LOCA}    & 1 & aggregate Location \\
    {\tt HTTP\_SERVA}   & 1 & aggregate Server \\
    {\tt HTTP\_PWRA}    & 1 & aggregate Powered by \\\\

    {\tt HTTP\_SAVE\_IMAGE} & 0 & save all images\\
    {\tt HTTP\_SAVE\_VIDEO} & 0 & save all videos\\
    {\tt HTTP\_SAVE\_AUDIO} & 0 & save all audios\\
    {\tt HTTP\_SAVE\_MSG}   & 0 & save all messages\\
    {\tt HTTP\_SAVE\_TEXT}  & 0 & save all texts\\
    {\tt HTTP\_SAVE\_APPL}  & 0 & save all applications\\
    {\tt HTTP\_SAVE\_PUNK}  & 0 & save all else\\\\

    {\tt HTTP\_RM\_PICDIR}  & 0 & delete directories at T2 start\\
    \bottomrule
\end{longtable}
Aggregate mode is on by default to save memory space.
Note that {\tt HTTP\_SAVE\_*} refers to the {\em Content-Type}, e.g., {\tt HTTP\_SAVE\_APPL}, will save all payload whose Content-Type starts with {\tt application/} (including forms, such as \\ {\tt application/x-www-form-urlencoded}).
The maximum memory allocation per item is defined by {\tt HTTP\_DATA\_C\_MAX} listed below.
The path of each extracted http content can be set by the {\tt HTTP\_XXXX\_PATH} constant.
HTTP content having no name is assigned a default name defined by {\tt HTTP\_NONAME\_IMAGE}. Each name is
appended by the findex, packet number and an index to facilitate the mapping between flows and its content.
The latter constant has to be chosen carefully because for each item: mime, cookie, image, etc,
{\tt HTTP\_MXFILE\_LEN * HTTP\_DATA\_C\_MAX * HASHCHAINTABLE\_SIZE * HASHFACTOR} bytes are allocated.
The filenames are defined as follows:\\
Filename\_Flow-Dir(0/1)\_findex\_\#Packet-in-Flow\_\#Mimetype-in-Flow\\
So they can easily being matched with the flow or packet file. If the flow containing the filename is not present Filename = HTTP\_NONAME, defined in httpSniffer.h.

\begin{longtable}{lcl}
    \toprule
    {\bf Name} & {\bf Default} & {\bf Description} \\
    \midrule\endhead%
    {\tt HTTP\_PATH} & {\tt "/tmp/"} & Root path \\
    {\tt HTTP\_IMAGE\_PATH}   & {\tt\small HTTP\_PATH"httpPicture/"} & Path for pictures \\
    {\tt HTTP\_VIDEO\_PATH}   & {\tt\small HTTP\_PATH"httpVideo/"}   & Path for videos \\
    {\tt HTTP\_AUDIO\_PATH}   & {\tt\small HTTP\_PATH"httpAudio/"}   & Path for audios \\
    {\tt HTTP\_MSG\_PATH}     & {\tt\small HTTP\_PATH"httpMSG/"}     & Path for messages \\
    {\tt HTTP\_TEXT\_PATH}    & {\tt\small HTTP\_PATH"httpText/"}    & Path for texts \\
    {\tt HTTP\_APPL\_PATH}    & {\tt\small HTTP\_PATH"httpAppl/"}    & Path for applications \\
    {\tt HTTP\_PUNK\_PATH}    & {\tt\small HTTP\_PATH"httpPunk/"}    & Path for put/else \\
    {\tt HTTP\_NONAME\_IMAGE} & {\tt\small "nudel"} & File name for unnamed content \\
    {\tt HTTP\_DATA\_C\_MAX}  &  20 & Maximum dim of all storage array: \# / flow \\
    {\tt HTTP\_CNT\_LEN}      &  13 & max \# of cnt digits attached to file name \\
    {\tt HTTP\_FINDEX\_LEN}   &  20 & string length of findex in decimal format. \\
    {\tt HTTP\_MXFILE\_LEN}   &  80 & Maximum image name length in bytes \\
    {\tt HTTP\_MXUA\_LEN}     & 400 & Maximum user agent name length in bytes \\
    {\tt HTTP\_MXXF\_LEN}     &  80 & Maximum x-forward-for name length in bytes \\
    \bottomrule
\end{longtable}

\subsection{Flow File Output}
The default settings will result in six tab separated columns in the flow file where the items
in column 4-6 are sequences of strings separated by ';'. Whereas an item switch is set to '0' only
the occurrence of this item during the flow is supplied. It is a high speed mode
for large datasets or real-time operation in order to produce an initial idea of
interesting flows maybe by script based post processing selecting also by the
information supplied by first three columns.

\begin{longtable}{llll}
    \toprule
    {\bf Column} & {\bf Type} & {\bf Description} & {\bf Flags}\\
    \midrule\endhead%
    {\tt \nameref{httpStat}} & H16 & Status & \\
    {\tt \nameref{httpAFlags}} & H16 & Anomaly flags  & \\
    {\tt \nameref{httpMethods}} & H8 & HTTP methods & \\
    {\tt \nameref{httpHeadMimes}} & H16 & HEADMIME-TYPES & \\
    {\tt \nameref{httpCFlags}} & H8 & HTTP content body info & {\tt HTTP\_BODY=1}\\
    {\tt httpGet\_Post} & 2U16 & Number of GET and POST requests & {\tt HTTP\_MCNT=1}\\
    {\tt httpRSCnt}  & U16 & Response status count & {\tt HTTP\_STAT=1}\\
    {\tt httpRSCode} & RU16 & Response status code  & {\tt HTTP\_STAT=1}\\
    {\tt httpURL\_Via\_Loc\_Srv\_} & 10U16 & Number of URL, Via, Location, Server,\\
    {\tt \qquad Pwr\_UAg\_XFr\_} && \qquad Powered-By, User-Agent, X-Forwarded-For,\\
    {\tt \qquad Ref\_Cky\_Mim}   && \qquad Referer, Cookie and Mime-Type\\
    {\tt httpImg\_Vid\_Aud\_Msg\_} & 7U16 & Number of images, videos, audios, messages,\\
    {\tt \qquad Txt\_App\_Unk}           && \qquad texts, applications and unknown\\
    {\tt httpHosts}   & RS & Host names   & {\tt HTTP\_HOST=1}\\
    {\tt httpURL}     & RS & URLs (including parameters) & {\tt HTTP\_URL=1}\\
    {\tt httpMimes}   & RS & MIME-types   & {\tt HTTP\_MIME=1}\\
    {\tt httpCookies} & RS & Cookies      & {\tt HTTP\_COOKIE=1}\\
    {\tt httpImages}  & RS & Images       & {\tt HTTP\_IMAGE=1}\\
    {\tt httpVideos}  & RS & Videos       & {\tt HTTP\_VIDEO=1}\\
    {\tt httpAudios}  & RS & Audios       & {\tt HTTP\_AUDIO=1}\\
    {\tt httpMsgs}    & RS & Messages     & {\tt HTTP\_MSG=1}\\
    {\tt httpAppl}    & RS & Applications & {\tt HTTP\_APPL=1}\\
    {\tt httpText}    & RS & Texts        & {\tt HTTP\_TEXT=1}\\
    {\tt httpPunk}    & RS & Punk        & {\tt HTTP\_PUNK=1}\\
    {\tt httpBdyURL}  & RS & Body: Refresh, set\_cookie URL & {\tt HTTP\_BODY=1\&\&}\\
                                                          &&& {\tt HTTP\_BDURL=1}\\
    {\tt httpUsrAg} & RS & User-Agent               & {\tt HTTP\_USRAG=1}\\
    {\tt httpXFor}  & RS & X-Forwarded-For          & {\tt HTTP\_XFRWD=1}\\
    {\tt httpRefrr} & RS & Referer                  & {\tt HTTP\_REFRR=1}\\
    {\tt httpVia}   & RS & Via (Proxy)              & {\tt HTTP\_VIA=1}\\
    {\tt httpLoc}   & RS & Location (Redirection)   & {\tt HTTP\_LOC=1}\\
    {\tt httpServ}  & RS & Server                   & {\tt HTTP\_SERV=1}\\
    {\tt httpPwr}   & RS & Powered-By / Application & {\tt HTTP\_PWR=1}\\
    \bottomrule
\end{longtable}

\subsubsection{httpStat}\label{httpStat}
The {\tt httpStat} column is to be interpreted as follows:
\begin{longtable}{rl}
    \toprule
    {\bf httpStat} & {\bf Description}\\
    \midrule\endhead%
    $2^{0}$  (={\tt 0x0001}) & Warning: {\tt HTTP\_DATA\_C\_MAX} entries in flow name array reached\\
    $2^{1}$  (={\tt 0x0002}) & Warning: Filename longer than {\tt HTTP\_MXFILE\_LEN} \\
    $2^{2}$  (={\tt 0x0004}) & Internal State: pending url name\\
    $2^{3}$  (={\tt 0x0008}) & HTTP Flow\\
    $2^{4}$  (={\tt 0x0010}) & Internal State: Chunked transfer \\
    $2^{5}$  (={\tt 0x0020}) & Internal State: HTTP Flow detected \\
    $2^{6}$  (={\tt 0x0040}) & Internal State: http header parsing in process \\
    $2^{7}$  (={\tt 0x0080}) & Internal State: sequence number init \\
    $2^{8}$  (={\tt 0x0100}) & Internal State: header shift \\
    $2^{9}$  (={\tt 0x0200}) & Internal State: PUT payload sniffing \\
    $2^{10}$ (={\tt 0x0400}) & Internal State: Image payload sniffing \\
    $2^{11}$ (={\tt 0x0800}) & Internal State: video payload sniffing \\
    $2^{12}$ (={\tt 0x1000}) & Internal State: audio payload sniffing \\
    $2^{13}$ (={\tt 0x2000}) & Internal State: message payload sniffing \\
    $2^{14}$ (={\tt 0x4000}) & Internal State: text payload sniffing \\
    $2^{15}$ (={\tt 0x8000}) & Internal State: application payload sniffing \\
    \bottomrule
\end{longtable}

\subsubsection{httpAFlags}\label{httpAFlags}
The {\tt httpAFlags} column denotes HTTP anomalies regarding the
protocol and the security. It is to be interpreted as follows:
\begin{longtable}{rl}
    \toprule
    {\bf httpAFlags} & {\bf Description}\\
    \midrule\endhead%
    $2^{0}$  (={\tt 0x0001}) & Warning: POST query with parameters, possible malware \\
    $2^{1}$  (={\tt 0x0002}) & Warning: Host is IPv4\\
    $2^{2}$  (={\tt 0x0004}) & Warning: Possible DGA\\
    $2^{3}$  (={\tt 0x0008}) & Warning: Mismatched content-type\\
    $2^{4}$  (={\tt 0x0010}) & Warning: Sequence number mangled or error retry detected\\
    $2^{5}$  (={\tt 0x0020}) & Warning: Parse Error\\
    $2^{6}$  (={\tt 0x0040}) & Warning: header without value, e.g., {\tt Content-Type:\ [missing]}\\
    $2^{7}$  (={\tt 0x0080}) & \\
    $2^{8}$  (={\tt 0x0100}) & Info: X-Site Scripting protection\\
    $2^{9}$  (={\tt 0x0200}) & Info: Content Security Policy\\
    $2^{10}$ (={\tt 0x0400}) & \\
    $2^{11}$ (={\tt 0x0800}) & \\
    $2^{12}$ (={\tt 0x1000}) & Warning: possible exe download, check also mime type for conflict\\
    $2^{13}$ (={\tt 0x2000}) & Warning: possible ELF download, check also mime type for conflict\\
    $2^{14}$ (={\tt 0x4000}) & Warning: HTTP 1.0 legacy protocol, often used by malware\\
    $2^{15}$ (={\tt 0x8000}) & \\
    \bottomrule
\end{longtable}

\subsubsection{httpMethods}\label{httpMethods}
The aggregated {\tt httpMethods} bit field provides an instant overview
about the protocol state and communication during a flow. It can also be used
during post processing in order to select only flows containing e.g. responses
or delete operations.

\begin{longtable}{rll}
    \toprule
    {\bf httpMethods} & {\bf Type} & {\bf Description}\\
    \midrule\endhead%
          (={\tt 0x00}) & RESPONSE & Response of server identified by URL\\
    $2^0$ (={\tt 0x01}) & OPTIONS & Return HTTP methods that server supports for specified URL\\
    $2^1$ (={\tt 0x02}) & GET & Request of representation of specified resource \\
    $2^2$ (={\tt 0x04}) & HEAD & Request of representation of specified resource without BODY \\
    $2^3$ (={\tt 0x08}) & POST & Request to accept enclosed entity as new subordinate of resource identified by URI \\
    $2^4$ (={\tt 0x10}) & PUT & Request to store enclosed entity under supplied URI \\
    $2^5$ (={\tt 0x20}) & DELETE & Delete specified resource \\
    $2^6$ (={\tt 0x40}) & TRACE & Echo back received request \\
    $2^7$ (={\tt 0x80}) & CONNECT & Convert request connection to transparent TCP/IP tunnel \\
    \bottomrule
\end{longtable}

\subsubsection{httpHeadMimes}\label{httpHeadMimes}
The aggregated {\tt httpHeadMimes} bit field provides an instant overview
about the content of the HTTP payload being transferred during a flow. Thus, the
selection of flows with certain content during post processing is possible even when
the plugin is set to count mode for all items in order to conserve memory and processing
capabilities. The 16 Bit information is separated into Mime Type (MT) and Common Subtype Prefixes (CSP) /
 special Flags each comprising of 8 Bit. This is experimental and is subject to change if a better arrangement is found.

\begin{longtable}{rll}
    \toprule
    {\bf httpHeadMimes} & {\bf MT / CSP} & {\bf Description} \\
    \midrule\endhead%
    $2^{0}$  (={\tt 0x0001}) & application & Multi-purpose files: java or post script, etc \\
    $2^{1}$  (={\tt 0x0002}) & audio & Audio file \\
    $2^{2}$  (={\tt 0x0004}) & image & Image file \\
    $2^{3}$  (={\tt 0x0008}) & message & Instant or email message type \\
    $2^{4}$  (={\tt 0x0010}) & model &  3D computer graphics \\
    $2^{4}$  (={\tt 0x0020}) & multipart & Archives and other objects made of more than one part \\
    $2^{5}$  (={\tt 0x0040}) & text & Human-readable text and source code\\
    $2^{6}$  (={\tt 0x0080}) & video & Video stream: Mpeg, Flash, Quicktime, etc \\
    $2^{8}$  (={\tt 0x0100}) & vnd &  vendor-specific files: Word, OpenOffice, etc \\
    $2^{9}$  (={\tt 0x0200}) & x & Non-standard files: tar, SW packages, LaTex, Shockwave Flash, etc \\
    $2^{10}$ (={\tt 0x0400}) & x-pkcs & public-key cryptography standard files\\
    $2^{11}$ (={\tt 0x0800}) & --- & ---\\
    $2^{12}$ (={\tt 0x1000}) & pdf & ---\\
    $2^{13}$ (={\tt 0x2000}) & java & ---\\
    $2^{14}$ (={\tt 0x4000}) & --- & ---\\
    $2^{15}$ (={\tt 0x8000}) & allelse & All else\\
    \bottomrule
\end{longtable}

\subsubsection{httpCFlags}\label{httpCFlags}
The {\tt httpCFlags} contain information about the content body, regarding to
information about rerouting. They have to be interpreted as follows:

\begin{longtable}{rll}
    \toprule
    {\bf httpBodyFlags} & {\bf MT / CSP} & {\bf Description}\\
    \midrule\endhead%
    $2^{0}$  (={\tt 0x0001}) & STCOOKIE & http set cookie\\
    $2^{1}$  (={\tt 0x0002}) & REFRESH & http refresh detected\\
    $2^{2}$  (={\tt 0x0004}) & HOSTNAME & host name detected\\
    $2^{3}$  (={\tt 0x0008}) & BOUND & Post Boundary marker\\
    $2^{4}$  (={\tt 0x0010}) & PCNT & Potential HTTP content\\
    $2^{5}$  (={\tt 0x0020}) & --- &\\
    $2^{6}$  (={\tt 0x0040}) & QUARA & Quarantine Virus upload\\
    $2^{15}$ (={\tt 0x8000}) & --- & \\
    \bottomrule
\end{longtable}

\subsection{Plugin Report Output}
The following information is reported:
\begin{itemize}
    \item Max number of file handles (only if {\tt HTTP\_SAVE=1})
    \item Number of HTTP IPv4/6 packets
    \item Number of HTTP \#GET, \#POST, \#GET/\#POST ratio
    \item Aggregated status flags ({\tt\nameref{httpStat}})
    \item Aggregated mimetype flags ({\tt\nameref{httpHeadMimes}})
    \item Aggregated anomaly flags ({\tt\nameref{httpAFlags}})
    \item Aggregated content flags ({\tt\nameref{httpCFlags}}, only if {\tt HTTP\_BODY=1})
\end{itemize}

The GET/POST ratio is very helpful in detecting malware operations, if you know the normal ratio of your
machines in the network. The file descriptor gives you an indication of the maximum file handles the
present pcap will produce. You can increase it by invoking {\tt uname -n mylimit}, but it should not
be necessary as we manage the number of handle being open to be always below the max limit.



%\subsection{TODO}
%
%Content extraction of:
%\begin{itemize}
%   \item http 2.0
%\end{itemize}

\end{document}
