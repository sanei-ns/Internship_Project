\IfFileExists{t2doc.cls}{
    \documentclass[documentation]{subfiles}
}{
    \errmessage{Error: could not find 't2doc.cls'}
}

\begin{document}

\trantitle
    {snmpDecode} % Plugin name
    {Simple Network Management Protocol (SNMP)} % Short description
    {Tranalyzer Development Team} % author(s)

\section{snmpDecode}\label{s:snmpDecode}

\subsection{Description}
The snmpDecode plugin analyzes SNMP traffic.

%\subsection{Dependencies}

%\traninput{file} % use this command to input files
%\traninclude{file} % use this command to include files

%\tranimg{image} % use this command to include an image (must be located in a subfolder ./img/)

%\subsubsection{External Libraries}
%This plugin depends on the {\bf XXX} library.
%\paragraph{Ubuntu:} {\tt sudo apt-get install XXX}
%\paragraph{Arch:} {\tt sudo pacman -S XXX}

%\subsubsection{Other Plugins}
%This plugin requires the {\bf XXX} plugin.

%\subsubsection{Required Files}
%The file {\tt file.txt} is required.

\subsection{Configuration Flags}
The following flags can be used to control the output of the plugin:
\begin{longtable}{lcl}
    \toprule
    {\bf Name} & {\bf Default} & {\bf Description}\\
    \midrule\endhead%
    {\tt SNMP\_STRLEN} & 64 & Maximum length for strings\\
    \bottomrule
\end{longtable}

\subsection{Flow File Output}
The snmpDecode plugin outputs the following columns:
\begin{longtable}{lll}
    \toprule
    {\bf Column} & {\bf Type} & {\bf Description}\\
    \midrule\endhead%
    {\tt \nameref{snmpStat}} & H8 & Status\\
    {\tt \nameref{snmpVersion}} & U8 & Version\\
    {\tt snmpCommunity} & S & Community (SNMPv1-2)\\
    {\tt snmpUsername} & S & Username (SNMPv3)\\
    {\tt \nameref{snmpMsgT}} & H16 & Message types\\
    {\tt snmpNumReq\_Next\_Resp\_} & U64\_U64\_U64\_ & Number of GetRequest, GetNextRequest, GetResponse,\\
    {\tt \qquad Set\_Trap1\_Bulk\_} & \qquad U64\_U64\_U64\_ & \qquad SetRequest, Trapv1, GetBulkRequest,\\
    {\tt \qquad Info\_Trap2\_Rep} & \qquad U64\_U64\_U64 & \qquad InformRequest, Trapv2, and Report packets\\
    \bottomrule
\end{longtable}

\subsubsection{snmpStat}\label{snmpStat}
The {\tt snmpStat} column is to be interpreted as follows:
\begin{longtable}{rl}
    \toprule
    {\bf snmpStat} & {\bf Description}\\
    \midrule\endhead%
    {\tt 0x01} & Flow is SNMP\\
    {\tt 0x40} & String was truncated\ldots increase {\tt SNMP\_STRLEN}\\
    {\tt 0x80} & Packet was malformed\\
    \bottomrule
\end{longtable}

\subsubsection{snmpVersion}\label{snmpVersion}
The {\tt snmpVersion} column is to be interpreted as follows:
\begin{longtable}{rl}
    \toprule
    {\bf snmpVersion} & {\bf Description}\\
    \midrule\endhead%
    {\tt 0} & SNMPv1\\
    {\tt 1} & SNMPv2c\\
    {\tt 3} & SNMPv3\\
    \bottomrule
\end{longtable}

\subsubsection{snmpMsgT}\label{snmpMsgT}
The {\tt snmpMsgT} column is to be interpreted as follows:
\begin{longtable}{rl}
    \toprule
    {\bf snmpMsgT} & {\bf Description}\\
    \midrule\endhead%
    {\tt 0x0001} & GetRequest\\
    {\tt 0x0002} & GetNextRequest\\
    {\tt 0x0004} & GetResponse\\
    {\tt 0x0008} & SetRequest\\
    {\tt 0x0010} & Trap (v1)\\
    {\tt 0x0020} & GetBulkRequest (v2c, v3)\\
    {\tt 0x0040} & InformRequest\\
    {\tt 0x0080} & Trap (v2c, v3)\\
    {\tt 0x0100} & Report\\
    \bottomrule
\end{longtable}

\subsubsection{snmpType}\label{snmpType}
The {\tt snmpType} column is to be interpreted as follows:
\begin{longtable}{rl}
    \toprule
    {\bf snmpType} & {\bf Description}\\
    \midrule\endhead%
    {\tt 0xa0} & GetRequest\\
    {\tt 0xa1} & GetNextRequest\\
    {\tt 0xa2} & GetResponse\\
    {\tt 0xa3} & SetRequest\\
    {\tt 0xa4} & Trap (v1)\\
    {\tt 0xa5} & GetBulkRequest (v2c, v3)\\
    {\tt 0xa6} & InformRequest\\
    {\tt 0xa7} & Trap (v2c, v3)\\
    {\tt 0xa8} & Report\\
    \bottomrule
\end{longtable}

\subsection{Packet File Output}
In packet mode ({\tt --s} option), the snmpDecode plugin outputs the following columns:
\begin{longtable}{lll}
    \toprule
    {\bf Column} & {\bf Type} & {\bf Description}\\
    \midrule\endhead%
    {\tt \nameref{snmpVersion}} & U8 & Version\\
    {\tt snmpCommunity} & S & Community\\
    {\tt \nameref{snmpType}} & H8 & Message type\\
    \bottomrule
\end{longtable}

\subsection{Plugin Report Output}
The following information is reported:
\begin{itemize}
    \item Number of SNMP packets
    \item Number of SNMP GetRequest packets
    \item Number of SNMP GetNextRequest packets
    \item Number of SNMP GetResponse packets
    \item Number of SNMP SetRequest packets
    \item Number of SNMP Trap v1 packets
    \item Number of SNMP GetBulkRequest packets
    \item Number of SNMP InformRequest packets
    \item Number of SNMP Trap v2 packets
    \item Number of SNMP Report packets
    %\item Aggregated status flags ({\tt\nameref{snmpStat}})
\end{itemize}

%\subsection{Additional Output}
%Non-standard output:
%\begin{itemize}
%    \item {\tt PREFIX\_suffix.txt}: description
%\end{itemize}

%\subsection{Post-Processing}

%\subsection{Example Output}

%\subsection{Known Bugs and Limitations}

%\subsection{TODO}
%\begin{itemize}
%    \item TODO1
%    \item TODO2
%\end{itemize}

%\subsection{References}
%\begin{itemize}
%    \item \href{https://tools.ietf.org/html/rfcXXXX}{RFCXXXX}: Title
%    \item \url{https://www.iana.org/assignments/}
%\end{itemize}

\end{document}
