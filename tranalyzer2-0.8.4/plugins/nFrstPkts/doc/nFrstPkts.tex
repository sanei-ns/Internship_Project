\IfFileExists{t2doc.cls}{
    \documentclass[documentation]{subfiles}
}{
    \errmessage{Error: could not find 't2doc.cls'}
}

\begin{document}

\trantitle
    {nFrstPkts}
    {Statistics Over the N First Packets}
    {Tranalyzer Development Team} % author(s)

\section{nFrstPkts}\label{s:nFrstPkts}

\subsection{Description}
The nFrstPkts plugin supplies the Packet Length (PL) and Interarrival Times (IAT) of the $N$ first packets per flow as a column. The default value for $N$ is 20.
It complements the packet mode ({\tt -s} option) with flow based view for the $N$ first packets signal. The plugin supplies several configuration options of how the
resulting packet length signal should be represented. Using the {\tt fpsGplt} script files are generated readily post processable by any command line tool (AWK, Perl),
Excel or Data mining suit, such as SPSS. As outlined in the configuration below, Signals can be produced with IAT, or relative/absolute time. Also the
aggregation of bursts into a single pulse can be configured via {\tt NFRST\_MINIAT}. {\tt NFRST\_MINPLAVE} controls the meaning of the PL value in puls aggregation
mode. If 0 it corresponds to the BPP measure currently used in research for categorizing media content.

\subsection{Configuration Flags}
The following flags can be used to control the output of the plugin:
\begin{longtable}{lcll}
    \toprule
    {\bf Name} & {\bf Default} & {\bf Description} & {\bf Flags}\\
    \midrule\endhead%
    {\tt NFRST\_IAT}        &  1 & 0: Time releative to flow start,\\
                            &    & 1: Interarrival Time,\\
                            &    & 2: Absolute Time\\
    {\tt NFRST\_BCORR}      &  0 & 0: A,B start at 0.0\\
                            &    & 1: B shift by flow start & {\tt\small NFRST\_MINIATS=0}\\
    {\tt NFRST\_MINIATS}    &  0 & 0: Standard IAT Sequence\\
                            &    & 1: Minimal Pkt IAT s defining a pulse signal\\
    {\tt NFRST\_MINIATU}    &  0 & 0: Standard IAT Sequence,\\
                            &    & 1: Minimal Pkt IAT us defining a pulse signal\\
    {\tt NFRST\_MINPLENFRC} &  2 & Minimal pulse length fraction\\
    {\tt NFRST\_PLAVE}      &  1 & 0: Sum PL (BPP), & {\tt\small NFRST\_MINIATS>0||}\\
                            &    & 1: Average PL    & {\tt\small NFRST\_MINIATU>0}\\
    {\tt NFRST\_PKTCNT}     & 20 & Number of packets to record\\
    {\tt NFRST\_HDRINFO}    &  0 & add L3,L4 Header length\\
    {\tt NFRST\_XCLD}       &  0 & 0: include all,\\
                            &    & 1: include {\tt\small [NFRST\_XMIN,NFRST\_XMAX]}\\
    {\tt NFRST\_XMIN}       &  1 & min PL boundary & {\tt\small NFRST\_XCLD=1}\\
    {\tt NFRST\_XMAX} & {\tt\small UINT16\_MAX} & max PL boundary & {\tt\small NFRST\_XCLD=1}\\
    \bottomrule
\end{longtable}

For the rest of this document, {\tt NFRST\_MINIAT} is used to represent {\tt (NFRST\_MINIATS>0||NFRST\_MINIATU>0)}.

\subsection{Flow File Output}
The nFrstPkts plugin outputs the following columns:
\begin{longtable}{llll}
    \toprule
    {\bf Column} & {\bf Type} & {\bf Description} & {\bf Flags}\\
    \midrule\endhead%
    {\tt nFpCnt}                  & U32               & Number of signal samples\\
    {\tt L2L3L4Pl\_Iat}           & R(U16\_UT)        & L2/L3/L4 or payload length and inter-arrival       & {\tt\small NFRST\_HDRINFO=0\&\&}\\
                                  &                   & \qquad times for the N first packets               & {\tt\small NFRST\_MINIAT=0}\\
    {\tt L2L3L4Pl\_Iat\_nP}       & R(U16\_UT\_UT)    & L2/L3/L4 or payload length, inter-arrival times    & {\tt\small NFRST\_HDRINFO=0\&\&}\\
                                  &                   & \qquad and pulse length for the N first packets    & {\tt\small NFRST\_MINIAT>0}\\
    {\tt HD3l\_HD4l\_}            & R(U8\_U8\_        & L3Hdr, L4Hdr, L2/L3/L4 or payload length and       & {\tt\small NFRST\_HDRINFO=1\&\&}\\
    {\tt\qquad L2L3L4Pl\_Iat}     & \qquad \_U16\_UT) & \qquad inter-arrival times for the N first packets & {\tt\small NFRST\_MINIAT=0}\\
    {\tt HD3l\_HD4l\_}            & R(U8\_U8\_U16\_   & L3Hdr, L4Hdr, L2/L3/L4 or payload length and       & {\tt\small NFRST\_HDRINFO=1\&\&}\\
    {\tt\qquad L2L3L4Pl\_Iat\_nP} & \qquad UT\_UT)    & \qquad inter-arrival times for the N first packets & {\tt\small NFRST\_MINIAT>0}\\
    \bottomrule
\end{longtable}

\subsection{Post-Processing}
By invoking the script {\tt fpsGplt} under {\em trunk/scripts} files are generated for the packet signal in a Gnuplot/Excel/SPSS readable column oriented format.
It produces several signal veriants which also can be used for signal processing and AI applications. S. traffic mining tutorial on our webpage\\

\begin{verbatim}
>fpsGplt -h
Usage:
    fpsGplt [OPTION...] <FILE>

Optional arguments:

    -f               Flow index to extract, default: all flows
    -d               Flow Direction: 0, 1; default both
    -t               noTime: counts on x axis; default time on x axis
    -i               invert B Flow PL
    -s               time sorted

    -h, --help       Show this help, then exit
\end{verbatim}

\end{document}
