\IfFileExists{t2doc.cls}{
    \documentclass[documentation]{subfiles}
}{
    \errmessage{Error: could not find 't2doc.cls'}
}

\begin{document}

\trantitle
    {pwX}
    {Clear-text Passwords Extractor}
    {Tranalyzer Development Team} % author(s)

\section{pwX}\label{s:pwX}

\subsection{Description}
The pwX plugin extracts usernames and passwords from different plaintext protocols.
This plugin produces only output to the flow file.
Configuration is achieved by user defined compiler switches in {\tt src/pwX.h}.

\subsection{Configuration Flags}
The following flags can be used to control the output of the plugin:
\begin{longtable}{lcl}
    \toprule
    {\bf Variable} & {\bf Default} & {\bf Description} \\
    \midrule\endhead%
    {\tt PWX\_USERNAME}    & 1 & Defines if username column is printed.\\
    {\tt PWX\_PASSWORD}    & 1 & Defines if password column is printed.\\\\
    {\tt PWX\_FTP}         & 1 & Defines if FTP authentication is extracted.\\
    {\tt PWX\_POP3}        & 1 & Defines if POP3 authentication is extracted.\\
    {\tt PWX\_IMAP}        & 1 & Defines if IMAP authentication is extracted.\\
    {\tt PWX\_SMTP}        & 1 & Defines if SMTP authentication is extracted.\\
    {\tt PWX\_HTTP\_BASIC} & 1 & Defines if HTTP Basic Authorization is extracted.\\
    {\tt PWX\_HTTP\_PROXY} & 1 & Defines if HTTP Proxy Authorization is extracted.\\
    {\tt PWX\_HTTP\_GET}   & 1 & Defines if HTTP GET authentication is extracted.\\
    {\tt PWX\_HTTP\_POST}  & 1 & Defines if HTTP POST authentication is extracted.\\
    {\tt PWX\_IRC}         & 1 & Defines if IRC authentication is extracted.\\
    {\tt PWX\_TELNET}      & 1 & Defines if Telnet authentication is extracted.\\
    {\tt PWX\_LDAP}        & 1 & Defines if LDAP bind request authentication is extracted.\\\\
    {\tt PWX\_PAP}         & 1 & Defines if Password Authentication Protocol (PAP) is extracted.\\\\
    {\tt PWX\_STATUS}      & 1 & Whether or not to extract authentication status (success, error, \ldots).\\\\
    {\tt PWX\_DEBUG}       & 0 & Whether or not to activate debug output.\\
    \bottomrule
\end{longtable}

%\subsection{Required files}
%none

\subsection{Flow File Output}
The pwX plugin outputs the following columns:
\begin{longtable}{llll}
    \toprule
    {\bf Name} & {\bf Type} & {\bf Description} & {\bf Flags}\\
    \midrule\endhead%
    {\tt \nameref{pwxType}} & U8 & Authentication type\\
    {\tt pwxUser} & S & Extracted username & {\tt PWX\_USERNAME != 0}\\
    {\tt pwxPass} & S & Extracted password & {\tt PWX\_PASSWORD != 0}\\
    {\tt \nameref{pwxStatus}} & U8 & Authentication status & {\tt PWX\_STATUS != 0}\\
    \bottomrule
\end{longtable}

\subsubsection{pwxType}\label{pwxType}
The {\tt pwxType} column is to be interpreted as follows:
\begin{longtable}{rl}
    \toprule
    {\bf pwxType} & {\bf Description}\\
    \midrule\endhead%
     0 & No password or username extracted\\
     1 & FTP authentication\\
     2 & POP3 authentication\\
     3 & IMAP authentication\\
     4 & SMTP authentication\\
     5 & HTTP Basic Authorization\\
     6 & HTTP Proxy Authorization\\
     7 & HTTP GET authentication\\
     8 & HTTP POST authentication \\
     9 & IRC authentication \\
    10 & Telnet authentication \\
    11 & LDAP authentication \\
    12 & PAP authentication \\
    \bottomrule
\end{longtable}

\subsubsection{pwxStatus}\label{pwxStatus}
The {\tt pwxStatus} column is to be interpreted as follows:
\begin{longtable}{rl}
    \toprule
    {\bf pwxStatus} & {\bf Description}\\
    \midrule\endhead%
    0 & Authentication status is unknown\\
    1 & Authentication was successful\\
    2 & Authentication failed\\
    \bottomrule
\end{longtable}

%\subsection{Additional Output}
%none

\subsection{Plugin Report Output}
The number of passwords extracted is reported.

\end{document}
