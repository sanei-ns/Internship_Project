\IfFileExists{t2doc.cls}{
    \documentclass[documentation]{subfiles}
}{
    \errmessage{Error: could not find 't2doc.cls'}
}

\begin{document}

\trantitle
    {findexer}
    {Create a binary index to quickly extract flows from pcaps}
    {Tranalyzer Development Team} % author(s)

\section{findexer}\label{s:findexer}

% do not indent paragraphs first line
\setlength{\parindent}{0cm}

This plugin produces a binary index mapping each flow index to its packets positions in the input pcaps.
The goal of this plugin is to be able to quickly extract flows from a big pcap without having to
re-process it completely. The {\tt fextractor} tool can be used to extract flows from the pcaps using
the generated index.

\subsection{Configuration Flags}
The following flags can be used to control the output of the plugin:
\begin{center}
    \begin{tabular}{lcl}
        \toprule
        {\bf Name} & {\bf Default} & {\bf Description} \\
        \midrule
        {\tt FINDEXER\_SPLIT} & 1 & Whether (1) or not (0) to split the findexer file with t2 {\tt -W} option\\
        \bottomrule
    \end{tabular}
\end{center}

\subsection{fextractor}

The {\tt fextractor} tool can be used to extract flows using the generated {\tt \_flows.xer} index.

\begin{small}
\begin{verbatim}
Usage: fextractor -r INPUT[:start][,end] (-w OUTPUT | -n) [OPTIONS]... \
            [[DIR@]FLOWINDEX[:start][,end]]...

Extract the flows FLOWINDEX using the _flows.xer INTPUT generated by Tranalyzer2 findexer plugin.
Alternatively use a list of findexer files generated by Tranalyzer2 -W option from index start
to end. The extracted flows are written to the OUTPUT pcap.

An optional packet range can be provided on each command line FLOWINDEX to only extract packets
in the range [start, end] of this flow. If start or end are ommitted, they are replaced by,
respectively, the first and the last available packets in the flow. The FLOWINDEX can also
optionally be prefixed with a direction A or B, by default both directions are extracted.

OPTIONS:
  -r INPUT[:start][,end]
            either read packet indexes from a single _flows.xer file named INPUT
            or read packet indexes from multiple _flows.xer files prefixed by INPUT
            and with suffix in range [start, end]. If start or end are ommitted,
            they are replaced by, respectively, first and last available PCAP.
  -w OUTPUT write packets to pcap file OUTPUT
            OUTPUT "-" means that the PCAP is written to stdout.
  -f        overwrite OUTPUT if it already exists
  -n        print oldest PCAP still available, its first packet timestamp and exit
  -h        print this help message
  -i FILE   read flow indexes from FILE. FILE can either be in _flows.txt format
            (flow index in 2nd tab-separated column), or have one flow index per line.
            FILE "-" means that flows are read from stdin.
  -b        by default when FILE is in _flows.txt format, only directions present in
            it are extracted, this option force both directions to be extracted even if
            only the A or B direction is present in the flow file.
  -s N      skip the first N PCAPs
  -p DIR    search pcaps in DIR
            should only be set if pcaps were moved since Tranalyzer2 was run
\end{verbatim}
\end{small}

Example to extract flow 42, 123 and 1337 to the {\tt output.pcap} file:
    \begin{verbatim}
    fextractor -r ~/t2_output/dmp1_flows.xer -w output.pcap 42 123 1337
    \end{verbatim}

\subsection{Example scenario}

We want to extract all the flows whose source or destination are in China, to look at them in Wireshark.\\

First, we run tranalyzer with at least the {\tt findexer}, {\tt basicFlow} and {\tt txtSink} plugins.
The findexer plugin will generate a {\tt \_flows.xer} index file which keeps a list of packets positions
in the original PCAP for each flow.

    \begin{verbatim}
    [user@machine]$ tranalyzer -r capture01.pcap -w t2_output/capture01
    \end{verbatim}

We now use the {\tt srcIPCC} and {\tt dstIPCC} columns to filter flows with IPs in China.

    \begin{verbatim}
    [user@machine]$ grep IPCC t2_output/capture01_headers.txt
    9       SC:N    srcIPCC Source IP country code
    12      SC:N    dstIPCC Destination IP country code
    \end{verbatim}

The country code are in the 9 and 12 columns. The flows to extract can directly be
piped to the fextractor which then pipe the extracted PCAP to Wireshark.

{\small
    \begin{verbatim}
    [user@machine]$ awk -F"\t" '$9 == "cn" || $12 == "cn"' t2_output/capture01_flows.txt | \
            fextractor -i - -r t2_output/capture01_flows.xer -w - | wireshark -k -i -
    \end{verbatim}
}

By using {\tt tawk} we don't even need to look at the column numbers in the header file, we can directly
extract the flows of interest using the column names. {\tt tawk} also provides a {\tt -k} option
which takes care of extracting the flows and opening them in Wireshark.

{\small
    \begin{verbatim}
    [user@machine]$ tawk -k '$srcIPCC == "cn" || $dstIPCC == "cn"' t2_output/capture01_flows.txt
    \end{verbatim}
}


\subsection{Additional Output (findexer v2)}

A binary index with suffix {\tt \_flows.xer} is generated. This file is composed of the following
sections in any order (except the findexer header which is always at the beginning of the file).
All numbers are written in little endian.

\paragraph{findexer header}
\begin{verbatim}
struct findexer_header {
    uint64_t magic;          // 0x32455845444e4946 = FINDEXE2
    uint32_t pcapCount;      // number of input pcaps provided to tranalyzer
                             // 1 if -r or number of lines in -R file
    uin64_t firstPcapHeader; // offset of the fist pcap headers (see next section)
                             // in the _flows.xer file
};
\end{verbatim}

\paragraph{pcap header}
\begin{verbatim}
struct pcap_header {
    uint64_t nextPcapHeader;  // offset of the next pcap header
                              // in the _flows.xer file
    uint64_t flowCount;       // number of flows in this pcap
    uint64_t firstFlowHeader; // offset of the first flow header (see next section)
                              // of this pcap in the _flows.xer file
    uint16_t pathLength;      // length of the path string
    char* pcapPath;           // path string (NOT null terminated)
};
\end{verbatim}

\paragraph{flow header}
\begin{verbatim}
struct flow_header {
    uint64_t nextFlowHeader; // offset of the next flow header
                             // in the _flows.xer file
    uint64_t flowIndex;      // Tranalyzer flow index (2nd column in flow file)
    uint8_t flags;           // flow flags (see next section)
    uint64_t packetCount;    // number of packets in this flow
#FOREACH packet in the flow
    uin64t_t offset;         // offset in the pcap where to find the packet
#ENDFOREACH
};
\end{verbatim}

\paragraph{flow flags}

\begin{center}
    \begin{tabular}{rl}
        \toprule
        {\bf flags} & {\bf Description} \\
        \midrule
        $2^0$ (=0x01) & This is a B flow.\\
        $2^1$ (=0x02) & This is the first XER file in which this flow appears.\\
        $2^2$ (=0x04) & This is the last XER file in which this flow appears.\\
        $2^3$ (=0x08) & and all higher values: reserved for future use.\\
        \bottomrule
    \end{tabular}
\end{center}

\subsection{Limitations}

\begin{itemize}
    \item PcapNg format is not supported (packet offsets in the pcap cannot be computed because of the
        additional block structures). PcapNg can however be converted in standard Pcap using the
        following command:

    {\tt editcap -F pcap input.pcapng output.pcap}
    \item The findexer file cannot be generated when a BPF is used. With a BPF, not all packets are
        processed by Tranalyzer2 which makes it impossible to compute packets offsets in a PCAP.
\end{itemize}

\subsection{Old format (findexer v1)}

\paragraph{findexer header}
\begin{verbatim}
struct findexer_header {
    uint64_t magic;           // 0x52455845444e4946 = FINDEXER
    uint32_t pcapCount;       // number of input pcaps provided to tranalyzer
                              // 1 if -r or number of lines in -R file
#FOREACH pcap
    uin64_t pcapHeaderOffset; // offset of the per pcap headers (see next section)
                              // in the _flows.xer file
#ENDFOREACH
};
\end{verbatim}

\paragraph{pcap header}
\begin{verbatim}
struct pcap_header {
    uint16_t pathLength;    // length of the path string
    char* pcapPath;         // path string (NOT null terminated)
    uint64_t flowCount;     // number of flows in this pcap
#FOREACH flow
    uint64_t flowIndex;     // Tranalyzer flow index (2nd column in flow file)
    uint64_t packetCount;   // number of packets in this flow
    uint64_t packetsOffset; // offset in the _flows.xer file where this flow packet
                            // offsets in the pcap (see next section) are located
#ENDFOREACH
};
\end{verbatim}

\paragraph{packet offsets}
\begin{verbatim}
#FOREACH packet in the flow
    uin64_t offset; // offset in the pcap where to find the packet
#ENDFOREACH
\end{verbatim}

To extract flow {\tt 123}, the following steps are followed:
\begin{itemize}
    \item open the {\tt \_flows.xer} file and check it has the right {\tt magic} value
    \item for each pcap in {\tt pcapCount}
    \begin{itemize}
        \item read the pcap header located at {\tt pcapHeaderOffset} in the {\tt \_find.xer} file.
        \item for each flow in {\tt flowCount}
        \begin{itemize}
            \item if {\tt flowIndex == 123}: read {\tt packetCount} offsets at position {\tt packetsOffset} in the {\tt \_flows.xer} file
            and extract packets located at these offsets in the pcap at {\tt pcapPath}
        \end{itemize}
    \end{itemize}
\end{itemize}

\end{document}
