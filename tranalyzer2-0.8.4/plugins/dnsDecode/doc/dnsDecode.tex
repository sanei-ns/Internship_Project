\IfFileExists{t2doc.cls}{
    \documentclass[documentation]{subfiles}
}{
    \errmessage{Error: could not find t2doc.cls}
}

\begin{document}

\trantitle
    {dnsDecode}
    {Domain Name System (DNS)}
    {Tranalyzer Development Team}

\section{dnsDecode}\label{s:dnsDecode}

\subsection{Description}
This plugin produces DNS header and content information encountered during the lifetime of a flow.
The idea is to identify DNS header and payload features using flow parameters in order to extract
information about applications or users. The DNS plugin requires no dependencies and produces
only output to the flow file. User defined compiler switches in {\em dnsDecode.h, malsite.h} produce
optimized code for the specific application.

\subsection{Configuration Flags}\label{dnsDecode:config}
The flow based output and the extracted information can be controlled by
switches and constants listed in the table below. The most important one is {\tt DNS\_MODE}
which controls the amount of information in the flow file. {\tt DNS\_AGGR} controls the aggregation
of duplicate names and values. The last three limit the amount of memory allocated for flow based DNS record storage.
The default values revealed reasonable performance in practise.

\begin{longtable}{lcll}
    \toprule
    {\bf Name} & {\bf Default} & {\bf Description} & {\bf Flags}\\
    \midrule\endhead%
    {\tt DNS\_MODE}    &  4 & 0: Only aggregated header count info \\
                       &    & 1: +REQ records \\
                       &    & 2: +ANS records \\
                       &    & 3: +AUX records \\
                       &    & 4: +ADD records\\
    {\tt DNS\_HEXON}   &  1 & 0: Hex Output flags off, 1: Hex output flags on\\
    {\tt DNS\_REQA}    &  0 & 0: full vectors, 1: Aggregate request records\\
    {\tt DNS\_ANSA}    &  0 & 0: full vectors, 1: Aggregate answer records\\
    {\tt DNS\_QRECMAX} & 15 & Max \# of query records / flow \\
    {\tt DNS\_ARECMAX} & 20 & Max \# of answer records / flow \\
    {\tt MAL\_TEST}    & 0 & 1: activate blacklist malware test mode (IPv4 only) \\
    {\tt MAL\_TYPE}    & 0 & 1: Type string; 0: Code \\\\

    \multicolumn{3}{l}{The following additional flag is available in {\tt malsite.h}:}\\\\

    {\tt MAL\_DOMAIN} & 1 & 0: malsite ip address labeling mode \\
                      &   & 1: malsite domain labeling mode\\
    \bottomrule
\end{longtable}

%\subsection{Required Files}
%none

\subsection{Flow File Output}
The default settings will result in 11 tab separated columns in the flow file where the items
in column 6-11 are sequences of strings containing DNS record name, address entries and specific
DNS entry information such as Type or TTL separated by semicolons. The idea is that the array elements of strings of the
different columns correspond to each other so that easy script based post processing is possible.
The different output modes controlled by {\tt DNS\_MODE} provide an incremental method from a high speed
compressed representation to a full human readable representation.

\begin{longtable}{llll}
    \toprule
    {\bf Column} & {\bf Type} & {\bf Description} & {\bf Flags}\\
    \midrule\endhead%
    {\tt\nameref{dnsStat}}        & H16 & Status, warnings and errors \\
    {\tt\nameref{dnsHdriOPField}} & H16 & Header field of last packet in flow \\
    {\tt\hyperref[dnsHStatOpCRetC]{DnsStat\_}}  & H8\_        & Aggregated header status,\\
    {\tt\quad\hyperref[dnsHStatOpCRetC]{OpC\_}} & \quad H16\_ & \quad opcode and\\
    {\tt\quad\hyperref[dnsHStatOpCRetC]{RetC}}  & \quad H16   & \quad return code \\
    {\tt dnsCntQu\_}  & R:U16\_     & \# of question records \\
    {\tt \quad Asw\_} & \quad U16\_ & \quad\# answer records\\
    {\tt \quad Aux\_} & \quad U16\_ & \quad\# of auxiliary records\\
    {\tt \quad Add}   & \quad U16   & \quad\# additional records \\
    {\tt dnsAAAqF}    & F           & DDOS DNS AAA / Query factor \\
    {\tt\hyperref[dnsTypeBF]{dnsTypeBF3\_BF2\_BF1\_BF0}} & H8\_H16\_H16\_H64 & Type bitfields  & {\tt DNS\_MODE > 0}\\
    {\tt dnsQname}    & RS   & Query Name records  & {\tt DNS\_MODE > 1}\\\\
    {\tt dnsMalType}  & RS   & Domain Malware Type String & {\tt MAL\_TEST=1 \&\&}\\
                      &      &                     & {\tt MAL\_TYPE=1 \&\&}\\
                      &      &                     & {\tt MAL\_DOMAIN=1}\\
    {\tt dnsMalCode}  & RH32 & Domain Malware code & {\tt MAL\_TEST=1 \&\&}\\
                      &      &                     & {\tt MAL\_TYPE=0 \&\&}\\
                      &      &                     & {\tt MAL\_DOMAIN=1}\\
    {\tt dnsAname}  & RS & Answer Name records  \\
    {\tt dnsAPname} & RS & Name CNAME entries  \\
    {\tt dns4Aaddress} & RIP4 & Address entries IPv4 \\
    {\tt dns6Aaddress} & RIP6 & Address entries IPv6 \\
    {\tt dnsIPMalCode} & RH32 & IP Malware code & {\tt MAL\_TEST=1 \&\&}\\
                       &      &                 & {\tt MAL\_DOMAIN=0}\\
    {\tt dnsAType}   & RU16 & Answer record Type entries \\
    {\tt dnsAClass}  & RU16 & Answer record Class entries \\
    {\tt dnsATTL}    & RU32 & Answer record TTL entries \\
    {\tt dnsMXpref}  & RU16 & MX record preference entries  \\
    {\tt dnsSRVprio} & RU16 & SRV record priority entries \\
    {\tt dnsSRVwgt}  & RU16 & SRV record weight entries\\
    {\tt dnsOptStat}  & RU32 & option status\\
    {\tt dnsOptCodeOwn}  & RU16 & option code owner\\
    \bottomrule
\end{longtable}

\subsubsection{dnsStat}\label{dnsStat}
The DNS status bit field listed below provides an efficient method to post process
flow data files in order to detect incidents during flow processing.

\begin{longtable}{rll}
    \toprule
    {\bf dnsStat} & {\bf Type} & {\bf Description}\\
    \midrule\endhead%
    $2^{0}$  (={\tt 0x0001}) & {\tt DNS\_PRTDT}   & DNS ports detected \\
    $2^{1}$  (={\tt 0x0002}) & {\tt DNS\_NBIOS}   & NetBios DNS\\
    $2^{2}$  (={\tt 0x0004}) & {\tt DNS\_FRAGA}   & DNS TCP aggregated fragmented content \\
    $2^{3}$  (={\tt 0x0008}) & {\tt DNS\_FRAGS}   & DNS TCP fragmented content state\\
    $2^{4}$  (={\tt 0x0010}) & {\tt DNS\_FTRUNC}  & Warning: Name truncated \\
    $2^{5}$  (={\tt 0x0020}) & {\tt DNS\_ANY}     & Warning: ANY: Zone all from a domain or cached server\\
    $2^{6}$  (={\tt 0x0040}) & {\tt DNS\_IZTRANS} & Warning: Incremental DNS zone transfer detected \\
    $2^{7}$  (={\tt 0x0080}) & {\tt DNS\_ZTRANS}  & Warning: DNS zone transfer detected \\
    $2^{8}$  (={\tt 0x0100}) & {\tt DNS\_WRNULN}  & Warning: DNS UDP Length exceeded \\
    $2^{9}$  (={\tt 0x0200}) & {\tt DNS\_WRNIGN}  & Warning: following Records ignored \\
    $2^{10}$ (={\tt 0x0400}) & {\tt DNS\_WRNDEX}  & Warning: Max DNS name records exceeded \\
    $2^{11}$ (={\tt 0x0800}) & {\tt DNS\_WRNAEX}  & Warning: Max address records exceeded \\
    $2^{12}$ (={\tt 0x1000}) & {\tt DNS\_ERRLEN}  & Error: DNS record length error\\
    $2^{13}$ (={\tt 0x2000}) & {\tt DNS\_ERRPTR}  & Error: Wrong DNS PTR detected\\
    $2^{14}$ (={\tt 0x4000}) & {\tt DNS\_WRNMLN}  & Warning: DNS length undercut\\
    $2^{15}$ (={\tt 0x8000}) & {\tt DNS\_ERRCRPT} & Error: UDP/TCP DNS Header corrupt or TCP packets missing\\
    \bottomrule
\end{longtable}

\subsubsection{dnsHdriOPField}\label{dnsHdriOPField}
From the 16 Bit DNS header the QR Bit and Bit five to nine are extracted and mapped in their correct sequence
into a byte as indicated below. It provides for a normal single packet exchange flow an accurate status of the
DNS transfer. For a multiple packet exchange only the last packet is mapped into the variable.
In that case the aggregated header state flags should be considered.

\begin{longtable}{cccccccccc}
    \toprule
    {\bf QR} & {\bf Opcode} & {\bf AA} & {\bf TC} & {\bf RD} & {\bf RA} & {\bf Z} & {\bf AD} & {\bf CD} & {\bf Rcode} \\
    \midrule\endhead%
    1 & 0000 & 1 & 0 & 1 & 1 & 1 & 0 & 0 & 0000 \\
    \bottomrule
\end{longtable}

\subsubsection{dnsHStat\_OpC\_RetC}\label{dnsHStatOpCRetC}
For multi-packet DNS flows e.g. via TCP the aggregated header state bit field describes
the status of all packets in a flow. Thus, flows with certain client and server states can
be easily identified and extracted during post-processing.

\begin{longtable}{rcl}
    \toprule
    {\bf dnsHStat} & {\bf Short} & {\bf Description}\\
    \midrule\endhead%
    $2^7$ (={\tt 0x01}) & {\tt CD} & Checking Disabled \\
    $2^6$ (={\tt 0x02}) & {\tt AD} & Authenticated Data \\
    $2^5$ (={\tt 0x04}) & {\tt Z}  & Zero \\
    $2^4$ (={\tt 0x08}) & {\tt RA} & Recursion Available \\
    $2^3$ (={\tt 0x10}) & {\tt RD} & Recursion Desired \\
    $2^2$ (={\tt 0x20}) & {\tt TC} & Truncated \\
    $2^1$ (={\tt 0x40}) & {\tt AA} & Authoritative Answer \\
    $2^0$ (={\tt 0x80}) & {\tt QR} & Query / Response \\
    \bottomrule
\end{longtable}

The four bit OpCode field of the DNS header is mapped via [$2^{\text{Opcode}}$] and an OR
into a 16 Bit field. Thus, the client can be monitored or anomalies easily identified.
E.g. appearance of reserved bits might be an indication for a covert channel or
malware operation.

\begin{longtable}{rl}
    \toprule
    {\bf dnsOpC} & {\bf Description}\\
    \midrule\endhead%
    $2^{0}$  (={\tt 0x0001}) & {\tt QUERY}, Standard query \\
    $2^{1}$  (={\tt 0x0002}) & {\tt IQUERY}, Inverse query \\
    $2^{2}$  (={\tt 0x0004}) & {\tt STATUS}, Server status request \\
    $2^{3}$  (={\tt 0x0008}) & --- \\
    $2^{4}$  (={\tt 0x0010}) & Notify \\
    $2^{4}$  (={\tt 0x0020}) & Update \\
    $2^{5}$  (={\tt 0x0040}) & reserved \\
    $2^{6}$  (={\tt 0x0080}) & reserved \\
    $2^{8}$  (={\tt 0x0100}) & reserved \\
    $2^{9}$  (={\tt 0x0200}) & reserved \\
    $2^{10}$ (={\tt 0x0400}) & reserved \\
    $2^{11}$ (={\tt 0x0800}) & reserved \\
    $2^{12}$ (={\tt 0x1000}) & reserved \\
    $2^{13}$ (={\tt 0x2000}) & reserved \\
    $2^{14}$ (={\tt 0x4000}) & reserved \\
    $2^{15}$ (={\tt 0x8000}) & reserved \\
    \bottomrule
\end{longtable}

The four bit RCode field of the DNS header is mapped via [$2^{\text{Rcode}}$] and an OR
into a 16 Bit field. It provides valuable information about success of DNS queries
and therefore facilitates the detection of failures, misconfigurations and malicious
operations.

\begin{longtable}{rcl}
    \toprule
    {\bf dnsRetC} & {\bf Short} & {\bf Description}\\
    \midrule\endhead%
    $2^{0}$  (={\tt 0x0001}) & No error & Request completed successfully \\
    $2^{1}$  (={\tt 0x0002}) & Format error & Name server unable to interpret query \\
    $2^{2}$  (={\tt 0x0004}) & Server failure & Name server unable to process query due to problem with name server \\
    $2^{3}$  (={\tt 0x0008}) & Name Error & Authoritative name server only: Domain name in query does not exist \\
    $2^{4}$  (={\tt 0x0010}) & Not Implemented & Name server does not support requested kind of query.\\
    $2^{4}$  (={\tt 0x0020}) & Refused & Name server refuses to perform the specified operation for policy reasons. \\
    $2^{5}$  (={\tt 0x0040}) & YXDomain & Name Exists when it should not \\
    $2^{6}$  (={\tt 0x0080}) & YXRRSet & RR Set Exists when it should not \\
    $2^{8}$  (={\tt 0x0100}) & NXRRSet & RR Set that should exist does not \\
    $2^{9}$  (={\tt 0x0200}) & NotAuth & Server Not Authoritative for zone \\
    $2^{10}$ (={\tt 0x0400}) & NotZone & Name not contained in zone \\
    $2^{11}$ (={\tt 0x0800}) & --- & --- \\
    $2^{12}$ (={\tt 0x1000}) & --- & --- \\
    $2^{13}$ (={\tt 0x2000}) & --- & --- \\
    $2^{14}$ (={\tt 0x4000}) & --- & --- \\
    $2^{15}$ (={\tt 0x8000}) & --- & --- \\
    \bottomrule
\end{longtable}

\subsubsection{dnsTypeBF3\_BF2\_BF1\_BF0}\label{dnsTypeBF}
The 16 bit Type Code field is extracted from each DNS record and mapped via [$2^{\text{Typecode}}$] into a
64 Bit fields. Gaps are avoided by additional higher bitfields defining higher codes.

\begin{longtable}{rcl}
    \toprule
    {\bf dnsTypeBF3} & {\bf Short} & {\bf Description}\\
    \midrule\endhead%
    $2^0$ (={\tt 0x01}) & TA & DNSSEC Trust Authorities \\
    $2^1$ (={\tt 0x02}) & DLV & DNSSEC Lookaside Validation \\
    $2^2$ (={\tt 0x04}) & --- & --- \\
    $2^3$ (={\tt 0x08}) & --- & --- \\
    $2^4$ (={\tt 0x10}) & --- & --- \\
    $2^5$ (={\tt 0x20}) & --- & --- \\
    $2^6$ (={\tt 0x40}) & --- & --- \\
    $2^7$ (={\tt 0x80}) & --- & --- \\
    \bottomrule
\end{longtable}

\begin{longtable}{rcl}
    \toprule
    {\bf dnsTypeBF2} & {\bf Short} & {\bf Description} \\
    \midrule\endhead%
    $2^{0}$  (={\tt 0x0001}) & TKEY & Transaction Key \\
    $2^{1}$  (={\tt 0x0002}) & TSIG & Transaction Signature \\
    $2^{2}$  (={\tt 0x0004}) & IXFR & Incremental transfer \\
    $2^{3}$  (={\tt 0x0008}) & AXFR & Transfer of an entire zone \\
    $2^{4}$  (={\tt 0x0010}) & MAILB & Mailbox-related RRs (MB, MG or MR) \\
    $2^{5}$  (={\tt 0x0020}) & MAILA & Mail agent RRs (OBSOLETE - see MX) \\
    $2^{6}$  (={\tt 0x0040}) & ZONEALL & Request for all records the server/cache has available  \\
    $2^{7}$  (={\tt 0x0080}) & URI & URI \\
    $2^{8}$  (={\tt 0x0100}) & CAA & Certification Authority Restriction \\
    $2^{9}$  (={\tt 0x0200}) & --- & --- \\
    $2^{10}$ (={\tt 0x0400}) & --- & --- \\
    $2^{11}$ (={\tt 0x0800}) & --- & --- \\
    $2^{12}$ (={\tt 0x1000}) & --- & --- \\
    $2^{13}$ (={\tt 0x2000}) & --- & --- \\
    $2^{14}$ (={\tt 0x4000}) & --- & --- \\
    $2^{15}$ (={\tt 0x8000}) & --- & --- \\
    \bottomrule
\end{longtable}

\begin{longtable}{rcl}
    \toprule
    {\bf dnsTypeBF1} & {\bf Short} & {\bf Description} \\
    \midrule\endhead%
    $2^{0}$  (={\tt 0x0001}) & SPF &  \\
    $2^{1}$  (={\tt 0x0002}) & UINFO &  \\
    $2^{2}$  (={\tt 0x0004}) & UID &  \\
    $2^{3}$  (={\tt 0x0008}) & GID & \\
    $2^{4}$  (={\tt 0x0010}) & UNSPEC &  \\
    $2^{4}$  (={\tt 0x0020}) & NID &  \\
    $2^{5}$  (={\tt 0x0040}) & L32 &  \\
    $2^{6}$  (={\tt 0x0080}) & L64 &  \\
    $2^{8}$  (={\tt 0x0100}) & LP &  \\
    $2^{9}$  (={\tt 0x0200}) & EUI48  & EUI-48 address \\
    $2^{10}$ (={\tt 0x0400}) & EUI64 & EUI-48 address \\
    $2^{11}$ (={\tt 0x0800}) & --- & --- \\
    $2^{12}$ (={\tt 0x1000}) & --- & --- \\
    $2^{13}$ (={\tt 0x2000}) & --- & --- \\
    $2^{14}$ (={\tt 0x4000}) & --- & --- \\
    $2^{15}$ (={\tt 0x8000}) & --- & --- \\
    \bottomrule
\end{longtable}

\begin{longtable}{rcl}
    \toprule
    {\bf dnsTypeBF0} & {\bf Short} & {\bf Description}\\
    \midrule\endhead%
    $2^{0}$  (={\tt 0x0000.0000.0000.0001}) & --- & --- \\
    $2^{1}$  (={\tt 0x0000.0000.0000.0002}) & A & IPv4 address \\
    $2^{2}$  (={\tt 0x0000.0000.0000.0004}) & NS & Authoritative name server \\
    $2^{3}$  (={\tt 0x0000.0000.0000.0008}) & MD & Mail destination. Obsolete use MX instead \\
    $2^{4}$  (={\tt 0x0000.0000.0000.0010}) & MF & Mail forwarder. Obsolete use MX instead \\
    $2^{5}$  (={\tt 0x0000.0000.0000.0020}) & CNAME & Canonical name for an alias \\
    $2^{6}$  (={\tt 0x0000.0000.0000.0040}) & SOA & Marks the start of a zone of authority \\
    $2^{7}$  (={\tt 0x0000.0000.0000.0080}) & MB & Mailbox domain name \\
    $2^{8}$  (={\tt 0x0000.0000.0000.0100}) & MG & Mail group member \\
    $2^{9}$  (={\tt 0x0000.0000.0000.0200}) & MR & Mail rename domain name \\
    $2^{10}$ (={\tt 0x0000.0000.0000.0400}) & NULL & Null resource record \\
    $2^{11}$ (={\tt 0x0000.0000.0000.0800}) & WKS & Well known service description \\
    $2^{12}$ (={\tt 0x0000.0000.0000.1000}) & PTR & Domain name pointer \\
    $2^{13}$ (={\tt 0x0000.0000.0000.2000}) & HINFO & Host information \\
    $2^{14}$ (={\tt 0x0000.0000.0000.4000}) & MINFO & Mailbox or mail list information \\
    $2^{15}$ (={\tt 0x0000.0000.0000.8000}) & MX & Mail exchange \\
    $2^{16}$ (={\tt 0x0000.0000.0001.0000}) & TXT & Text strings \\
    $2^{17}$ (={\tt 0x0000.0000.0002.0000}) & --- & Responsible Person \\
    $2^{18}$ (={\tt 0x0000.0000.0004.0000}) & AFSDB & AFS Data Base location \\
    $2^{19}$ (={\tt 0x0000.0000.0008.0000}) & X25 & X.25 PSDN address \\
    $2^{20}$ (={\tt 0x0000.0000.0010.0000}) & ISDN & ISDN address \\
    $2^{21}$ (={\tt 0x0000.0000.0020.0000}) & RT & Route Through \\
    $2^{22}$ (={\tt 0x0000.0000.0040.0000}) & NSAP & NSAP address. NSAP style A record \\
    $2^{23}$ (={\tt 0x0000.0000.0080.0000}) & NSAP-PTR & --- \\
    $2^{24}$ (={\tt 0x0000.0000.0100.0000}) & SIG & Security signature \\
    $2^{25}$ (={\tt 0x0000.0000.0200.0000}) & KEY & Security key \\
    $2^{26}$ (={\tt 0x0000.0000.0400.0000}) & PX & X.400 mail mapping information \\
    $2^{27}$ (={\tt 0x0000.0000.0800.0000}) & GPOS & Geographical Position \\
    $2^{28}$ (={\tt 0x0000.0000.1000.0000}) & AAAA & IPv6 Address \\
    $2^{29}$ (={\tt 0x0000.0000.2000.0000}) & LOC & Location Information \\
    $2^{30}$ (={\tt 0x0000.0000.4000.0000}) & NXT & Next Domain (obsolete) \\
    $2^{31}$ (={\tt 0x0000.0000.8000.0000}) & EID & Endpoint Identifier \\
    $2^{32}$ (={\tt 0x0000.0001.0000.0000}) & NIMLOC/NB & Nimrod Locator / NetBIOS general Name Service \\
    $2^{33}$ (={\tt 0x0000.0002.0000.0000}) & SRV/NBSTAT & Server Selection / NetBIOS NODE STATUS \\
    $2^{34}$ (={\tt 0x0000.0004.0000.0000}) & ATMA & ATM Address \\
    $2^{35}$ (={\tt 0x0000.0008.0000.0000}) & NAPTR & Naming Authority Pointer \\
    $2^{36}$ (={\tt 0x0000.0010.0000.0000}) & KX & Key Exchanger \\
    $2^{37}$ (={\tt 0x0000.0020.0000.0000}) & CERT & --- \\
    $2^{38}$ (={\tt 0x0000.0040.0000.0000}) & A6 & A6 (OBSOLETE - use AAAA) \\
    $2^{39}$ (={\tt 0x0000.0080.0000.0000}) & DNAME & --- \\
    $2^{40}$ (={\tt 0x0000.0100.0000.0000}) & SINK & --- \\
    $2^{41}$ (={\tt 0x0000.0200.0000.0000}) & OPT & --- \\
    $2^{42}$ (={\tt 0x0000.0400.0000.0000}) & APL & --- \\
    $2^{43}$ (={\tt 0x0000.0800.0000.0000}) & DS & Delegation Signer \\
    $2^{44}$ (={\tt 0x0000.1000.0000.0000}) & SSHFP & SSH Key Fingerprint \\
    $2^{45}$ (={\tt 0x0000.2000.0000.0000}) & IPSECKEY & --- \\
    $2^{46}$ (={\tt 0x0000.4000.0000.0000}) & RRSIG & --- \\
    $2^{47}$ (={\tt 0x0000.8000.0000.0000}) & NSEC & NextSECure \\
    $2^{48}$ (={\tt 0x0001.0000.0000.0000}) & DNSKEY & --- \\
    $2^{49}$ (={\tt 0x0002.0000.0000.0000}) & DHCID & DHCP identifier \\
    $2^{50}$ (={\tt 0x0004.0000.0000.0000}) & NSEC3 & --- \\
    $2^{51}$ (={\tt 0x0008.0000.0000.0000}) & NSEC3PARAM & --- \\
    $2^{52}$ (={\tt 0x0010.0000.0000.0000}) & TLSA & --- \\
    $2^{53}$ (={\tt 0x0020.0000.0000.0000}) & SMIMEA & S/MIME cert association \\
    $2^{54}$ (={\tt 0x0040.0000.0000.0000}) & --- & \\
    $2^{55}$ (={\tt 0x0080.0000.0000.0000}) & HIP & Host Identity Protocol \\
    $2^{56}$ (={\tt 0x0100.0000.0000.0000}) & NINFO & --- \\
    $2^{57}$ (={\tt 0x0200.0000.0000.0000}) & RKEY & --- \\
    $2^{58}$ (={\tt 0x0400.0000.0000.0000}) & TALINK & Trust Anchor LINK \\
    $2^{59}$ (={\tt 0x0800.0000.0000.0000}) & CDS & Child DS \\
    $2^{60}$ (={\tt 0x1000.0000.0000.0000}) & CDNSKEY & DNSKEY(s) the Child wants reflected in DS \\
    $2^{61}$ (={\tt 0x2000.0000.0000.0000}) & OPENPGPKEY & OpenPGP Key \\
    $2^{62}$ (={\tt 0x4000.0000.0000.0000}) & CSYNC & Child-To-Parent Synchronization \\
    $2^{63}$ (={\tt 0x8000.0000.0000.0000}) & --- & \\
    \bottomrule
\end{longtable}

\subsection{Plugin Report Output}
The following information is reported:
\begin{itemize}
    \item Number of DNS IPv4/6 packets
    \item Number of DNS IPv4/6 Q,R packets
    \item Aggregated status flags ({\tt\nameref{dnsStat}})
    \item Number of alarms ({\tt\hyperref[dnsDecode:config]{MAL\_TEST}})
\end{itemize}

\subsection{Example Output}
The idea is that the string and integer array elements of question, answer, TTL and Type record entries
match by column index so that easy script based mapping and post processing is possible. A sample output
is shown below. Especially when large records are present the same name is printed several times which
might degrade the readability. Therefore, a next version will have a multiple Aname suppressor switch,
which should be off for script based post-processing.

\begin{small}
    \begin{longtable}{ccccc}
        \toprule
        {\bf Query name} & {\bf Answer name} & {\bf Answer address} & {\bf TTL} & {\bf Type} \\
        \midrule\endhead%
        www.macromedia.com; & www.macromedia.com;www-mm.wip4.adobe.com & 0.0.0.0;8.118.124.64 & 2787;4 & 5;1 \\
        \bottomrule
    \end{longtable}
\end{small}

\subsection{TODO}
\begin{itemize}
    \item Compressed mode for DNS records
\end{itemize}

\end{document}
