\IfFileExists{t2doc.cls}{
    \documentclass[documentation]{subfiles}
}{
    \errmessage{Error: could not find 't2doc.cls'}
}

\begin{document}

\trantitle
    {vrrpDecode} % Plugin name
    {Virtual Router Redundancy Protocol (VRRP)} % Short description
    {Tranalyzer Development Team} % author(s)

\section{vrrpDecode}\label{s:vrrpDecode}

\subsection{Description}
The vrrpDecode plugin analyzes Virtual Router Redundancy Protocol (VRRP) traffic.

%\subsection{Dependencies}
%None.

\subsection{Configuration Flags}
The following flags can be used to control the output of the plugin:
\begin{longtable}{lcll}
    \toprule
    {\bf Name} & {\bf Default} & {\bf Description} & {\bf Flags}\\
    \midrule\endhead%
    {\tt VRRP\_NUM\_VRID} &  5 & number of unique virtual router ID to store\\
    {\tt VRRP\_NUM\_IP}   & 25 & number of unique IPs to store\\
    {\tt VRRP\_RT}        &  0 & Whether (1) or not (0) to output routing tables\\
    {\tt VRRP\_SUFFIX} & {\tt "\_vrrp.txt"} & Suffix for routing tables file & {\tt VRRP\_RT=1}\\
    \bottomrule
\end{longtable}

\subsection{Flow File Output}
The vrrpDecode plugin outputs the following columns:
\begin{longtable}{lll}
    \toprule
    {\bf Column} & {\bf Type} & {\bf Description}\\% & {\bf Flags}\\
    \midrule\endhead%
    {\tt \nameref{vrrpStat}}     & H16   & Status\\
    {\tt \nameref{vrrpVer}}      & H8    & Version\\
    {\tt \nameref{vrrpType}}     & H8    & Type\\
    {\tt vrrpVRIDCnt}            & U32   & Virtual router ID count\\
    {\tt vrrpVRID}               & RU8   & Virtual router ID\\
    {\tt vrrpMinPri}             & U8    & Minimum priority\\
    {\tt vrrpMaxPri}             & U8    & Maximum priority\\
    {\tt vrrpMinAdvInt}          & U8    & Minimum advertisement interval [s]\\
    {\tt vrrpMaxAdvInt}          & U8    & Maximum advertisement interval [s]\\
    {\tt \nameref{vrrpAuthType}} & H8    & Autentication type\\
    {\tt vrrpAuth}               & SC    & Authentication string\\
    {\tt vrrpIPCnt}              & U32   & IP address count\\
    {\tt vrrpIP}                 & R(IP) & IP addresses\\
    \bottomrule
\end{longtable}

\subsubsection{vrrpStat}\label{vrrpStat}
The {\tt vrrpStat} column is to be interpreted as follows:
\begin{longtable}{rl}
    \toprule
    {\bf vrrpStat} & {\bf Description}\\
    \midrule\endhead%
    {\tt 0x0001} & flow is VRRP\\
    {\tt 0x0002} & invalid version\\
    {\tt 0x0004} & invalid type\\
    {\tt 0x0008} & invalid checksum\\
    {\tt 0x0010} & invalid TTL (should be 255)\\
    {\tt 0x0020} & invalid destination IP (should be 224.0.0.18)\\
    {\tt 0x0040} & invalid destination MAC (should be 00:00:5e:00:01:routerID)\\
    {\tt 0x0100} & Virtual Router ID list truncated\ldots increase {\tt VRRP\_NUM\_VRID}\\
    {\tt 0x0200} & IP list truncated\ldots increase {\tt VRRP\_NUM\_IP}\\
    {\tt 0x4000} & Packet snapped\\
    {\tt 0x8000} & Malformed packet\ldots covert channel?\\
    \bottomrule
\end{longtable}

\subsubsection{vrrpVer}\label{vrrpVer}
The {\tt vrrpVer} column is to be interpreted as follows:
\begin{longtable}{rl}
    \toprule
    {\bf vrrpVer} & {\bf Description}\\
    \midrule\endhead%
    {\tt 0x04} & VRRP v2\\
    {\tt 0x08} & VRRP v3\\
    \bottomrule
\end{longtable}

\subsubsection{vrrpType}\label{vrrpType}
The {\tt vrrpType} column is to be interpreted as follows:
\begin{longtable}{rl}
    \toprule
    {\bf vrrpType} & {\bf Description}\\
    \midrule\endhead%
    {\tt 0x01} & Advertisement\\
    \bottomrule
\end{longtable}

\subsubsection{vrrpAuthType}\label{vrrpAuthType}
The {\tt vrrpAuthType} column is to be interpreted as follows:
\begin{longtable}{rl}
    \toprule
    {\bf vrrpAuthType} & {\bf Description}\\
    \midrule\endhead%
    {\tt 0x01} & No authentication\\
    {\tt 0x02} & Simple text password\\
    {\tt 0x04} & IP Authentication Header\\
    \bottomrule
\end{longtable}

\subsection{Additional Output}
Non-standard output:
\begin{itemize}
    \item {\tt PREFIX\_vrrp.txt}: VRRP routing tables
\end{itemize}
The routing tables contain the following columns:
\begin{longtable}{rl}
    \toprule
    {\bf Name} & {\bf Description}\\
    \midrule\endhead%
    {\tt VirtualRtrID}          & Virtual router ID\\
    {\tt Priority}              & Priority\\
    {\tt SkewTime[s]}           & Skew time (seconds)\\
    {\tt MasterDownInterval[s]} & Master down interval (seconds)\\
    {\tt AddrCount}             & Number of addresses\\
    {\tt Addresses}             & List of addresses\\
    {\tt Version}               & VRRP version\\
    {\tt Type}                  & \hyperref[vrrpType]{Message type}\\
    {\tt AdverInt[s]}           & Advertisement interval\\
    {\tt AuthType}              & \hyperref[vrrpAuthType]{Authentication type}\\
    {\tt AuthString}            & Authentication string\\
    {\tt Checksum}              & Stored checksum\\
    {\tt CalcChecksum}          & Calculated checksum\\
    {\tt flowIndex}             & Flow index\\
    \bottomrule
\end{longtable}

\subsection{Plugin Report Output}
The number of VRRP v2 and v3 packets is reported.

\subsection{Post-Processing}
The routing tables can be pruned by using the following command:
\begin{verbatim}
sort -u PREFIX_vrrp.txt > PREFIX_vrrp_pruned.txt
\end{verbatim}

\end{document}
