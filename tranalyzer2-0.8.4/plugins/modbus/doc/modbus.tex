\IfFileExists{t2doc.cls}{
    \documentclass[documentation]{subfiles}
}{
    \errmessage{Error: could not find 't2doc.cls'}
}

\begin{document}

\trantitle
    {modbus} % Plugin name
    {Modbus} % Short description
    {Tranalyzer Development Team} % author(s)

\section{modbus}\label{s:modbus}

\subsection{Description}
The modbus plugin analyzes Modbus traffic.

%\subsection{Dependencies}
%None.

\subsection{Configuration Flags}
The following flags can be used to control the output of the plugin:
\begin{longtable}{lcll}
    \toprule
    {\bf Name} & {\bf Default} & {\bf Description}\\
    \midrule\endhead%
    {\tt MB\_DEBUG}      & 0 & Whether (1) or not (0) to activate debug output\\\\
    {\tt MB\_FE\_FRMT}   & 0 & Function/Exception codes representation: 0: hex, 1: int\\\\
    {\tt MB\_NUM\_FUNC}  & 0 & Number of function codes to store (0 to hide \hyperref[modbusFC]{modbusFC})\\
    {\tt MB\_UNIQ\_FUNC} & 0 & Whether or not to aggregate multiply defined function codes\\\\
    {\tt MB\_NUM\_FEX}   & 0 & Number of function codes causing exceptions to store (0 to hide \hyperref[modbusFEx]{modbusFEx})\\
    {\tt MB\_UNIQ\_FEX}  & 0 & Whether or not to aggregate multiply defined function codes causing exceptions\\\\
    {\tt MB\_NUM\_EX}    & 0 & Number of exception codes to store (0 to hide \hyperref[modbusExC]{modbusExC})\\
    {\tt MB\_UNIQ\_EX}   & 0 & Whether or not to aggregate multiply defined exception codes\\
    \bottomrule
\end{longtable}

\subsection{Flow File Output}
The modbus plugin outputs the following columns:
\begin{longtable}{llll}
    \toprule
    {\bf Column} & {\bf Type} & {\bf Description} & {\bf Flags}\\
    \midrule\endhead%
    {\tt \nameref{modbusStat}} & H16 & Status\\
    {\tt modbusUID}   & U8  & Unit identifier\\
    {\tt modbusNPkts} & U32 & Number of Modbus packets\\
    {\tt modbusNumEx} & U16 & Number of exceptions\\
    {\tt \hyperref[modbusFC]{modbusFCBF}}   & H64 & Aggregated function codes\\
    {\tt \hyperref[modbusFC]{modbusFC}}     & RH8 & List of function codes & {\tt MB\_NUM\_FUNC>0}\\
    {\tt \hyperref[modbusFEx]{modbusFExBF}} & H64 & Aggregated function codes which caused exceptions\\
    {\tt \hyperref[modbusFEx]{modbusFEx}}   & RH8 & List of function codes which caused exceptions & {\tt MB\_NUM\_FEX>0}\\
    {\tt \hyperref[modbusExC]{modbusExCBF}} & H16 & Aggregated exception codes\\
    {\tt \hyperref[modbusExC]{modbusExC}}   & RH8 & List of exception codes & {\tt MB\_NUM\_EX>0}\\
    \bottomrule
\end{longtable}

\subsubsection{modbusStat}\label{modbusStat}
The {\tt modbusStat} column is to be interpreted as follows:
\begin{longtable}{rl}
    \toprule
    {\bf modbusStat} & {\bf Description}\\
    \midrule\endhead%
    {\tt 0x0001} & Flow is Modbus\\
    {\tt 0x0002} & Non-modbus protocol identifier\\
    {\tt 0x0004} & Unknown function code\\
    {\tt 0x0008} & Unknown exception code\\
    {\tt 0x0010} & Multiple unit identifiers\\
    {\tt 0x0100} & List of function codes truncated\ldots increase {\tt MB\_NUM\_FUNC}\\
    {\tt 0x0200} & List of function codes which caused exceptions truncated\ldots increase {\tt MB\_NUM\_FEX}\\
    {\tt 0x0400} & List of exception codes truncated\ldots increase {\tt MB\_NUM\_EX}\\
    {\tt 0x4000} & Snapped packet\\
    {\tt 0x8000} & Malformed packet\\
    \bottomrule
\end{longtable}

\subsubsection{modbusFC and modbusFCBF}\label{modbusFC}
The {\tt modbusFC} and {\tt modbusFCBF} columns are to be interpreted as follows:
\begin{longtable}{rll}
    \toprule
    {\bf modbusFC} & {\bf modbusFCBF} & {\bf Description}\\
    \midrule\endhead%
    % Data Access
    {\tt  1 = 0x01} & {\tt 0x0000 0000 0000 0002} & Read Coils\\
    {\tt  2 = 0x02} & {\tt 0x0000 0000 0000 0004} & Read Discrete Inputs\\
    {\tt  3 = 0x03} & {\tt 0x0000 0000 0000 0008} & Read Multiple Holding Registers\\
    {\tt  4 = 0x04} & {\tt 0x0000 0000 0000 0010} & Read Input Registers\\
    {\tt  5 = 0x05} & {\tt 0x0000 0000 0000 0020} & Write Single Coil\\
    {\tt  6 = 0x06} & {\tt 0x0000 0000 0000 0040} & Write Single Holding Register\\
    {\tt  7 = 0x07} & {\tt 0x0000 0000 0000 0080} & Read Exception Status\\
    {\tt  8 = 0x08} & {\tt 0x0000 0000 0000 0100} & Diagnostic\\
    {\tt 11 = 0x0b} & {\tt 0x0000 0000 0000 0800} & Get Com Event Counter\\
    {\tt 12 = 0x0c} & {\tt 0x0000 0000 0000 1000} & Get Com Event Log\\
    {\tt 15 = 0x0f} & {\tt 0x0000 0000 0000 8000} & Write Multiple Coils\\
    {\tt 16 = 0x10} & {\tt 0x0000 0000 0001 0000} & Write Multiple Holding Registers\\
    {\tt 17 = 0x11} & {\tt 0x0000 0000 0002 0000} & Report Slave ID\\
    {\tt 20 = 0x14} & {\tt 0x0000 0000 0010 0000} & Read File Record\\
    {\tt 21 = 0x15} & {\tt 0x0000 0000 0020 0000} & Write File Record\\
    {\tt 22 = 0x16} & {\tt 0x0000 0000 0040 0000} & Mask Write Register\\
    {\tt 23 = 0x17} & {\tt 0x0000 0000 0080 0000} & Read/Write Multiple Registers\\
    {\tt 24 = 0x18} & {\tt 0x0000 0000 0100 0000} & Read FIFO Queue\\
    {\tt 43 = 0x2b} & {\tt 0x0000 0800 0000 0000} & Read Decide Identification\\
    \bottomrule
\end{longtable}

\subsubsection{modbusFEx and modbusFExBF}\label{modbusFEx}
The {\tt modbusFEx} and {\tt modbusFExBF} columns are to be interpreted as {\tt\hyperref[modbusFC]{modbusFC}} and {\tt\hyperref[modbusFC]{modbusFCBF}}, respectively.

\subsubsection{modbusExC and modbusExCBF}\label{modbusExC}
The {\tt modbusExC} and {\tt modbusExCBF} column are to be interpreted as follows:
\begin{longtable}{rrl}
    \toprule
    {\bf modbusExC} & {\bf modbusExCBF} & {\bf Description}\\
    \midrule\endhead%
    {\tt  1 = 0x01} & {\tt 0x0002} & Illegal function code\\
    {\tt  2 = 0x02} & {\tt 0x0004} & Illegal data address\\
    {\tt  3 = 0x03} & {\tt 0x0008} & Illegal data value\\
    {\tt  4 = 0x04} & {\tt 0x0010} & Slave device failure\\
    {\tt  5 = 0x05} & {\tt 0x0020} & Acknowledge\\
    {\tt  6 = 0x06} & {\tt 0x0040} & Slave device busy\\
    {\tt  7 = 0x07} & {\tt 0x0080} & Negative acknowledge\\
    {\tt  8 = 0x08} & {\tt 0x0100} & Memory parity error\\
    {\tt 10 = 0x0a} & {\tt 0x0400} & Gateway path unavailable\\
    {\tt 11 = 0x0b} & {\tt 0x0800} & Gateway target device failed to respond\\
    \bottomrule
\end{longtable}

\subsection{Packet File Output}
In packet mode ({\tt --s} option), the modbus plugin outputs the following columns:
\begin{longtable}{llll}
    \toprule
    {\bf Column} & {\bf Type} & {\bf Description} & {\bf Flags}\\
    \midrule\endhead%
    {\tt mbTranId} & U16 & Transaction Identifier\\
    {\tt mbProtId} & U16 & Protocol Identifier\\
    {\tt mbLen}    & U16 & Length\\
    {\tt mbUnitId} & U8 & Unit identifier\\
    {\tt \nameref{mbFuncCode}} & H8 & Function code & {\tt MB\_FE\_FRMT=0}\\
    {\tt \nameref{mbFuncCode}} & U8 & Function code & {\tt MB\_FE\_FRMT=1}\\
    \bottomrule
\end{longtable}

\subsubsection{mbFuncCode}\label{mbFuncCode}
If {\tt mbFuncCode} column is to be interpreted as follows:
\begin{longtable}{ll}
    \toprule
    {\bf mbFuncCode} & {\bf Description}\\
    \midrule\endhead%
    $< 128$ (={\tt 0x80}) & refer to \nameref{modbusFC}\\
    $\geq 128$ (={\tt 0x80}) & subtract 128 (={\tt 0x80}) and refer to \nameref{modbusFEx}\\
    \bottomrule
\end{longtable}

\subsection{Plugin Report Output}
The number of Modbus packets is reported.

\end{document}
