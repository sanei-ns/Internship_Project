\IfFileExists{t2doc.cls}{
    \documentclass[documentation]{subfiles}
}{
    \errmessage{Error: could not find 't2doc.cls'}
}

\begin{document}

\trantitle
    {descriptiveStats}
    {Descriptive Statistics}
    {Tranalyzer Development Team} % author(s)

\section{descriptiveStats}\label{s:descriptiveStats}

\subsection{Description}
The descriptiveStats plugin calculates various statistics about a flow.
Because the inter-arrival time of the first packet is per definition always zero, it is removed from the statistics.
Therefore the inter-arrival time statistics values for flows with only one packet is set to zero.

\subsection{Dependencies}

%\traninput{file} % use this command to input files
%\traninclude{file} % use this command to include files

%\tranimg{image} % use this command to include an image (must be located in a subfolder ./img/)

%\subsubsection{External Libraries}
%This plugin depends on ...

\subsubsection{Other Plugins}
This plugin requires the \tranrefpl{pktSIATHisto} plugin.

%\subsubsection{Required Files}
%The file ... is required.

\subsection{Configuration Flags}
The following flags can be used to control the output of the plugin:
\begin{longtable}{lcl}
    \toprule
    {\bf Name} & {\bf Default} & {\bf Description} \\
    \midrule\endhead%
    {\tt ENABLE\_PS\_CALC}  & 1 & 1: Enables / 0: Disables calculation of statistics for packet sizes\\
    {\tt ENABLE\_IAT\_CALC} & 1 & 1: Enables / 0: Disables calculation of statistics for inter-arrival times\\
    \bottomrule
\end{longtable}

\subsection{Flow File Output}
The descriptiveStats plugin outputs the following columns:
\begin{longtable}{llll}
    \toprule
    {\bf Column} & {\bf Type} & {\bf Description} & {\bf Flags}\\
    \midrule\endhead%
    %\\
    %\multicolumn{3}{l}{If {\tt ENABLE\_PS\_CALC=1}, the following columns are displayed:}\\\\
    {\tt MinPl}          & F & Minimum packet length                       & {\tt ENABLE\_PS\_CALC=1}\\
    {\tt MaxPl}          & F & Maximum packet length                       & {\tt ENABLE\_PS\_CALC=1}\\
    {\tt MeanPl}         & F & Mean packet length                          & {\tt ENABLE\_PS\_CALC=1}\\
    {\tt LowQuartilePl}  & F & Lower quartile of packet lengths            & {\tt ENABLE\_PS\_CALC=1}\\
    {\tt MedianPl}       & F & Median of packet lengths                    & {\tt ENABLE\_PS\_CALC=1}\\
    {\tt UppQuartilePl}  & F & Upper quartile of packet lengths            & {\tt ENABLE\_PS\_CALC=1}\\
    {\tt IqdPl}          & F & Inter quartile distance of packet lengths   & {\tt ENABLE\_PS\_CALC=1}\\
    {\tt ModePl}         & F & Mode of packet lengths                      & {\tt ENABLE\_PS\_CALC=1}\\
    {\tt RangePl}        & F & Range of packet lengths                     & {\tt ENABLE\_PS\_CALC=1}\\
    {\tt StdPl}          & F & Standard deviation of packet lengths        & {\tt ENABLE\_PS\_CALC=1}\\
    {\tt RobStdPl}       & F & Robust standard deviation of packet lengths & {\tt ENABLE\_PS\_CALC=1}\\
    {\tt SkewPl}         & F & Skewness of packet lengths                  & {\tt ENABLE\_PS\_CALC=1}\\
    {\tt ExcPl}          & F & Excess of packet lengths                    & {\tt ENABLE\_PS\_CALC=1}\\
    \\
    %\multicolumn{3}{l}{If {\tt ENABLE\_IAT\_CALC=1}, the following columns are displayed:}\\\\
    {\tt MinIat}         & F & Minimum inter-arrival time                       & {\tt ENABLE\_IAT\_CALC=1}\\
    {\tt MaxIat}         & F & Maximum inter-arrival time                       & {\tt ENABLE\_IAT\_CALC=1}\\
    {\tt MeanIat}        & F & Mean inter-arrival time                          & {\tt ENABLE\_IAT\_CALC=1}\\
    {\tt LowQuartileIat} & F & Lower quartile of inter-arrival times            & {\tt ENABLE\_IAT\_CALC=1}\\
    {\tt MedianIat}      & F & Median of inter-arrival times                    & {\tt ENABLE\_IAT\_CALC=1}\\
    {\tt UppQuartileIat} & F & Upper quartile of inter-arrival times            & {\tt ENABLE\_IAT\_CALC=1}\\
    {\tt IqdIat}         & F & Inter quartile distance of inter-arrival times   & {\tt ENABLE\_IAT\_CALC=1}\\
    {\tt ModeIat}        & F & Mode of inter-arrival times                      & {\tt ENABLE\_IAT\_CALC=1}\\
    {\tt RangeIat}       & F & Range of inter-arrival times                     & {\tt ENABLE\_IAT\_CALC=1}\\
    {\tt StdIat}         & F & Standard deviation of inter-arrival times        & {\tt ENABLE\_IAT\_CALC=1}\\
    {\tt RobStdIat}      & F & Robust standard deviation of inter-arrival times & {\tt ENABLE\_IAT\_CALC=1}\\
    {\tt SkewIat}        & F & Skewness of inter-arrival times                  & {\tt ENABLE\_IAT\_CALC=1}\\
    {\tt ExcIat}         & F & Excess of inter-arrival times                    & {\tt ENABLE\_IAT\_CALC=1}\\
    \bottomrule
\end{longtable}

%\subsection{Custom File Output}
%Non-standard output

%\subsection{Plugin Report Output}

%\subsection{Post-Processing}

%\subsection{Example Output}

\subsection{Known Bugs and Limitations}
Because the packet length and inter-arrival time plugin stores the inter-arrival times in statistical bins the original time information is lost. Therefore the calculation of the inter-arrival times statistics is due to its logarithmic binning only a rough approximation of the original timing information. Nevertheless, this representation has shown to be useful in practical cases of anomaly and application classification.

%\subsection{TODO}
%\begin{itemize}
%    \item TODO1
%    \item TODO2
%\end{itemize}

\end{document}
