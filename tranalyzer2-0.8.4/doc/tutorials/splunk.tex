\IfFileExists{t2doc.cls}{
    \documentclass[documentation]{subfiles}
}{
    \errmessage{Error: could not find 't2doc.cls'}
}

\begin{document}

\trantitle
    {Importing Tranalyzer Flows in Splunk}
    {Type of Import: JSON Stream} % Short description
    {Tranalyzer Development Team} % author(s)

\section{Importing Tranalyzer Flows in Splunk}\label{splunk_tutorial}

\subsection{Prerequisites}
\begin{itemize}
    \item Tranalyzer version 0.6.x is installed with standard/default plugins,
    \item Splunk 6.5.x is installed, Splunk account exists,
    \item At least one network interface (Ethernet or WLAN) has network traffic.
\end{itemize}

\subsection{Select Network Interface}

Determine the network interface name by entering the following command:
\begin{verbatim}
ifconfig
\end{verbatim}
at the terminal command line. In the output look for the interface
name which has the IP address where the network traffic should be
collected from:
\begin{verbatim}
en0: flags=8863<UP,BROADCAST,SMART,RUNNING,SIMPLEX,MULTICAST>
mtu 1500 inet 10.20.6.79 netmask 0xfffffc00 broadcast 10.20.7.255
\end{verbatim}

\subsection{Configure Tranalyzer jsonSink Plugin}

Go to {\em tranalyzer2-0.6.XlmY/trunk/jsonSink/src/jsonSink.h} and
set the configuration parameters as needed:
\begin{verbatim}
#define SOCKET_ON             1 // Whether to output to a socket (1) or file (0)
#define SOCKET_ADDR "127.0.0.1" // address of the socket
#define SOCKET_PORT        5000 // port of the socket
\end{verbatim}
Set {\tt SOCKET\_ON} to {\tt 1} to configure the output to a socket. Set the
IP address of the destination server which should receive the data
stream. If the localhost will be the destination, leave the default setting
{\tt "127.0.0.1"}. Set the socket server port of the destination.

\subsection{Recompile the jsonSink Plugin}

Enter the following command:
\begin{verbatim}
tranalyzer2-0.6.8lm4/trunk/jsonSink/autogen.sh
\end{verbatim}
Make sure that the plugin is compiled successfully. In this case the
following message will be shown at the command line:
\begin{verbatim}
Plugin jsonSink copied into USER_DIRECTORY/.tranalyzer/plugins
\end{verbatim}

\subsection{Start Tranalyzer2}

Start generating flow records by launching Tranalyzer2 with the interface
name determined on the previous step and setting a file name as the
command line arguments by entering the command:
\begin{verbatim}
tranalyzer -i en0 -w test1 &
\end{verbatim}

Note that the file name is optional for JSON stream import, if file
name is not indicated the records will be shown in the standard output
(besides being streamed over the configured TCP socket).

\subsubsection{Check File Output}

Check that the flow records are written to the file by entering the
command:
\begin{verbatim}
tail -f test1_flows.txt
\end{verbatim}
Flow records should be shown in the terminal.

\subsubsection{Collect Traffic}

Let Tranalyzer2 run and collect network traffic.

\subsection{Start Splunk}

Start Splunk by entering the following command:
\begin{verbatim}
splunk start
\end{verbatim}
in the directory where Splunk is installed. Wait for the confirmation
message that Splunk is up and running:
\begin{verbatim}
The Splunk web interface is at http://splunk_hostname:8000
\end{verbatim}

\subsection{Login to Splunk, Import and Search Data }

\begin{figure}[!ht]
    \centering
    \tranimg[width=.98\textwidth]{splunk/CapturFiles-07-42-2016_03-42-46}
    \caption{Select ``Add Data''.}
\end{figure}

\begin{figure}[!ht]
    \centering
    \tranimg[width=.98\textwidth]{splunk/CapturFiles-07-43-2016_03-43-12}
    \caption{Select ``TCP/UDP'' and set protocol to ``TCP'' and set the correct
             port number (same as in the Tranalyzer2 plugin configuration file,
             in this example --- 5000).}
\end{figure}

\begin{figure}[!ht]
    \centering
    \tranimg[width=.98\textwidth]{splunk/CapturFiles-07-43-2016_03-43-50}
    \caption{Select ``\_json'' as Source Type and proceed to ``Review''.}
\end{figure}

\begin{figure}[!ht]
    \centering
    \tranimg[width=.98\textwidth]{splunk/CapturFiles-07-44-2016_03-44-28}
    \caption{Select ``Start Searching'' to make sure that the data is being received by Splunk.}
\end{figure}

\begin{figure}[!ht]
    \centering
    \tranimg[width=.98\textwidth]{splunk/CapturFiles-07-48-2016_03-48-05}
    \caption{Note that the data is being received, but the Tranalyzer2 specific
        data record field are not shown yet.}
\end{figure}

\begin{figure}[!ht]
    \centering
    \tranimg[width=.98\textwidth]{splunk/CapturFiles-07-52-2016_03-52-46}
    \caption{Go to ``Settings''->''DATA''->''Source Types'' and click on
        ``\_json'' data source type to edit it.}
\end{figure}

\begin{figure}[!ht]
    \centering
    \tranimg[width=.98\textwidth]{splunk/CapturFiles-07-02-2016_04-02-20}
    \caption{Change option ``KV\_MODE'' from ``none'' to ``json'' and save
        the changes.}
\end{figure}

\begin{figure}[!ht]
    \centering
    \tranimg[width=.98\textwidth]{splunk/CapturFiles-07-54-2016_03-54-33}
    \caption{Return to the Search window and make sure that the Tranalyzer2
             specific fields are recognized by Splunk.}
\end{figure}

\begin{figure}[!ht]
    \centering
    \tranimg[width=.98\textwidth]{splunk/CapturFiles-07-55-2016_03-55-15}
    \caption{Query data, e.g. show top destination IP addresses by number of the records.}
\end{figure}

\end{document}
