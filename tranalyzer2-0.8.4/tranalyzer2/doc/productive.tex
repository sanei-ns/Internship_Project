\section{Tranalyzer2}\label{s:tranalyzer2}
Tranalyzer2 is designed in a modular way. Thus, the packet flow aggregation and the flow statistics are separated. While the main program performs the header dissection and flow organisation, the plugins produce specialized output such as packet statistics, mathematical transformations, signal analysis and result file generation.

\subsection{Supported Link-Layer Header Types}
Tranalyzer handles most PCAP link-layer header types automatically. Some specific types can be analyzed by switching on flags in {\tt linktypes.h}. The following table summarises the link-layer header types handled by Tranalyzer:
\begin{longtable}{lll}
    \toprule
    {\bf Linktype} & {\bf Description} & {\bf Flags}\\
    \midrule\endhead%
    {\tt\small DLT\_C\_HDLC} & Cisco PPP with HDLC framing\\
    {\tt\small DLT\_C\_HDLC\_WITH\_DIR} & Cisco PPP with HDLC framing preceded by one byte direction\\
    {\tt\small DLT\_EN10MB} & IEEE 802.3 Ethernet (10Mb, 100Mb, 1000Mb and up)\\
    {\tt\small DLT\_FRELAY} & Frame Relay\\
    {\tt\small DLT\_FRELAY\_WITH\_DIR} & Frame Relay preceded by one byte direction\\
    {\tt\small DLT\_IEEE802\_11} & IEEE802.11 wireless LAN\\
    {\tt\small DLT\_IEEE802\_11\_RADIO} & Radiotap link-layer information followed by an 802.11 header\\
    {\tt\small DLT\_IPV4} & Raw IPv4\\
    {\tt\small DLT\_IPV6} & Raw IPv6\\
    {\tt\small DLT\_JUNIPER\_ATM1} & Juniper ATM1 PIC (experimental) & {\tt LINKTYPE\_JUNIPER=1}\\
    {\tt\small DLT\_JUNIPER\_ETHER} & Juniper Ethernet (experimental) & {\tt LINKTYPE\_JUNIPER=1}\\
    {\tt\small DLT\_JUNIPER\_PPPOE} & Juniper PPPoE PIC (experimental) & {\tt LINKTYPE\_JUNIPER=1}\\
    {\tt\small DLT\_LINUX\_SLL} & Linux ``cooked'' capture encapsulation\\
    {\tt\small DLT\_NULL} & BSD loopback encapsulation\\
    {\tt\small DLT\_PPI} & Per-Packet Information\\
    {\tt\small DLT\_PPP} & Point-to-Point Protocol\\
    {\tt\small DLT\_PPP\_SERIAL} & PPP in HDLC-like framing\\
    {\tt\small DLT\_PPP\_WITH\_DIR} & PPP preceded by one byte direction\\
    {\tt\small DLT\_PRISM\_HEADER} & Prism monitor mode information followed by an 802.11 header\\
    {\tt\small DLT\_RAW} & Raw IP\\
    {\tt\small DLT\_SYMANTEC\_FIREWALL} & Symantec Enterprise Firewall\\
    \bottomrule
\end{longtable}

\subsection{Enabling/Disabling Plugins}\label{ss:plugins}
The plugins are stored under {\em\textasciitilde{}/.tranalyzer/plugins}.
This folder can be changed with the {\tt --p} option.\\

By default, all the plugins found in the plugin folder are loaded.
This behaviour can be changed by altering the value of {\tt USE\_PLLIST} in {\em loadPlugins.h:12}. The valid options are
\begin{longtable}{rl}
    \toprule
    {\tt USE\_PLLIST} & {\bf Description}\\
    \midrule\endhead%
    0 & load all plugins in the plugin folder (default)\\ %\lstinline{^[0-9]\{3\}_[a-zA-Z0-9]+.so$} (default)\\
    1 & use a whitelist (loading list)\\
    2 & use a blacklist\\
    \bottomrule
\end{longtable}

This following sections discuss the various ways to selectively enable/disable plugins.

\subsubsection{Default}
By default, all the files in the plugin folder named according to the following pattern are loaded:\\
\begin{longtable}{>{\ttfamily}c}
    \textasciicircum[0-9]\{3\}\_[a--zA--Z0--9]+.so\$
\end{longtable}
To disable a plugin, it must be removed from the plugin folder. A subfolder, e.g., {\em disabled}, can be used to store unused plugins.

\subsubsection{Whitelisting Plugins}
If {\tt USE\_PLLIST=1}, the whitelist (loading list) is searched under the plugins folder with the name {\tt plugins.txt}.
The name can be changed by adapting the value {\tt PL\_LIST} in {\em loadPlugins.h:13}.
If the file is stored somewhere else, Tranalyzer2 {\tt --b} option can be used.

The format of the whitelist is as follows (empty lines and lines starting with a {\tt `\#'} are ignored):
\begin{verbatim}
        # This is a comment

        # This plugin is whitelisted (will be loaded)
        001_protoStats.so

        # This plugin is NOT whitelisted (will NOT be loaded)
        #010_basicFlow.so
\end{verbatim}

Note that if a plugin is not present in the list, it will NOT be loaded.

\subsubsection{Blacklisting Plugins}
If {\tt USE\_PLLIST=2}, the blacklist is searched under the plugins folder with the name {\tt plugins.txt}.
The name can be changed by adapting the value {\tt PL\_LIST} in {\em loadPlugins.h:13}.
If the file is stored somewhere else, Tranalyzer2 {\tt --b} option can be used.

The format of the blacklist is as follows (empty lines and lines starting with a {\tt `\#'} are ignored):
\begin{verbatim}
        # This is a comment

        # This plugin is blacklisted (will NOT be loaded)
        001_protoStats.so

        # This plugin is NOT blacklisted (will be loaded)
        #010_basicFlow.so
\end{verbatim}

\subsubsection{Graphical Configuration and Building of T2 and Plugins}\label{sss:t2conf}
Tranalyzer2 comes with a script named {\tt t2conf} allowing easy configuration of all the plugins through a command line based graphical menu:

\begin{figure}[!ht]
    \centering
    \begin{minipage}[b]{0.45\linewidth}
        \tranimg[width=\textwidth]{t2conf1}
    \end{minipage}
    ~
    \begin{minipage}[b]{0.45\linewidth}
        \tranimg[width=\textwidth]{t2conf2}
    \end{minipage}
\end{figure}

Use the arrows on your keyboard to navigate up and down and between the buttons. The first window is only displayed if the {\tt --t2} option is used. The {\tt Edit} and {\tt Configure} buttons will launch a text editor ({\tt \$EDITOR} or vim\footnote{The default editor can be changed by editing the variable {\tt DEFAULT\_EDITOR} (line 7)} if the environment variable is not defined). The second window can be used to activate and deactivate plugins (toggle the active/inactive state with the space key).

To access the script from anywhere, use the provided {\tt install.sh} script, install \hyperref[s:aliases]{\tt t2\_aliases} or manually add the following alias to {\tt \textasciitilde{}/.bash\_aliases}:
\begin{center}
{\tt alias t2conf="\$T2HOME/scripts/t2conf/t2conf"}
\end{center}
Where {\tt \$T2HOME} is the folder containing the source code of Tranalyzer2 and its plugins.

A man page for {\tt t2conf} is also provided and can be installed with the {\tt install.sh} script.
%\begin{verbatim}
%Usage:
%    ./t2conf [OPTION...]
%
%Optional arguments:
%    -t path        set tranalyzer folder to `path' [~/tranalyzer2]
%    -p path        set plugin folder to `path' [~/.tranalyzer/plugins]
%
%    -s [file]      save loaded plugins in `file' [/tmp/t2pl_state]
%    -r [file]      restore previous state from `file' [/tmp/t2pl_state]
%
%    -t2            configure and build Tranalyzer2 first
%    -i             install Tranalyzer2 in /usr/local/bin
%
%    --patch file   configures tranalyzer according to `file' whose format is
%                   FLAG NEWVAL OLDVAL FILE, where FILE is relative to `-t'
%    --rpatch file  same as -patch, but restores OLDVAL
%
%    --ip6          activates IPv6 dissection (deactivates IPv4)
%    --ip4          activates IPv4 dissection (deactivates IPv6)
%    --eth          activates Ethernet flows
%    --noeth        deactivates Ethernet flows
%    --mon          activates monitoring mode
%
%    -a             activates all plugins
%    -0             deactivates all plugins
%
%    -9             activates pcapd only
%    -q             deactivates pcapd and restore previous state
%
%    -x             exit after performing -t2, -a, -r, -p and -q option
%
%    -b             silently rebuild all active plugins (no menu)
%    -c             run $EDITOR for every plugin that needs to be built
%    -f             force rebuild of the plugins (./autogen.sh)
%    -d             build plugins in debug mode (./autogen.sh -d)
%
%    -l             lazy mode (only build plugins if they do not exist)
%
%    -n             sort by plugin name instead of plugin number
%
%    -v             enable verbose mode
%
%    -h             display this help and exit
%\end{verbatim}

\subsection{Man Page}
If the man page was installed (\refs{t2install}), then accessing the man page is as simple as calling

\begin{center}
    {\tt man tranalyzer}\\
\end{center}

If it was not installed, then the man page can be invoked by calling

\begin{center}
{\tt man \$T2HOME/tranalyzer2/man/tranalyzer.1}
\end{center}

\subsection{Invoking Tranalyzer}
As stated earlier Tranalyzer2 either operates on Ethernet/DAG interfaces or pcap files. It may be invoked using a BPF if only certain flows are interesting. The required arguments are listed below. Note that the {\tt --i}, {\tt --r}, {\tt --R} and {\tt --D} options cannot be used at the same time.

%To access Tranalyzer2 from anywhere, the following alias can be added to {\tt \textasciitilde{}/.bash\_aliases}:
%\begin{center}
%{\tt alias t2="\$T2HOME/tranalyzer2/src/tranalyzer"}
%\end{center}
%Where {\tt \$T2HOME} is the folder containing the source code of Tranalyzer2 and its plugins.

\subsubsection{Help}
For a full list of options, use the {\tt --h} option: {\tt tranalyzer --h}

\begin{verbatim}
Tranalyzer 0.8.2 (Anteater), Tarantula - High performance flow based network traffic analyzer

Usage:
    tranalyzer [OPTION...] <INPUT>

Input:
    -i IFACE     Listen on interface IFACE
    -r PCAP      Read packets from PCAP file or from stdin if PCAP is "-"
    -R FILE      Process every PCAP file listed in FILE
    -D EXPR[:SCHR][,STOP]
                 Process every PCAP file whose name matches EXPR, up to an
                 optional last index STOP. If STOP is omitted, then Tranalyzer
                 never stops. EXPR can be a filename, e.g., file.pcap0, or an
                 expression, such as "dump*.pcap00", where the star matches
                 anything (note the quotes to prevent the shell from
                 interpreting the expression). SCHR can be used to specify the
                 the last character before the index (default: 'p')

Output:
    -w PREFIX    Append PREFIX to any output file produced. If omitted, then
                 output is diverted to stdout
    -W PREFIX[:SIZE][,START]
                 Like -w, but fragment flow files according to SIZE, producing
                 files starting with index START. SIZE can be specified in bytes
                 (default), KB ('K'), MB ('M') or GB ('G'). Scientific notation,
                 i.e., 1e5 or 1E5 (=100000), can be used as well. If a 'f' is
                 appended, e.g., 10Kf, then SIZE denotes the number of flows.
    -l           Print end report in PREFIX_log.txt instead of stdout
    -s           Packet forensics mode

Optional arguments:
    -p PATH      Load plugins from path PATH instead of ~/.tranalyzer/plugins
    -b FILE      Use plugin list FILE instead of plugin_folder/plugins.txt
    -e FILE      Creates a PCAP file by extracting all packets belonging to
                 flow indexes listed in FILE
    -f FACTOR    Sets hash multiplication factor
    -x ID        Sensor ID
    -c CPU       Bind tranalyzer to one core. If CPU is 0 then OS selects the
                 core to bind
    -F FILE      Read BPF filter from FILE

    -v           Show the version of the program and exit

    -h           Show help options and exit

Remaining arguments:
    BPF          Berkeley Packet Filter command, as in tcpdump
\end{verbatim}

\subsubsection{--i INTERFACE}
Capture data from an Ethernet interface {\tt INTERFACE} (requires {\em root} privileges).
If high volume of traffic is expected, then enable internal buffering in \nameref{ioBuffer.h}.

\begin{center}
    {\tt tranalyzer --i eth0 --w out}
\end{center}

\subsubsection{--r FILE}
Capture data from a pcap file {\tt FILE}.
\begin{center}
    {\tt tranalyzer --r file.pcap --w out}
\end{center}

The special file {\tt `-'} can be used to read data from {\em stdin}. This can be used, e.g., to process compressed pcap files, e.g., {\em file.pcap.gz}, using the following command:
\begin{center}
    {\tt zcat file.pcap.gz | tranalyzer --r -- --w out}
\end{center}

\subsubsection{--R FILE}
Process all the pcap files listed in {\tt FILE}. All files are being treated as one large file. The life time of a flow can extend over many files. The processing order is defined by the location of the filenames in the text file. The absolute path has to be specified. The {\tt gpl} script documented in {\tt \$T2HOME/scripts/scripts.pdf} can be used to generate such a list. All lines starting with a {\tt `\#'} are considered as comments and thus ignored.

\begin{figure}[!ht]
\centering
\begin{verbbox}
cd ~/pcap/
$T2HOME/scripts/gpl > pcap_list.txt
tranalyzer -R pcap_list.txt -w out
\end{verbbox}
%stopzone  % hack to correct vim syntax highlighting
\theverbbox
\end{figure}

\subsubsection{--D FILE[*][.ext]\#1[:SCHR][,\#2]}
Process files in a directory using file start and stop index, defined by {\tt \#1} and {\tt \#2} respectively. {\tt ext} can be anything, e.g., {\tt .pcap}, and can be omitted. If {\tt \#2} is omitted and not in round robin mode, then Tranalyzer2 never stops and waits until the next file in the increment is available. If leading zeroes are used, {\tt \#2} defaults to $10^\text{number\_length}-1$. Note that only the last occurence of {\tt SCHR} is considered, e.g., if {\tt SCHR=`p'}, then out.pca{\bf p}001 will work, but out001pca{\bf p}, will not. with the {\tt :[SCHR]} option a new separation character can be set, superseeding {\tt SCHR} defined in \nameref{tranalyzer.h}.\\

The following variables in \nameref{tranalyzer.h} can be used to configure this mode:

\begin{figure}[!ht]
\begin{longtable}{lcll}
    \toprule
    {\bf Name} & {\bf Default} & {\bf Description}\\
    \midrule\endhead%
    {\tt RROP}     & 0 & Activate round robin operations\\
                   &   & WARNING: if set to 1, then findexer will not work anymore\\
    {\tt POLLTM}   & 5 & Poll timing (in seconds) for files \\
    {\tt MFPTMOUT} & 0 & > 0: timeout for poll > POLLTM, 0: no poll timout \\
    {\tt SCHR} & {\tt `p'} & Separating character for file number\\
    \bottomrule
\end{longtable}
\end{figure}

For example, when using {\tt tcpdump} to capture traffic from an interface (eth0) and produce 100MB files as follows:
\begin{center}
    {\tt sudo tcpdump -C 100 --i eth0 --w out.pcap}
\end{center}
The following files are generated: {\em out.pcap, out.pcap1, out.pcap2, \ldots, out.pcap10, \ldots}\\

Then {\tt SCHR} must be set to {\tt `p'}, i.e., the last character before the file number (out.pca{\bf p}NUM) and Tranalyzer must be run as follows:
\begin{center}
    {\tt tranalyzer --D out.pcap --w out}
\end{center}

Or to process files 10 to 100:
\begin{center}
    {\tt tranalyzer --D out.pcap10,100 --w out}
\end{center}

Or to process files 10 to 100 in another format:
\begin{center}
    {\tt tranalyzer --D out10.pcap,100 --w out}
\end{center}

Or to process files from 0 to $2^{32}-1$ using regex characters:
\begin{center}
    {\tt tranalyzer --D "out*.pcap" --w out}
\end{center}

The last command can be shortened further, the only requirement being the presence of {\tt SCHR} (the last character before the file number) in the pattern:
\begin{center}
    {\tt tranalyzer --D "*p" --w out}
\end{center}
Note the quotes ({\tt "}) which are necessary to avoid preemptive interpretation of regex characters and {\tt SCHR} which MUST appear in the pattern. The same configuration can be used for filenames using one or more leading zeros, e.g., {\em out.pcap000, out.pcap001, out.pcap002, \ldots, out.pcap010, \ldots}\\

The following table summarises the supported naming patterns and the configuration required:
\begin{longtable}{lll}
    \toprule
    {\bf Filenames} & {\bf SCHR} & {\bf Command}\\
    \midrule\endhead%
    out.pca{\bf p}, out.pca{\bf p}1, out.pca{\bf p}2, \ldots     & {\tt `p'} & {\tt tranalyzer -D out.pcap -w out}\\
    out.pca{\bf p}00, out.pca{\bf p}01, out.pca{\bf p}02, \ldots & {\tt `p'} & {\tt tranalyzer -D out.pcap00 -w out}\\
    %ou{\bf t}.pcap, ou{\bf t}1.pcap, ou{\bf t}2.pcap, \ldots     & {\tt `t'} & {\tt tranalyzer -D "out*.pcap" -w out}\\
    ou{\bf t}0.pcap, ou{\bf t}1.pcap, ou{\bf t}2.pcap, \ldots    & {\tt `t'} & {\tt tranalyzer -D out0.pcap -w out}\\
    ou{\bf t}00.pcap, ou{\bf t}01.pcap, ou{\bf t}02.pcap, \ldots & {\tt `t'} & {\tt tranalyzer -D out00.pcap -w out}\\
    out\_24.04.2016.20h00.pca{\bf p},\\
    \qquad out\_24.04.2016.20h00.pca{\bf p}1, \ldots             & {\tt `p'} & {\tt tranalyzer -D "out*.pcap" -w out}\\
    out\_24.04.2016.20h00.pca{\bf p}00, & \\
    \qquad out\_24.04.2016.20h00.pca{\bf p}01, \ldots            & {\tt `p'} & {\tt tranalyzer -D "out*.pcap00" -w out}\\
    ou{\bf t}0.pcap, ou{\bf t}1.pcap, ou{\bf t}2.pcap, \ldots    & {\tt `t'} & {\tt tranalyzer -D out0.pcap:t -w out}\\
    out.pca{\bf p}00, out.pca{\bf p}01, out.pca{\bf p}02, \ldots & {\tt `p'} & {\tt tranalyzer -D out.pcap00:p -w out}\\
    \bottomrule
\end{longtable}

\subsubsection{--w PREFIX}
Use a {\tt PREFIX} for all output file types. The number of files being produced vary with the number of activated plugins. The file suffixes are defined in the file \nameref{tranalyzer.h} (see \refs{file_suffixes}) or in the header files for the plugins. If you forget to specify an output file, Tranalyzer will use the input interface name or the file name as file prefix and print the flows to {\em stdout}. Thus, Tranalyzer output can be piped into other command line tools, such as netcat in order to produce centralized logging to another host or an AWK script for further post processing without intermediate writing to a slow disk storage.

\subsubsection{--W PREFIX[:SIZE][,START]}
This option allows the fragmentation of flow files produced by Tranalyzer independent of the input mode. The expression before the {\tt `:'} is the output prefix, the expression after the {\tt `:'} denotes the maximal file size for each fragment and the number after the {\tt `,'} denotes the start index of the first file. If omitted it defaults to {\tt 0}. The size of the files can be specified in bytes (default), KB ({\tt `K'}), MB ({\tt `M'}) or GB ({\tt `G'}). Scientific notation, i.e., 1e5 or 1E5 (=100000), can be used as well. If no size is specified, the default value of 500MB, defined by {\tt OFRWFILELN} in \nameref{tranalyzer.h} is used. If no size is specified, then the {\tt `:'} can be omitted. The same happens if no start index is specified. If an additional {\tt `f'} is appended the unit is flow count. This enables the user to produce file chunks containing the same amount of flows. Some typical examples are shown below.

\begin{longtable}{lccl}
    \toprule
    {\bf Command} & {\bf Fragment Size} & {\bf Start Index} & {\bf Output Files}\\
    \midrule\endhead%
    {\tt tranalyzer --r nudel.pcap --W out:1.5E9,10} & 1.5GB            & 10 & out10, out11, \ldots\\
    {\tt tranalyzer --r nudel.pcap --W out:1.5e9,5}  & 1.5GB            &  5 & out5, out6, \ldots\\
    {\tt tranalyzer --r nudel.pcap --W out:1.5G,1}   & 1.5GB            &  1 & out1, out2, \ldots\\
    {\tt tranalyzer --r nudel.pcap --W out:5000K}    & 0.5MB            &  0 & out0, out1, \ldots\\
    {\tt tranalyzer --r nudel.pcap --W out:5Kf}      & 5000 Flows       &  0 & out0, out1, \ldots\\
    {\tt tranalyzer --r nudel.pcap --W out:180M}     & 180MB            &  0 & out0, out1, \ldots\\
    {\tt tranalyzer --r nudel.pcap --W out:2.5G}     & 2.5GB            &  0 & out0, out1, \ldots\\
    {\tt tranalyzer --r nudel.pcap --W out,5}        & {\tt OFRWFILELN} &  0 & out0, out1, \ldots\\
    {\tt tranalyzer --r nudel.pcap --W out}          & {\tt OFRWFILELN} &  0 & out0, out1, \ldots\\
    \bottomrule
\end{longtable}

\subsubsection{--l}\label{t2-loption}
All Tranalyzer command line and report output is diverted to the log file: {\tt PREFIX\_log.txt}.
Fatal error messages still appear on the command line.

\subsubsection{--s}
Initiates the packet mode, where a file with the suffix {\tt PREFIX\_packets.txt} is created. The content of the file depends on the plugins loaded.
The display of the packet number (first column is controlled by {\tt SPKTMD\_PKTNO} in \nameref{main.h}. The layer 7 payload can be displayed in hexadecimal and/or as characters by using the {\tt SPKTMD\_PCNTH} and {\tt SPKTMD\_PCNTC} respectively.
A tab separated header description line is printed at the beginning of the packet file. The first two lines then read as follows:

\begin{center}
\begin{scriptsize}
\begin{lstlisting}
%pktNo    time    pktIAT    duration    flowInd    flowStat    numHdrDesc   hdrDesc   ethVlanID    ethType    srcMac    dstMac    srcIP4    srcPort    dstIP4    dstPort    l4Proto    ipTOS    ipID    ipIDDiff    ipFrag    ipTTL    ipHdrChkSum    ipCalChkSum    l4HdrChkSum    l4CalChkSum    ipFlags    pktLen    ipOptLen    ipOpts    seq    ack    seqDiff    ackDiff    seqPktLen    ackPktLen    tcpStat    tcpFlags    tcpAnomaly    tcpWin    tcpOptLen    tcpOpts    l7Content
...
25    1291753225.446846    0.000000    0.000000    23    0x00006000   6    eth:vlan:mpls{2}:ipv4:tcp   20    0x0800    00:90:1a:41:fa:45    00:13:c4:52:4a:07    188.62.56.56    62701    212.243.221.241    80    6    0x00    0x26f6    0    0x4000    62    0x6ca6    0x6ca6    0x0247    0x0247    0x0040    460    0        0xb2a08909    0x90314073    0    0    0    0    0x59    0x18    0x0000    65535    12    0x01 0x01 0x08 0x0a 0x29 0x2d 0xc3 0x6e 0x83 0x63 0xc5 0x76    GET /images/I/01TnJ0+mhnL.png HTTP/1.1\r\nHost: ecx.images-amazon.com\r\nUser-Agent: Mozilla/5.0 (Macintosh; U; Intel Mac OS X 10.6; de; rv:1.9.2.8) Gecko/20100722 Firefox/3.6.8\r\nAccept: image/png,image/*;q=0.8,*/*;q=0.5\r\nAccept-Language: de-de,de;q=0.8,en-us;q=0.5,en;q=0.3\r\nAccept-Encoding: gzip,deflate\r\nAccept-Charset: ISO-8859-1,utf-8;q=0.7,*;q=0.7\r\nKeep-Alive: 115\r\nConnection: keep-alive\r\nReferer: http://z-ecx.images-amazon.com/images/I/11J5cf408UL.css\r\n\r\n
...
\end{lstlisting}
\end{scriptsize}
\end{center}

\subsubsection{--p FOLDER}
Changes the plugin folder from standard {\em \textasciitilde{}/.tranalyzer/plugins} to {\tt FOLDER}.

\subsubsection{--b FILE}
Changes the plugin blacklist file from {\em plugin\_folder/plugin\_blacklist.txt} to {\tt FILE}, where {\tt plugin\_folder} is either\\
{\em\textasciitilde{}/.tranalyzer/plugins} or the folder specified with the {\tt --p} option.

\subsubsection{--e FLOWINDEXFILE}
Denotes the filename and path of the flow index file when the \tranrefpl{pcapd} plugin is loaded.
The path and name of the pcap file depends on {\tt FLOWINDEXFILE}.
If omitted the default names for the PCAP file are defined in {\em pcapd.h}. The format of the {\tt FLOWINDEXFILE} is a list of 64 bit flow indices which define the packets to be extracted from the pcap being read by the {\tt --r} option.
In general the user should use a plain file with the format displayed below:

\begin{scriptsize}
\begin{lstlisting}
# Comments (ignored)
% Flow file info (ignored)
30
3467
656697
5596
\end{lstlisting}
\end{scriptsize}

For more information on the \tranrefpl{pcapd} plugin please refer to its \href{../../pcapd/doc/pcapd.pdf}{documentation}.

\subsubsection{--f HASHFACTOR}\label{s:foption}
Sets and superseeds the {\tt HASHFACTOR} constant in \nameref{tranalyzer.h}.

\subsubsection{--x SENSORID}
Each T2 can have a separate sensor ID which can be listed in a flow file in order to differentiate flows
originating from several interfaces during post processing, e.g., in a DB. If not specified {\tt T2\_SENSORID} ({\tt 666}), defined in \nameref{tranalyzer.h}, will be the default value.

\subsubsection{--c CPU}
Bind Tranalyzer to core number {\tt CPU}; if {\tt CPU == 0} then the operating system selects the core to bind.

\subsubsection{--F FILE}
Read BPF filter from FILE. A filter can span multiple lines and can be commented using the {\tt `\#'} character (everything following a {\tt `\#'} is ignored).

\subsubsection{BPF Filter}
A Berkeley Packet Filter (BPF) can be specified at any time in order to reduce the amount of flows being produced and to increase speed during life capture ops. All rules of pcap BPF apply.

\subsection{ioBuffer.h}\label{ioBuffer.h}
\begin{longtable}{lcl}
    \toprule
    {\bf Name} & {\bf Default} & {\bf Description}\\
    \midrule\endhead%
    {\tt ENABLE\_IO\_BUFFERING} & 0 & Enables buffering of the packets in a queue\\\\

    \multicolumn{3}{l}{If {\tt ENABLE\_IO\_BUFFERING == 1}, the following flags are available:}\\\\

    {\tt IO\_BUFFER\_FULL\_WAIT\_MS} & 200 & Number of milliseconds to wait if queue is full\\
    {\tt IO\_BUFFER\_SIZE} & 8192 & Maximum number of packets that can be stored in the buffer (power of 2)\\
    {\tt IO\_BUFFER\_MAX\_MTU} & 2048 & Maximum size of a packet (divisible by 4)\\
    \bottomrule
\end{longtable}

\subsection{main.h}\label{main.h}
The monitoring mode can be configured with the following constants:\\
\begin{center}
    \begin{tabularx}{\textwidth}{lcX}
        \toprule
        {\bf Name} & {\bf Default} & {\bf Description}\\
        \midrule
        {\tt SPKTMD\_PKTNO} & 1 & Print the packet number\\
        {\tt SPKTMD\_PCNTC} & 1 & Print L7 content as characters\\
        {\tt SPKTMD\_PCNTH} & 0 & Print L7 content as hex\\\\

        {\tt MIN\_MAX\_ESTIMATE} & 0 & Min/Max bandwidth statistics\\\\

        \multicolumn{3}{l}{The following flags control the monitoring mode:}\\\\

        {\tt MONINTTHRD}      & 1   & Activate threaded interrupt handling \\
        {\tt MONINTBLK}       & 0   & Block interrupts in main loop during packet processing, disables {\tt MONINTTHRD} \\
        {\tt MONINTSYNC}      & 1   & Synchronized print statistics \\
        {\tt MONINTTMPCP}     & 0   & 0: real time base, 1: pcap time base \\
        {\tt MONINTTMPCP\_ON} & 0   & Startup monitoring. 1: on 0: off (require {\tt MONINTTMPCP=1})\\
        {\tt MONINTV}         & 1   & $\geq 1\sec$ interval of monitoring output if {\tt USR2} is sent or {\tt MONINTTMPCP=0}\\
        {\tt MONPROTMD}       & 1   & 0: report protocol numbers; 1: report protocol names \\
        {\tt MONPROTFL} & {\small\tt "proto.txt"} & proto file \\
        \bottomrule
    \end{tabularx}
\end{center}

The {\tt MONPROTL2} and {\tt MONPROTL3} flags can be used to configure the L2 and L3 protocols to monitor.
Their default values are
\begin{itemize}
    \item {\tt MONPROTL2}: {\tt 0x0042,0x00fe,ETHERTYPE\_ARP,ETHERTYPE\_RARP,ETHERTYPE\_IP,ETHERTYPE\_IPV6}
    \item {\tt MONPROTL3}: {\tt L3\_ICMP,L3\_IGMP,L3\_TCP,L3\_UDP,L3\_GRE,L3\_ICMP6,L3\_SCTP}
\end{itemize}

\subsection{networkHeaders.h}\label{networkHeaders.h}
\begin{center}
    \begin{tabularx}{\textwidth}{lcXl}
        \toprule
        {\bf Name} & {\bf Default} & {\bf Description} & {\bf Flags}\\
        \midrule
        {\tt IPV6\_ACTIVATE}   & 2 & 0: IPv4 only\\
                               &   & 1: IPv6 only\\
                               &   & 2: Dual mode\\
        {\tt ETH\_ACTIVATE}    & 1 & 0: No Ethernet flows\\
                               &   & 1: Activate Ethernet flows\\
                               &   & 2: Also use Ethernet addresses for IPv4/6 flows\\
        {\tt SCTP\_ACTIVATE}   & 0 & SCTP protocol decoder for stream $\rightarrow$ flow generation is activated \\
        {\tt SCTP\_STATFINDEX} & 0 & 1: findex constant for all SCTP streams in a packet & {\tt SCTP\_ACTIVATE=1}\\
                               &   & 0: findex increments \\
        {\tt MULTIPKTSUP}      & 0 & Multi-packet suppression (discard duplicated packets) & {\tt IPV6\_ACTIVATE=0}\\
        {\tt T2\_PRI\_HDRDESC}  &   1 & 1: keep track of the headers traversed\\
        {\tt T2\_HDRDESC\_AGGR} &   1 & 1: aggregate repetitive headers, e.g., {\tt vlan\{2\}} & {\tt T2\_PRI\_HDRDESC=1}\\
        {\tt T2\_HDRDESC\_LEN}  & 128 & max length of the headers description & {\tt T2\_PRI\_HDRDESC=1}\\
        \bottomrule
    \end{tabularx}
\end{center}

\clearpage

\subsection{packetCapture.h}\label{packetCapture.h}
The config file {\em packetCapture.h} provides control about the packet capture and packet structure process of Tranalyzer2.
The most important fields are described below. Please note that after changing any value in define statements a rebuild is required.
Note that the PACKETLENGTH switch controles the packetLength variable in the packet structure, from where the packet length is
measured from. So statistical plugins such as basicStats can have a layer dependent output. If only L7 length is needed,
use the packetL7length variable in the packet structure.

\begin{longtable}{lcll}
    \toprule
    {\bf Name} & {\bf Default} & {\bf Description} & {\bf Flags}\\
    \midrule\endhead%
    {\tt PACKETLENGTH}      & 3 & 0: including L2, L3 and L4 header\\
                            &   & 1: including L3 and L4 header\\
                            &   & 2: including L4 header\\
                            &   & 3: only higher layer payload (Layer 7)\\
    {\tt FRGIPPKTLENVIEW}   & 1 & 0: IP header stays with 2nd++ fragmented packets & {\tt PACKETLENGTH=1}\\
                            &   & 1: IP header stripped from 2nd++ fragmented packets\\
    {\tt NOLAYER2}          & 0 & 0: Automatic L3 header discovery\\
                            &   & 1: Manual L3 header positioning\\
    {\tt NOL2\_L3HDROFFSET} & 0 & Offset of L3 header & {\tt NOLAYER2=1}\\
    {\tt MAXHDRCNT}         & 5 & Maximal header count (MUST be $\geq 3$)& {\tt IPV6\_ACTIVATE=1}\\
    \bottomrule
\end{longtable}

\subsection{tranalyzer.h}\label{tranalyzer.h}

\begin{center}
    \begin{tabularx}{\textwidth}{lcXl}
        \toprule
        {\bf Name} & {\bf Default} & {\bf Description} & {\bf Flags}\\
        \midrule
        {\tt REPSUP}            & 0   & Activate alive mode\\
        {\tt PID\_FNM\_ACT}     & 0   & Save the PID into a file {\tt PID\_FNM}\\
                                &     & (default: {\tt "tranalyzer.pid"})\\
        {\tt DEBUG}             & 0   & 0: no debug output\\
                                &     & 1: debug output which occurs only once or very seldom\\
                                &     & 2: + debug output which occurs in special situations, but not regularly\\
                                &     & 3: + debug output which occurs regularly (every packet)\\
        {\tt VERBOSE}           & 2   & 0: no output\\
                                &     & 1: basic pcap report\\
                                &     & 2: + full traffic statistics\\
                                &     & 3: + info about fragmentation anomalies\\
        {\tt MEMORY\_DEBUG}     & 0   & 0: no memory debug\\
                                &     & 1: detect leaks and overflows (see {\em utils/memdebug.h})\\
        {\tt NO\_PKTS\_DELAY\_US} & 1000 & If no packets are available, sleep for $n$ microseconds\\
        {\tt NON\_BLOCKING\_MODE} & 1 & Non-blocking mode\\
        {\tt MAIN\_OUTPUT\_BUFFER\_SIZE} & 1000000 & Size of the main output buffer\\
        {\tt SNAPLEN}           & {\small\tt BUFSIZ} & Snapshot length (live capture)\\
        {\tt CAPTURE\_TIMEOUT}  & 1000 & Read timeout in milliseconds (live capture)\\
        {\tt ENABLE\_BPF\_OPTIMIZATION} & 0   & Optimize BPF filters\\
        {\tt TSTAMP\_PREC}      & 0   & Timestamp precision: 0: microseconds, 1: nanoseconds\\
        {\tt TSTAMP\_UTC}       & 1   & Time representation: 0: localtime, 1: UTC\\
        {\tt TSTAMP\_R\_UTC}    & 0   & Time report representation: 0: localtime, 1: UTC\\
        {\tt ALARM\_MODE}       & 0   & Only output flow if an alarm-based plugin fires\\
        {\tt ALARM\_AND}        & 0   & 0: logical OR, 1: logical AND & {\tt ALARM\_MODE=1}\\
        {\tt FORCE\_MODE}       & 0   & Parameter induced flow termination (NetFlow mode)\\
        {\tt BLOCK\_BUF}        & 0   & Block unnecessary buffer output when non Tranalyzer format event based plugins are active\\
        \bottomrule
    \end{tabularx}
\end{center}
\begin{center}
    \begin{tabularx}{\textwidth}{lcXl}
        \toprule
        {\bf Name} & {\bf Default} & {\bf Description} & {\bf Flags}\\
        \midrule
        {\tt PLUGIN\_REPORT}    & 1   & Enable plugins to contribute to Tranalyzer end report\\
        {\tt DIFF\_REPORT}      & 0   & 0: absolute Tranalyzer command line USR1 report\\
                                &     & 1: differential report\\
        {\tt MACHINE\_REPORT}   & 0   & USR1 report: 0: human compliant, 1: machine compliant\\
        {\tt REPORT\_HIST}      & 0   & Store statistical report history in {\tt REPORT\_HIST\_FILE} after
                                        shutdown and reload it when restarted\\
        {\tt ESOM\_DEP}         & 0   & Allow plugins to globally access plugin dependent variables\\
        {\tt AYIYA}             & 1   & Activate AYIYA processing\\
        {\tt GENEVE}            & 1   & Activate GENEVE processing\\
        {\tt TEREDO}            & 1   & Activate TEREDO processing\\
        {\tt L2TP}              & 1   & Activate L2TP processing\\
        {\tt GRE}               & 1   & Activate GRE processing\\
        {\tt GTP}               & 1   & Activate GTP (GPRS Tunneling Protocol) processing\\
        {\tt VXLAN}             & 1   & Activate VXLAN (Virtual eXtensible Local Area Network) processing\\
        {\tt IPIP}              & 1   & Activate IPv4/6 in IPv4/6 processing\\
        {\tt ETHIP}             & 1   & Activate Ethernet within IP IPv4/6 processing\\
        {\tt CAPWAP}            & 1   & Activate CAPWAP processing\\
        {\tt CAPWAP\_SWAP\_FC}  & 1   & Swap frame control (required for Cisco)\\
        {\tt LWAPP}             & 1   & Activate LWAPP processing\\
        {\tt LWAPP\_SWAP\_FC}   & 1   & Activate LWAPP processing (required for Cisco)\\
        {\tt FRAGMENTATION}     & 1   & Activate fragmentation processing\\
        {\tt FRAG\_HLST\_CRFT}  & 1   & Enables crafted packet processing & {\tt FRAGMENTATION=1}\\
        {\tt FRAG\_ERROR\_DUMP} & 0   & Dumps flawed fragmented packet to {\tt stdout} & {\tt FRAGMENTATION=1}\\
        {\tt IPVX\_INTERPRET}   & 0   & Interpret bogus IPvX packets\\
        {\tt ETH\_STAT\_MODE}   & 0   & Whether to use the innermost (0) or outermost (1) layer 2 type for the statistics\\
        {\tt RELTIME}           & 0   & 0: absolute time\\
                                &     & 1: relative time\\
        {\tt FDURLIMIT}         & 0   & If $>0$, force flow life span to $n\pm1$ seconds\\
        {\tt FLOW\_TIMEOUT}     & 182 & Flow timeout after a packet is not seen after $n$ seconds\\

        {\tt \nameref{hash_autopilot}} & 1 & 0: terminate when main hash map is full\\
                                       &   & 1: flushes oldest {\tt NUMFLWRM} flow(s) when main hash is full\\
        {\tt NUMFLWRM}                 & 1 & Number of flows to flush when main hash map is full & {\tt HASH\_AUTOPILOT=1}\\
        \bottomrule
    \end{tabularx}
\end{center}

\subsubsection{-D constants}
the following constants influence the file name convention:

\begin{longtable}{lcl}
    \toprule
    {\bf Name} & {\bf Default} & {\bf Description}\\
    \midrule\endhead%
    {\tt RROP} & 0 & round robin operations \\
    {\tt POLLTM} & 5 & poll timing for files \\
    {\tt SCHR} & {\tt `p'} & separating character for file number \\
    \bottomrule
\end{longtable}

\subsubsection{alive signal}
The alive signal is a derivate of the passive monitoring mode by the USR1 signal, where the
report is deactivated. If {\tt REPSUP=1} then only the command defined by {\tt REPCMDAS/W} is sent to the
control program defined by {\tt ALVPROG} as defined below:

\begin{longtable}{lcl}
    \toprule
    {\bf Name} & {\bf Default} & {\bf Description}\\
    \midrule\endhead%
    {\tt REPSUP}   & 0 & 0: alive mode off,\\
                   &   & 1: alive mode on, monitoring report suppressed \\
    {\tt ALVPROG}  & {\tt "t2alive"} & name of control program \\
    {\tt REPCMDAS} & {\tt "a=`pgrep ALVPROG`; \textbackslash} & alive and stall USR1 signal (no packets) \\
                   & {\tt if [ \$a ]; then kill -USR1 \$a; fi"} \\
    {\tt REPCMDAW} & {\tt "a=`pgrep ALVPROG`; \textbackslash} & alive and well USR2 signal (working) \\
                   & {\tt if [ \$a ]; then kill -USR2 \$a; fi"} \\
    \bottomrule
\end{longtable}

If T2 crashes or is stopped a syslog message is issued by the t2alive deamon. Same if T2 gets started.

\subsubsection{FORCE\_MODE}
A 1 enables the force mode which enables any plugin to force the output of flows independent of the timeout value. Hence, Cisco NetFlow
similar periodic output can be produced or overflows of counters can produce a flow and restart a new one.

\subsubsection{ALARM\_MODE}
A 1 enables the alarm mode which differs from the default flow mode by the plugin based control of the Tranalyzer core flow output. It is useful for
classification plugins generating alarms, thus emulating alarm based SW such as Snort, etc. The default value is 0.
The plugin sets the global output suppress variable {\tt supOut=1} in the {\em onFlowTerminate()} function before any output is generated. This mode also allows multiple classification plugins producing an `AND' or an `OR' operation if many alarm generating plugins are loaded. The variable {\tt ALARM\_AND} controls the logical alarm operation. A sample code which has to be present at the beginning of the {\em onFlowTerminate()} function is shown below:

\begin{figure}[ht]
\centering
%\begin{tabular}{c}
\begin{lstlisting}
#if ALARM_MODE == 1
#if ALARM_AND == 0
        if (!Alarm) supOut = 0;
#else // ALARM_AND == 1
        if (!Alarm) {
                supOut = 1;
                return;
        }
#endif // ALARM_AND
#endif // ALARM_MODE == 1
\end{lstlisting}
%\end{tabular}
\caption{A sample code in the {\em onFlowTerminate()} routine}
\end{figure}


\subsubsection{BLOCK\_BUF}
if set to `1' unnecessary buffered output from all plugins is blocked when non Tranalyzer format event based plugins are active: e.g. syslog, arcsight and
text-based or binary output plugins are not loaded.

\subsubsection{Report Modes}
Tranalyzer provides a user interrupt based report and a final report. The interrupt based mode can be configured in a variety of ways being defined below.

\begin{longtable}{lcl}
    \toprule
    {\bf Name} & {\bf Default} & {\bf Description}\\
    \midrule\endhead%
    {\tt PLUGIN\_REPORT} & 0 & enable plugins to contribute to the tranalyzer command line end report \\
    {\tt DIFF\_REPORT} & 0 & 1: differential, 0: Absolute tranalyzer command line {\tt USR1} report \\
    {\tt MACHINE\_REPORT} & 0 & {\tt USR1} Report 1: machine compliant; 0: human compliant \\
    \bottomrule
\end{longtable}

The following interrupts are being caught by Tranalyzer2:

\begin{longtable}{ll}
    \toprule
    {\bf Signal Name} & {\bf Description}\\
    \midrule\endhead%
    {\tt SIGINT}  & like {\tt\textasciicircum{}C} terminates new flow production%
                    \footnote{If two {\tt SIGINT} interrupts are being sent in short order Tranalyzer will be terminated instantly.}\\
    {\tt SIGTERM} & terminates tranalyzer \\
    {\tt SIGUSR1} & prints statistics report \\
    {\tt SIGUSR2} & toggles repetitive statistics report \\
    \bottomrule
\end{longtable}

\subsubsection{State and statistical save mode}
T2 is capable to preserve its internal statistical state and certain viable global variables, such as the findex.

\begin{longtable}{lcl}
    \toprule
    {\bf Name} & {\bf Default} & {\bf Description}\\
    \midrule\endhead%
    {\tt\small REPORT\_HIST} & 0 & Store statistical report history after shutdown, reload it upon restart\\
    {\tt\small REPORT\_HIST\_FILE} & {\tt\small "stat\_hist.txt"} & default statistical report history filename\\
    \bottomrule
\end{longtable}
The history file is stored by default under {\tt ./tranalyzer/plugins} or under the directory defined by a {\tt --p} option.

\subsubsection{L2TP}
A `1' activates the L2TP processing of the Tranalyzer2 core. All L2TP headers either encapsulated in MPLS or not will be processed and followed down via PPP headers to the IP header and then passed to the IP processing. The default value of the variable is `0'. Then the stack will be parsed until the first IP header is detected. So all L2TP UDP headers having src and dest port 1701 will be processed as normal UDP packets.

\subsubsection{GRE}
A `1' activates the L3 General Routing Encapsulation (L4proto=47) processing of the Tranalyzer2 core. All GRE headers either encapsulated in MPLS or not will be processed and followed down via PPP headers to the IP header and then passed to the IP processing. The default value of the variable is 0. Then the stack will be parsed until the first IP header is detected. If the following content is not existing or compressed the flow will contain only L4proto = 47 information.

\subsubsection{FRAGMENTATION}
A `1' activates the fragmentation processing of the Tranalyzer2 core. All packets following the header packet will be assembled in the same flow. The core and the plugin tcpFlags will provide special flags for fragmentation anomalies. If {\tt FRAGMENTATION} is set to 0 only the initial fragment will be processed; all later fragments will be ignored.

\subsubsection{FRAG\_HLST\_CRFT}
A `1' enables crafted packet processing even when the lead fragment is missing or packets contain senseless flags as being used in attacks or equipment failure.

\subsubsection{FRAG\_ERROR\_DUMP}
A `1' activates the dump of packet information on the command line for time based identification of ill-fated or crafted fragments in tcpdump or wireshark. It provides the Unix timestamp, the six tuple, IPID and fragID as outlined in figure below.

\begin{figure}[!ht]
\begin{lstlisting}
MsgType   msg   time   vlan   srcIP   srcPort   dstIP   dstPort   proto   fragID   fragOffset
[WRN] packetCapture: 1. frag not found @ 1291753225.449690 20 92.104.181.154 42968 93.144.66.3 52027 17 - 0x191F 0x00A0
[WRN] packetCapture: 1. frag not found @ 1291753225.482611 20 92.104.181.154 43044 93.144.66.3 1719 17 - 0x1922 0x00A0
[WRN] packetCapture: 1. frag not found @ 1291753225.492830 20 92.104.181.154 55841 93.144.66.3 28463 17 - 0x1923 0x00A0
[WRN] packetCapture: 1. frag not found @ 1291753225.503955 20 92.104.181.154 25668 93.144.66.3 8137 17 - 0x1924 0x00A0
[WRN] packetCapture: 1. frag not found @ 1291753225.551094 20 92.105.93.227 41494 24.218.128.232 27796 17 - 0x5A21 0x00A0
[WRN] packetCapture: 1. frag not found @ 1291753225.639627 20 86.51.18.243 38824 92.105.108.208 55133 17 - 0x0DAE 0x00AC
\end{lstlisting}
\caption{A sample report on stdout for packets with an elusive first fragment}
\end{figure}

{\bf WARNING:} If {\tt FRAG\_HLST\_CRFT == 1} then every fragmented headerless packet will be reported!

\subsubsection{*\_SUFFIX}\label{file_suffixes}
This constant defines the suffix of all plugin output files. For example if you specify the output {\em foo.foo} (with the {\tt --w} option), the generated file for the per-packet output will be in the default setting {\em foo.foo\_packets}.

\subsubsection{RELTIME}
{\tt RELTIME} renders all time based plugin output into relative to the beginning of the pcap or start of packet capture. In {\tt --D} or {\tt --R} read operation the first file defines the start time.

\subsubsection{FLOW\_TIMEOUT}
This constant specifies the default time in seconds (182) after which a flow will be considered as terminated since the last packet is captured. Note: Plugins are able to change the timeout values of a flow. For example the \tranrefpl{tcpStates} plugin adjusts the timeout of a flow according to the TCP state machine. A reduction of the flow timeout has an effect on the necessary flow memory defined in {\tt HASHCHAINTABLE\_SIZE}, see below.

\subsubsection{FDURLIMIT}
{\tt FDURLIMIT} defines the maximum flow duration in seconds which is then forced to be released.
It is a special force mode for the duration of flows and a special feature for Dalhousie University.
If {\tt FDURLIMIT > 0} then {\tt FLOW\_TIMEOUT} is overwritten if {\tt FURLIMIT} seconds are reached.

\subsubsection{HASHFACTOR}
A factor to be multiplied with the {\tt HASHTABLE\_SIZE} and {\tt HASHCHAINTABLE\_SIZE} described below.
It facilitates the correct setting of the hash space.
Moreover, if T2 runs out of hash it will give an upper estimate the user can choose for {\tt HASHFACTOR}.
Set it to this value, recompile and rerun T2.
This constant is superseeded by the \hyperref[s:foption]{{\tt --f} option}.

\subsubsection{HASHTABLE\_SIZE}
The number of buckets in the hash table.
As a separate chaining hashing method is used, this value does not denote the amount of elements the hash table is able to manage!
The larger, the less likely are hash collisions.
The current default value is $2^{18}$.
Its value should be selected at least two times larger as the value of {\tt HASHCHAINTABLE\_SIZE} discussed in the following chapter.

\subsubsection{HASHCHAINTABLE\_SIZE}
Specifies the amount of flows the main hash table is able to manage.
The default value is $2^{19}$, so roughly half the size of {\tt HASHTABLE\_SIZE}.
T2 supplies information about the hash space in memory in: {\tt Max number of IPv4 flows in memory: 113244 (50.220\%)}.
Together with the amount of traffic already processed the total value can be computed.
An example is given in \reff{fig:t2finalreport}.

\subsubsection{HASH\_AUTOPILOT}\label{hash_autopilot}
Default 1. Avoids overrun of main hash, flushes oldest flow on every flowInsert if hashmap is full. 0 disables
hash overrun protection. If speed is an issue avoid overruns by invoking T2 with a laged -f option value, as T2 recommends.

\subsubsection{Aggregation Mode}
The aggregation mode enables the user to confine certain IP, port or protocol ranges into a single flow. The variable {\tt AGGREGATIONFLAG} in \nameref{tranalyzer.h} defines a
bit field which enables specific aggregation modes according to the six tuple values listed below.
\begin{longtable}{ll}
    \toprule
    {\bf Aggregation Flag} & {\bf Value} \\
    \midrule\endhead%
    {\tt L4PROT}  & {\tt 0x01} \\
    {\tt DSTPORT} & {\tt 0x02} \\
    {\tt SRCPORT} & {\tt 0x04} \\
    {\tt DSTIP}   & {\tt 0x08} \\
    {\tt SRCIP}   & {\tt 0x10} \\
    {\tt VLANID}  & {\tt 0x20} \\
    {\tt SUBNET}  & {\tt 0x80} \\
    \bottomrule
\end{longtable}
If a certain aggregation mode is enabled the following variables in \nameref{tranalyzer.h}
define the aggregation range.
\begin{longtable}{lcl}
    \toprule
    {\bf Aggregation Flag} & {\bf Type} & {\bf Description}\\
    \midrule\endhead%
    {\tt SRCIP4CMSK} & {\tt uint8\_t}  & src IPv4 aggregation CIDR mask \\
    {\tt DSTIP4CMSK} & {\tt uint8\_t}  & dst IPv4 aggregation CIDR mask \\
    {\tt SRCIP6CMSK} & {\tt uint8\_t}  & src IPv6 aggregation CIDR mask \\
    {\tt DSTIP6CMSK} & {\tt uint8\_t}  & dst IPv6 aggregation CIDR mask \\
    {\tt SRCPORTLW}  & {\tt uint16\_t} & src port lower bound \\
    {\tt SRCPORTHW}  & {\tt uint16\_t} & src port upper bound \\
    {\tt DSTPORTLW}  & {\tt uint16\_t} & dst port lower bound \\
    {\tt DSTPORTHW}  & {\tt uint16\_t} & dst port upper bound \\
    \bottomrule
\end{longtable}

\subsection{bin2txt.h}\label{bin2txt.h}
\begin{longtable}{lcl}%l}
    \toprule
    {\bf Name} & {\bf Default} & {\bf Description}\\% & {\bf Flags}\\
    \midrule\endhead%
    {\tt HEX\_CAPITAL}               & 0           & hex output: 0: lower case; 1: upper case\\
    {\tt IP4\_FORMAT}                & 0           & IPv4 addresses representation:\\
                                     &             & \qquad 0: normal,\\
                                     &             & \qquad 1: normalized (padded with zeros),\\
                                     &             & \qquad 2: one 32-bits hex number\\
                                     &             & \qquad 3: one 32-bits unsigned number\\
    {\tt IP6\_FORMAT}                & 0           & IPv6 addresses representation:\\
                                     &             & \qquad 0: compressed,\\
                                     &             & \qquad 1: uncompressed,\\
                                     &             & \qquad 2: one 128-bits hex number,\\
                                     &             & \qquad 3: two 64-bits hex numbers\\
    {\tt MAC\_FORMAT}                & 0           & MAC addresses representation:\\
                                     &             & \qquad 0: normal (edit {\tt MAC\_SEP} to change the separator),\\
                                     &             & \qquad 1: one 64-bits hex number,\\
    {\tt MAC\_SEP}                   & {\tt ":"}   & Separator to use in MAC addresses: {\tt 11:22:33:44:55:66}\\
    {\tt TFS\_EXTENDED\_HEADER}      & 0 & Extended header in flow file\\
    {\tt B2T\_TIME\_IN\_MICRO\_SECS} & 1 & Time precision: 0: nanosecs, 1: microsecs\\
    {\tt TFS\_NC\_TYPE}              & 1 & Types in header file: 0: numbers, 1: C types\\
    {\tt TFS\_SAN\_UTF8}             & 1 & Activates the UTF-8 sanitizer for strings\\
    {\tt B2T\_TIMESTR}               & 0 & Print unix timestamps as human readable dates\\

    {\tt HDR\_CHR}                   & {\tt "\%"}                & start character(s) of comments\\
    {\tt SEP\_CHR}                   & {\tt "\textbackslash{}t"} & column separator in the flow file\\
                                     &                           & {\tt ";"}, {\tt "."}, {\tt "\_"} and {\tt "\textbackslash""} should not be used\\
    {\tt JSON\_KEEP\_EMPTY} & 0 & Whether or not to output empty fields\\
    {\tt JSON\_PRETTY}      & 0 & Whether to add spaces to make the output more readable\\
    \bottomrule
\end{longtable}

\subsection{outputBuffer.h}\label{outputBuffer.h}
\begin{longtable}{lcll}
    \toprule
    {\bf Name} & {\bf Default} & {\bf Description} & {\bf Flags}\\
    \midrule\endhead%
    {\tt BUF\_DATA\_SHFT}    & 0 & Adds for each binary output record the length and shifts\\
                             &   & the record by $n$ {\tt uint32\_t} words to the right\\
                             &   & (see \tranrefpl{binSink} and \tranrefpl{socketSink} plugin)\\
    {\tt OUTBUF\_AUTOPILOT}  & 1 & Automatically increase the output buffer when required\\
    {\tt OUTBUF\_MAXSIZE\_F} & 5 & Maximal factor to increase the output buffer size to & {\tt\small OUTBUF\_AUTOPILOT=1}\\
    \bottomrule
\end{longtable}

\subsection{Tranalyzer2 Output}\label{tranalyzer_output}
As stated before, the functionality and output of Tranalyzer2 is defined by the activated plugins.
Basically, there are two ways a plugin can generate output.
First, it can generate its own output file and write any arbitrary content into any stream.
The second way is called standard output or per-flow output. After flow termination Tranalyzer2 provides an output buffer and appends the direction of the flow to it. For example, in case of textual output, an {\tt ``A''} flow is normally followed by a {\tt ``B''} flow or if the {\tt ``B''} flow does not exist it is followed by the next {\tt ``A''} flow. Then, the output buffer is passed to the plugins providing their per-flow output. Finally the buffer is sent to the activated output plugins. This process repeats itself for the {\tt ``B''} flow. For detailed explanation about the functionality of the output plugins refer to the section plugins.

\subsubsection{Hierarchical Ordering of Numerical or Text Output}
Tranalyzer2 provides a hierarchical ordering of each output. Each plugin controls the:
\begin{itemize}
\item volume of its output
\item number of values or bins
\item hierarchical ordering of the data
\item repetition of data substructures
\end{itemize}
Thus, complex structures such as lists or matrices can be presented in a single line.\\
The following sample of text output shows the hierarchical ordering for four data outputs, separated by tabulators:
\begin{figure*}[!ht]
\centering
\begin{lstlisting}
A    0.3   2.0_3.4_2.1    2;4;2;1    (1_2_9)_(1_3_1)_(7_5_3)_(2_3_7)
\end{lstlisting}
\end{figure*}

The {\tt A} indicates the direction of the flow; in this case it is the initial flow. The next number denotes a singular descriptive statistical result. Output number two consists of three values separated by ``\_'' characters. Output number three consists of one value, that can be repeated, indicated by the character ``;''. Output number four is a more complex example: It consists of four values containing three subvalues indicated by the braces. This could be interpreted as a matrix of size $4\times3$.

\subsection{Final Report}
Standard configuration of Tranalyzer2 produces a statistical report to {\em stdout} about timing, packets, protocol encapsulation type, average bandwidth, dump length, etc. A sample report including some current protocol relevant warnings is depicted in the figure below. Warnings are not fatal hence are listed at the end of the statistical report when Tranalyzer2 terminates naturally. The {\em Average total Bandwidth} estimation refers to the processed bandwidth during the data acquisition process. It is only equivalent to the actual bandwidth if the total packet length including all encapsulations is not truncated and all traffic is IP. The {\em Average IP Traffic Bandwidth} is an estimate comprising all IP traffic actually present on the wire. Plugins can report extra information when {\tt PLUGIN\_REPORT} is activated. This report can be saved in a file, by using one of the following command:

\begin{center}
    {\tt tranalyzer -r file.pcap -w out -l (See \refs{t2-loption})}\\
    {\tt tranalyzer -r file.pcap -w out | tee out\_stdout.txt}\\
    {\tt tranalyzer -r file.pcap -w out > out\_stdout.txt}
\end{center}

Both commands will create a file {\tt out\_stdout.txt} containing the report. The only difference between those two commands is that the first one still outputs the report to stdout.

Fatal errors regarding the invocation, configuration and operation of Tranalyzer2 are printed to {\em stderr} after the plugins are loaded, thus before the processing is activated, see the {\em Hash table error} example in \refl{fig:t2finalreport}. These errors terminate Tranalyzer2 immediately and are located before the final statistical report as being indicated by the {\em ``Shutting down\ldots''} key phrase. If the final report is to be used in a following script a pipe can be appended and certain lines can be filtered using grep or awk.

\begin{lstlisting}[
    caption = {A sample Tranalyzer2 final report including encapsulation warning, Hash Autopilot engagement when hash table full},
    label   = fig:t2finalreport,
]

$ ./tranalyzer -r ~/data/knoedel.pcap -w ~/results/
================================================================================
Tranalyzer 0.8.2 (Anteater), Tarantula. PID: 16298
================================================================================
[INF] Creating flows for L2, IPv4, IPv6
Active plugins:
    01: protoStats, 0.8.2
    02: basicFlow, 0.8.2
    03: macRecorder, 0.8.2
    04: portClassifier, 0.8.2
    05: basicStats, 0.8.3
    06: tcpFlags, 0.8.2
    07: tcpStates, 0.8.2
    08: icmpDecode, 0.8.2
    09: dnsDecode, 0.8.3
    10: httpSniffer, 0.8.2
    11: connStat, 0.8.2
    12: txtSink, 0.8.2
[INF] basicFlow: IPv4 Ver: 3, Rev: 20190114, Range Mode: 0, subnet ranges loaded: 2821502 (2.82 M)
[INF] basicFlow: IPv6 Ver: 3, Rev: 20190114, Range Mode: 0, subnet ranges loaded: 36123 (36.12 K)
Processing file: /home/user/data/knoedel.pcap
Link layer type: Ethernet [EN10MB/1]
Dump start: 1291753225.446732 sec (Tue 07 Dec 2010 20:20:25 GMT)
[WRN] snapL2Length: 1550 - snapL3Length: 1484 - IP length in header: 1492
[WRN] Hash Autopilot: main HashMap full: flushing 1 oldest flow(s)
[INF] Hash Autopilot: Fix: Invoke Tranalyzer with '-f 5'
Dump stop : 1291753452.373884 sec (Tue 07 Dec 2010 20:24:12 GMT)
Total dump duration: 226.927152 sec (3m 46s)
Finished processing. Elapsed time: 284.541828 sec (4m 44s)
Finished unloading flow memory. Time: 295.292141 sec (4m 55s)
Percentage completed: 100.00%
Number of processed packets: 53982409 (53.98 M)
Number of processed bytes: 42085954664 (42.09 G)
Number of raw bytes: 42101578296 (42.10 G)
Number of pcap bytes: 42949673232 (42.95 G)
Number of IPv4 packets: 53768016 (53.77 M) [99.60%]
Number of IPv6 packets: 214108 (214.11 K) [0.40%]
Number of A packets: 31005784 (31.01 M) [57.44%]
Number of B packets: 22976625 (22.98 M) [42.56%]
Number of A bytes: 14212871985 (14.21 G) [33.77%]
Number of B bytes: 27873082679 (27.87 G) [66.23%]
Average A packet load: 458.39
Average B packet load: 1213.11 (1.21 K)
--------------------------------------------------------------------------------
basicStats: Biggest Talker: 92.122.216.218: 154922 (154.92 K) [0.29%] packets
basicStats: Biggest Talker: 92.122.216.218: 224224088 (224.22 M) [0.53%] bytes
tcpFlags: Aggregated IP anomaly flags : 0x3d6e
tcpFlags: Aggregated TCP anomaly flags: 0xfe07
tcpFlags: Number of TCP scans, succ scans, retries: 175001 (175.00 K), 29259 (29.26 K), 15782 (15.78 K)
tcpFlags: Number WinSz below 1: 162725 (162.72 K) [0.38%]
tcpStates: Aggregated anomaly flags: 0xdf
icmpDecode: Number of ICMP echo request packets: 14979 (14.98 K) [12.00%]
icmpDecode: Number of ICMP echo reply packets: 2690 (2.69 K) [2.16%]
icmpDecode: ICMP echo reply / request ratio: 0.18
icmpDecode: Number of ICMPv6 echo request packets: 1440 (1.44 K) [9.63%]
icmpDecode: Number of ICMPv6 echo reply packets: 951 [6.36%]
icmpDecode: ICMPv6 echo reply / request ratio: 0.66
dnsDecode: Number of DNS packets: 237597 (237.60 K) [0.44%]
dnsDecode: Number of DNS Q packets: 125260 (125.26 K) [52.72%]
dnsDecode: Number of DNS R packets: 112337 (112.34 K) [47.28%]
dnsDecode: Aggregated status: 0xe72f
httpSniffer: Number of HTTP packets: 36614826 (36.61 M) [67.83%]
httpSniffer: Number of HTTP GET  requests: 426825 (426.82 K) [1.17%]
httpSniffer: Number of HTTP POST requests: 39134 (39.13 K) [0.11%]
httpSniffer: HTTP GET/POST ratio: 10.91
httpSniffer: Aggregated status flags : 0x003f
httpSniffer: Aggregated anomaly flags: 0x5143
httpSniffer: Aggregated content flags: 0x007a
httpSniffer: Aggregated mime type    : 0x80ef
connStat: Max unique number of IP source connections: 275532 (275.53 K)
connStat: Max unique number of IP destination connections: 301252 (301.25 K)
connStat: Max unique number of IP source/destination connections: 1242 (1.24 K)
connStat: Max unique number of source IP / destination port connections: 1521 (1.52 K)
connStat: prtcon/sdcon, prtcon/scon: 1.224638, 0.005520
connStat: Source IP with max connections: X.Y.Z.U: 1515 (1.51 K) connections
connStat: Destination IP with max connections: V.W.A.B: 3241 (3.24 K) connections
--------------------------------------------------------------------------------
Headers count: min: 4, max: 14, average: 7.10
Max VLAN header count: 1
Max MPLS header count: 2
Number of LLC packets: 285 [0.00%]
Number of GRE packets: 27670 (27.67 K) [0.05%]
Number of Teredo packets: 213877 (213.88 K) [0.40%]
Number of AYIYA packets: 231 [0.00%]
Number of IGMP packets: 401 [0.00%]
Number of ICMP packets: 124800 (124.80 K) [0.23%]
Number of ICMPv6 packets: 14946 (14.95 K) [0.03%]
Number of TCP packets: 43273340 (43.27 M) [80.16%]
Number of UDP packets: 10311931 (10.31 M) [19.10%]
Number of IPv4 fragmented packets: 19155 (19.16 K) [0.04%]
Number of IPv6 fragmented packets: 9950 (9.95 K) [4.65%]
~~~~~~~~~~~~~~~~~~~~~~~~~~~~~~~~~~~~~~~~~~~~~~~~~~~~~~~~~~~~~~~~~~~~~~~~~~~~~~~~
Number of processed   flows: 1439153 (1.44 M)
Number of processed A flows: 1209728 (1.21 M) [84.06%]
Number of processed B flows: 229425 (229.43 K) [15.94%]
Number of request     flows: 930935 (930.93 K) [64.69%]
Number of reply       flows: 508218 (508.22 K) [35.31%]
Total   A/B    flow asymmetry: 0.68
Total req/rply flow asymmetry: 0.29
Number of processed   packets/flows: 37.51
Number of processed A packets/flows: 25.63
Number of processed B packets/flows: 100.15
Number of processed total packets/s: 237884.31 (237.88 K)
Number of processed A+B packets/s: 237884.31 (237.88 K)
Number of processed A   packets/s: 136633.20 (136.63 K)
Number of processed   B packets/s: 101251.10 (101.25 K)
~~~~~~~~~~~~~~~~~~~~~~~~~~~~~~~~~~~~~~~~~~~~~~~~~~~~~~~~~~~~~~~~~~~~~~~~~~~~~~~~
Number of average processed flows/s: 6341.92 (6.34 K)
Average full raw bandwidth: 1484232320 b/s (1.48 Gb/s)
Average snapped bandwidth : 1483681536 b/s (1.48 Gb/s)
Average full bandwidth : 1483899904 b/s (1.48 Gb/s)
Max number of flows in memory: 262144 (262.14 K) [100.00%]
Number of flows terminated by autopilot: 816044 (816.04 K) [56.70%]
Memory usage: 2.73 GB [4.05%]
Aggregate flow status: 0x01003cfad298fb04
[WRN] L3 SnapLength < Length in IP header
[WRN] L4 header snapped
[WRN] Consecutive duplicate IP ID
[WRN] IPv4/6 fragmentation header packet missing
[WRN] IPv4/6 packet fragmentation sequence not finished
[INF] IPv4
[INF] IPv6
[INF] IPv4/6 fragmentation
[INF] VLAN encapsulation
[INF] MPLS encapsulation
[INF] L2TP encapsulation
[INF] PPP/HDLC encapsulation
[INF] GRE encapsulation
[INF] AYIYA tunnel
[INF] Teredo tunnel
[INF] CAPWAP/LWAPP tunnel
[INF] SSDP/UPnP flows
[INF] Ethernet flows
[INF] SIP/RTP flows
[INF] Authentication Header (AH)
[INF] Encapsulating Security Payload (ESP)
[INF] TOR addresses
\end{lstlisting}
%stopzone % hack to correct vim syntax highlighting

T2 runs in IPv4 mode, but warns the user that there is IPv6 encapsulated.
%Packets are dissected until a layer 4 is found. or an IPv6 header is found and then backtrack.
Note that the new \hyperref[hash_autopilot]{Hash Autopilot} warns you when the main hash map is full.
T2 then removes the oldest Flow and continues processing your pcap.
To avoid that, run T2 again, but this time, use the {\tt --f 5} option as indicated in the warning message:
{\tt [INF] Hash Autopilot: Fix: Invoke Tranalyzer with '-f 5'} \\
{\tt t2 -r \textasciitilde/wurst/data/knoedel.pcap -w \textasciitilde/results -f 5}
or just let it run to the finish.

\subsection{Monitoring Modes During Runtime}
If debugging is deactivated or the verbose level is zero (see \refs{tranalyzer.h}), Tranalyzer2 prints no status information or end report.
Interrupt signal has been introduced to force intermediate status information to {\tt stdout}. Appropriate Unix commands and their effect are listed below:

\begin{longtable}{ll}
    \toprule
    {\bf Command} & {\bf Description}\\
    \midrule\endhead%
    {\tt kill --USR1 PID} & T2 sends configured monitoring report to stdout \\
    {\tt kill --USR2 PID} & T2 toggles between on demand and continuous monitoring operation \\
    \bottomrule
\end{longtable}
The script {\tt t2stat} has the same function as {\tt kill --USR1 PID}.
An example of a typical signal requested report ({\tt MACHINE\_REPORT=0}) is shown in \refl{fig:t2reportaggr}.

\begin{lstlisting}[
    caption = {A sample Tranalyzer2 human readable report aggregate mode},
    label   = fig:t2reportaggr,
]
                                    @      @
                                     |    |
===============================vVv==(a    a)==vVv===============================
=====================================\    /=====================================
======================================\  /======================================
                                       oo
USR1 A type report: Tranalyzer 0.8.2 (Anteater), Tarantula. PID: 16355
PCAP time: 1291753261.106203 sec (Tue 07 Dec 2010 20:21:01 GMT)
PCAP duration: 35.659471 sec
Time: 1546607860.487896 sec (Fri 04 Jan 2019 14:17:40 CET)
Elapsed time: 21.399345 sec
Total bytes to process: 42949673232 (42.95 G)
Percentage completed: 16.01%
Total bytes processed so far: 6875527168 (6.88 G)
Remaining time: 112.276935 sec (1m 52s)
ETF: 1546607972.764831 sec (Fri 04 Jan 2019 14:19:32 CET)
Number of processed packets: 8576716 (8.58 M)
Number of processed bytes: 6738299436 (6.74 G)
Number of raw bytes: 6740587316 (6.74 G)
Number of IPv4 packets: 8543638 (8.54 M) [99.61%]
Number of IPv6 packets: 33035 (33.03 K) [0.39%]
Number of A packets: 4916815 (4.92 M) [57.33%]
Number of B packets: 3659901 (3.66 M) [42.67%]
Number of A bytes: 2267361054 (2.27 G) [33.65%]
Number of B bytes: 4470938382 (4.47 G) [66.35%]
Average A packet load: 461.14
Average B packet load: 1221.60 (1.22 K)
--------------------------------------------------------------------------------
tcpFlags: Number of TCP scans, succ scans, retries: 9533, 4061, 2125
icmpDecode: Number of ICMP echo request packets: 2610 (2.61 K) [11.39%]
icmpDecode: Number of ICMP echo reply packets: 703 [3.07%]
dnsDecode: Number of DNS packets: 37156 (37.16 K) [0.43%]
dnsDecode: Number of DNS Q packets: 18735 (18.73 K) [50.42%]
dnsDecode: Number of DNS R packets: 18421 (18.42 K) [49.58%]
dnsDecode: Aggregated status: 0x0201
httpSniffer: Number of HTTP packets: 5833252 (5.83 M) [68.01%]
connStat: Max unique number of IP source connections: 61971 (61.97 K)
connStat: Max unique number of IP destination connections: 67895 (67.89 K)
connStat: Max unique number of IP source/destination connections: 712
connStat: Max unique number of source IP / destination port connections: 712
connStat: IP prtcon/sdcon, prtcon/scon: 1.000000, 0.011489
--------------------------------------------------------------------------------
Headers count: min: 4, max: 13, average: 6.95
Max VLAN header count: 1
Max MPLS header count: 2
Number of LLC packets: 43 [0.00%]
Number of GRE packets: 3677 (3.68 K) [0.04%]
Number of Teredo packets: 33001 (33.00 K) [0.38%]
Number of AYIYA packets: 34 [0.00%]
Number of IGMP packets: 58 [0.00%]
Number of ICMP packets: 20491 (20.49 K) [0.24%]
Number of ICMPv6 packets: 2426 (2.43 K) [0.03%]
Number of TCP packets: 6907067 (6.91 M) [80.53%]
Number of UDP packets: 1608944 (1.61 M) [18.76%]
Number of IPv4 fragmented packets: 2511 (2.51 K) [0.03%]
Number of IPv6 fragmented packets: 540 [1.63%]
~~~~~~~~~~~~~~~~~~~~~~~~~~~~~~~~~~~~~~~~~~~~~~~~~~~~~~~~~~~~~~~~~~~~~~~~~~~~~~~~
Number of processed   flows: 270826 (270.83 K)
Number of processed A flows: 228096 (228.10 K) [84.22%]
Number of processed B flows: 42730 (42.73 K) [15.78%]
Number of request     flows: 224453 (224.45 K) [82.88%]
Number of reply       flows: 46373 (46.37 K) [17.12%]
Total   A/B    flow asymmetry: 0.68
Total req/rply flow asymmetry: 0.66
Number of processed   packets/flows: 31.67
Number of processed A packets/flows: 21.56
Number of processed B packets/flows: 85.65
Number of processed total packets/s: 240517.21 (240.52 K)
Number of processed A+B packets/s: 240517.21 (240.52 K)
Number of processed A   packets/s: 137882.45 (137.88 K)
Number of processed   B packets/s: 102634.76 (102.63 K)
~~~~~~~~~~~~~~~~~~~~~~~~~~~~~~~~~~~~~~~~~~~~~~~~~~~~~~~~~~~~~~~~~~~~~~~~~~~~~~~~
Number of average processed flows/s: 7594.78 (7.59 K)
Average full raw bandwidth: 1512212608 b/s (1.51 Gb/s)
Average snapped bandwidth : 1511699328 b/s (1.51 Gb/s)
Average full bandwidth : 1511878912 b/s (1.51 Gb/s)
Fill size of main hash map: 236513 [90.22%]
Max number of flows in memory: 236517 (236.52 K) [90.22%]
Memory usage: 2.49 GB [3.69%]
Aggregate flow status: 0x01003872d298fb04
[WRN] L3 SnapLength < Length in IP header
[WRN] Consecutive duplicate IP ID
[WRN] IPv4/6 fragmentation header packet missing
[INF] IPv4
[INF] IPv6
[INF] IPv4/6 fragmentation
[INF] VLAN encapsulation
[INF] MPLS encapsulation
[INF] L2TP encapsulation
[INF] PPP/HDLC encapsulation
[INF] GRE encapsulation
[INF] AYIYA tunnel
[INF] Teredo tunnel
[INF] CAPWAP/LWAPP tunnel
[INF] SSDP/UPnP flows
[INF] Ethernet flows
[INF] SIP/RTP flows
[INF] Authentication Header (AH)
[INF] Encapsulating Security Payload (ESP)
[INF] TOR addresses
================================================================================
\end{lstlisting}

\refl{fig:t2machinereport} illustrates the output of the header line and subsequent data lines generated when {\tt MACHINE\_REPORT=1}.
\begin{lstlisting}[
    caption = {A sample Tranalyzer2 machine report aggregate mode},
    label   = fig:t2machinereport,
]
%RepTyp	Time	Dur	memUsg[KB]	fillSzHashMap	Flws	AFlws	BFlws	SIP	DIP	SDIP	Prts	Pkts	APkts	BPkts	V4Pkts	V6Pkts	VxPkts	Byts	AByts	BByts	ARPPkts	RARPPkts	ICMPPkts	EchoReq	EchoRep	DnsPkts	DnsQPkts	DnsRPkts	HttpPkts	FrgV4Pkts	FrgV6Pkts	Alrms	TCPScn	TCPSScn	TcpRtry	RawBndWdth	Fave	GlblWrn	ICMP	IGMP	TCP	UDP	SCTP	0x0042	0x00fe	0x0806	0x0800	0x86dd
USR1MR_A	1175364020.458606	103.877184	1856384		6456055	518079	360556	157523	92140	153825	765	1706	9453374	5485249	3967298	9452802	0	0	680258574	2271454049	3288136123	0	0	492113898	1244	0	0	0	3309	20599	0	0	26244	22467	14833	428175.000	2.230065	0x02030052	49211	102	8186447	1169051	0	284	0	0	9452802	0
USR1MR_A	1175364043.564798	126.983376	2030752		6945643	680946	462230	218716	118181	183651	1164	3626	13104550	7235588	5867866	13103799	0	0	944828393	3007472308	4847491640	0	066918	5110	1695	0	0	0	4738	28844	0	0	33600	32326	23344	494874.688	3.115120	0x02030052	66918	130	11364263	1611189	0	372	0	0	13103799	0
USR1MR_A	1175364072.842855	156.261433	2184732		7103403	860105	573862	286243	148357	217272	1164	3626	17716953	9443884	8271657	17715947	0	0	1278800701	3927484585	6829832070	0	086334	6700	2226	0	0	0	6572	38940	0	0	48853	45035	34063	550743.562	3.115120	0x02030052	86334	177	15394866	2158059	0	492	0	0	17715947	0
\end{lstlisting}

\subsubsection{Configuration for Monitoring Mode}
To enable monitoring mode, configure Tranalyzer as follows:

\begin{longtable}{llcll}
    \toprule
    \multicolumn{2}{c}{\bf main.h}       & \hspace{2cm} & \multicolumn{2}{c}{\bf tranalyzer.h}\\
    \midrule
    {\tt \#define MONINTPSYNC} & {\tt 1} & \hspace{2cm} & {\tt \#define VERBOSE}         & {\tt 0}\\
    {\tt \#define MONINTTMPCP} & {\tt 1} & \hspace{2cm} & {\tt \#define DIFF\_REPORT}    & {\tt 1}\\
    {\tt \#define MONINTTHRD } & {\tt 1} & \hspace{2cm} & {\tt \#define MACHINE\_REPORT} & {\tt 1}\\
    \bottomrule
\end{longtable}

\clearpage
The following plugins contribute to the output:
\begin{multicols}{2}
    \begin{itemize}
        \item \tranrefpl{basicStats}
        \item \tranrefpl{connStat}
        \item \tranrefpl{dnsBCmp}
        \item \tranrefpl{dnsDecode}
        \item \tranrefpl{httpBCmp}
        \item \tranrefpl{icmpDecode}
        \item \tranrefpl{protoStats}
    \end{itemize}
\end{multicols}

The generated output is illustrated in \reff{fig:t2machinereport}. The columns are as follows:
\begin{multicols}{3}
    \begin{enumerate}
        \item RepType
        \item Time
        \item Dur
        \item memUsage[KB]
        \item fillSzHashMap
        \item Flows
        \item AFlows
        \item BFlows
        \item SIP
        \item DIP
        \item SDIP
        \item Prts
        \item Pkts
        \item APkts
        \item BPkts
        \item V4Pkts
        \item V6Pkts
        \item VxPkts
        \item Byts
        \item AByts
        \item BByts
        \item ARPPkts
        \item RARPPkts
        \item ICMPPkts
        \item EchoReq
        \item EchoRep
        \item DnsPkts
        \item DnsQPkts
        \item DnsRPkts
        \item HttpPkts
        \item FrgV4Pkts
        \item FrgV6Pkts
        \item Alrms
        \item TCPScn
        \item TCPSScn
        \item TcpRtry
        \item RawBndWdth
        \item Fave
        \item GlblWrn
    \end{enumerate}
\end{multicols}

When capturing from a live interface, the following three columns are output (between {\tt Dur} and {\tt memUsage[KB]}):
\begin{multicols}{3}
    \begin{itemize}
        \item PktsRec
        \item PktsDrp
        \item IfDrp
    \end{itemize}
\end{multicols}

\subsubsection{Monitoring Mode to syslog}
In order to send monitoring info to a syslog server T2 must be configured in machine mode as indicated above.
Then the output has to be piped into the following script:

\begin{lstlisting}
t2 -D ... -w ... | awk -F"\t" `{ print "<25> ", strftime("%b %d %T"), "Monitoring: " $0; }' | \
        nc -u w.x.y.z 514
\end{lstlisting}

Netcat will send it to the syslog server at address {\tt w.x.y.z}.
Specific columns from the monitoring output can be selected in the awk script.

\subsubsection{RRD Graphing of Monitoring Output}
The monitoring output can be stored in a RRD database using the \tranref{rrdmonitor} script.
To start creating a RRD database, launch Tranalyzer2 (in monitoring mode) as follows:
\begin{center}
    {\tt ./tranalyzer2/src/tranalyzer -r file.pcap | ./scripts/rrdmonitor}
\end{center}
Or for monitoring from a live interface:
\begin{center}
    {\tt sudo ./tranalyzer2/src/tranalyzer -i eth0 | ./scripts/rrdmonitor}
\end{center}

Plots for the various fields can then be generated using the \tranref{rrdplot} script.
%\begin{verbatim}
%Usage:
%    ./rrdplot [OPTION...] [FIELD1=flows] [FIELD2]
%\end{verbatim}
To specify intervals, use s (seconds), m (minutes), h (hour), d (day), w (week), mo (month), y (year).
For example, to plot the data from the last two weeks, use {\tt --i 2w} or {\tt --s --2w}.

An example graph is depicted in \reff{fig:rrdplot}.

\begin{figure}[!ht]
    \centering
    \tranimg[width=.6\textwidth]{rrdplot}
    \caption{T2 monitoring using RRD}
    \label{fig:rrdplot}
\end{figure}

\subsection{Cancellation of the Sniffing Process}
Processing of a pcap file stops upon end of file. In case of live capture from an interface Tranalyzer2 stops upon {\tt CTRL+C} interrupt or a {\tt kill --9 PID} signal. The disconnection of the interface cable will stop Tranalyzer2 also after a timeout of 182 seconds. The console based {\tt CTRL+C} interrupt does not immediately terminate the program to avoid corrupted entries in the output files. It stops creating additional flows and finishes only currently active flows. Note that waiting the termination of active flow depends on the activity or the lifetime of a connection and can take a very long time. In order to mitigate that problem the user can issue the {\tt CTRL+C} for {\tt GI\_TERM\_THRESHOLD} times to immediately terminate the program.
